\documentclass{llncs}
\usepackage{llncsdoc}

%In case of a space crunch...
	%\usepackage[subtle]{savetrees}
	%\usepackage{times}
	\usepackage{a4,geometry}
%In case of a space surplus
	%\usepackage[parfill]{parskip}

\usepackage{graphicx}
\usepackage{amsmath,amssymb,amsthm,mathtools}
\usepackage{paralist}
\usepackage{bm}
\usepackage{xspace}
\usepackage{url}
\usepackage{fullpage, prettyref}
\usepackage{boxedminipage}
\usepackage{wrapfig}
\usepackage{ifthen}
\usepackage{color}
\usepackage{xcolor}
\usepackage{framed}
\usepackage{algorithm}
\usepackage[pagebackref,letterpaper=true,colorlinks=true,pdfpagemode=none,urlcolor=blue,linkcolor=blue,citecolor=violet,pdfstartview=FitH]{hyperref}
\usepackage{fullpage}
\usepackage[noend]{algpseudocode}
\usepackage{enumitem}
%\usepackage{times}
% For restating theorems in appendix
\usepackage{thmtools}
\usepackage{thm-restate,cleveref}
%Usage:
	%\begin{restatable}[Goldbach's conjecture]{thm}{goldbach}
	%\label{thm:goldbach}
	%Every even integer greater than 2 can be expressed as the sum of two primes.
	%\end{restatable}
	%Then type \goldbach* to recall.

\newtheorem{theorem}{Theorem}[section]
\newtheorem{conjecture}[theorem]{Conjecture}
\newtheorem{lemma}[theorem]{Lemma}
\newtheorem{claim}[theorem]{Claim}
\newtheorem{question}{Question}
\newtheorem{corollary}[theorem]{Corollary}
\newtheorem{definition}{Definition}
\newtheorem{proposition}[theorem]{Proposition}
\newtheorem{fact}[theorem]{Fact}
\newtheorem{example}[theorem]{Example}
\newtheorem{assumption}[theorem]{Assumption}
\newtheorem{observation}[theorem]{Observation}
\newtheorem{remark}{Remark}

\newcommand{\comment}[1]{\textit {\em \color{blue} \footnotesize[#1]}\marginpar{\tiny\textsc{\color{blue} To Do!}}}
\newcommand{\mcomment}[1]{\marginpar{\tiny\textsf{\color{red} #1}}}

\newcommand{\ignore}[1]{}

%% Calligraphic letters

\newcommand{\cA}{{\cal A}}
\newcommand{\cB}{\mathcal{B}}
\newcommand{\cC}{{\cal C}}
\newcommand{\cD}{\mathcal{D}}
\newcommand{\cE}{{\cal E}}
\newcommand{\cF}{\mathcal{F}}
\newcommand{\cG}{\mathcal{G}}
\newcommand{\cH}{{\cal H}}
\newcommand{\cI}{{\cal I}}
\newcommand{\cJ}{{\cal J}}
\newcommand{\cL}{{\cal L}}
\newcommand{\cM}{{\cal M}}
\newcommand{\cP}{\mathcal{P}}
\newcommand{\cQ}{\mathcal{Q}}
\newcommand{\cR}{{\cal R}}
\newcommand{\cS}{\mathcal{S}}
\newcommand{\cT}{{\cal T}}
\newcommand{\cU}{{\cal U}}
\newcommand{\cV}{{\cal V}}
\newcommand{\cX}{{\cal X}}


\newcommand{\R}{\mathbb R}
\newcommand{\N}{\mathbb N}
\newcommand{\F}{\mathbb F}
\newcommand{\Z}{{\mathbb Z}}
\newcommand{\eps}{\varepsilon}
\newcommand{\lam}{\lambda}
\newcommand{\sgn}{\mathrm{sgn}}
\newcommand{\poly}{\mathrm{poly}}
\newcommand{\polylog}{\mathrm{polylog}}
\newcommand{\littlesum}{\mathop{{\textstyle \sum}}}
\newcommand{\half}{{\textstyle \frac12}}
\newcommand{\la}{\langle}
\newcommand{\ra}{\rangle}
\newcommand{\wh}{\widehat}
\newcommand{\wt}{\widetilde}
\newcommand{\calE}{{\cal E}}
\newcommand{\calL}{{\cal L}}
\newcommand{\calF}{{\cal F}}
\newcommand{\calW}{{\cal W}}
\newcommand{\calH}{{\cal H}}
\newcommand{\calN}{{\cal N}}
\newcommand{\calO}{{\cal O}}
\newcommand{\calP}{{\cal P}}
\newcommand{\calV}{{\cal V}}
\newcommand{\calS}{{\cal S}}
\newcommand{\calT}{{\cal T}}
\newcommand{\calD}{{\cal D}}
\newcommand{\calC}{{\cal C}}
\newcommand{\calX}{{\cal X}}
\newcommand{\calY}{{\cal Y}}
\newcommand{\calZ}{{\cal Z}}
\newcommand{\calA}{{\cal A}}
\newcommand{\calB}{{\cal B}}
\newcommand{\calG}{{\cal G}}
\newcommand{\calI}{{\cal I}}
\newcommand{\calJ}{{\cal J}}
\newcommand{\calR}{{\cal R}}
\newcommand{\calK}{{\cal K}}
\newcommand{\calU}{{\cal U}}
\newcommand{\barx}{\overline{x}}
\newcommand{\bary}{\overline{y}}

\newcommand{\ba}{\boldsymbol{a}}
\newcommand{\bb}{\boldsymbol{b}}
\newcommand{\bp}{\boldsymbol{p}}
\newcommand{\bt}{\boldsymbol{t}}
\newcommand{\bv}{\boldsymbol{v}}
\newcommand{\bx}{\boldsymbol{x}}
\newcommand{\by}{\boldsymbol{y}}
\newcommand{\bz}{\boldsymbol{z}}
\newcommand{\br}{\boldsymbol{r}}
\newcommand{\bh}{\boldsymbol{h}}

\newcommand{\bA}{\boldsymbol{A}}
\newcommand{\bD}{\boldsymbol{D}}
\newcommand{\bG}{\boldsymbol{G}}

\newcommand{\bR}{\boldsymbol{R}}
\newcommand{\bS}{\boldsymbol{S}}
\newcommand{\bX}{\boldsymbol{X}}
\newcommand{\bY}{\boldsymbol{Y}}
\newcommand{\bZ}{\boldsymbol{Z}}

\newcommand{\NN}{\mathbb{N}}
\newcommand{\RR}{\mathbb{R}}

\newcommand{\abs}[1]{\left\lvert #1 \right\rvert}
\newcommand{\norm}[1]{\left\lVert #1 \right\rVert}
\newcommand{\ceil}[1]{\lceil#1\rceil}
\newcommand{\Exp}{\EX}
\newcommand{\floor}[1]{\lfloor#1\rfloor}

\newcommand{\EX}{\hbox{\bf E}}
\newcommand{\prob}{{\rm Prob}}

\newcommand{\gset}{Y}
\newcommand{\gcol}{{\cal Y}}

%% Hyper-linked References
\newcommand{\Sec}[1]{\hyperref[sec:#1]{\S\ref*{sec:#1}}} %section
\newcommand{\Eqn}[1]{\hyperref[eq:#1]{(\ref*{eq:#1})}} %equation
\newcommand{\Fig}[1]{\hyperref[fig:#1]{Fig.\,\ref*{fig:#1}}} %figure
\newcommand{\Tab}[1]{\hyperref[tab:#1]{Tab.\,\ref*{tab:#1}}} %table
\newcommand{\Thm}[1]{\hyperref[thm:#1]{Theorem\,\ref*{thm:#1}}} %theorem
\newcommand{\Fact}[1]{\hyperref[fact:#1]{Fact\,\ref*{fact:#1}}} %fact
\newcommand{\Lem}[1]{\hyperref[lem:#1]{Lemma\,\ref*{lem:#1}}} %lemma
\newcommand{\Prop}[1]{\hyperref[prop:#1]{Prop.~\ref*{prop:#1}}} %property
\newcommand{\Cor}[1]{\hyperref[cor:#1]{Corollary~\ref*{cor:#1}}} %corollary
\newcommand{\Conj}[1]{\hyperref[conj:#1]{Conjecture~\ref*{conj:#1}}} %conjecture
\newcommand{\Def}[1]{\hyperref[def:#1]{Definition~\ref*{def:#1}}} %definition
\newcommand{\Alg}[1]{\hyperref[alg:#1]{Alg.~\ref*{alg:#1}}} %algorithm
\newcommand{\Ex}[1]{\hyperref[ex:#1]{Ex.~\ref*{ex:#1}}} %example
\newcommand{\Clm}[1]{\hyperref[clm:#1]{Claim~\ref*{clm:#1}}} %example

%\usepackage{times}

\def\diam{\mathrm{diam}}
\def\dist{\mathrm{dist}}
\def\mckc{{\sffamily Heterogeneous Cap-$k$-Center}\xspace}
%\def\cckp{{\sffamily Cardinality-Constrained Knapsack Cover}\xspace}
\def\cckp{$Q|f_i|C_{min}$\xspace}
\def\opt{\mathsf{OPT}}
\def\capkc{{\sffamily Cap-$k$-Center }}
\def\x{{\mathsf x}}
\def\y{y^{{\mathsf{int}}}}
\def\z{\bar{z}}
\def\Supp{\mathsf{Supp}\xspace}
\def\n{n^{(1)}}
\def\nn{n^{(2)}}
\def\effc{c_{\mathrm{eff}}}
\def\zz{z^{\mathsf{int}}}
\def\suff{\mathsf{suff}}
\def\Sone{\calS_{\mathsf{round}}}
\def\Stwo{\calS_{\mathsf{nexp}}}

\def\Cb{C_{\mathsf{blue}}}
\def\Cbb{C_{\mathsf{black}}}
\def\Cr{C_{\mathsf{red}}}
\def\Cp{C_{\mathsf{purple}}}
\def\Cd{C_{\mathsf{del}}}

%\newcommand{\gset}{Y}
%\newcommand{\gcol}{{\cal Y}}
\newcommand{\brp}{{(p)}}
\renewcommand{\br}[1]{{(#1)}}
\newcommand{\bc}{{\bar c}}
\renewcommand{\epsilon}{\varepsilon}

\DeclareMathOperator*{\argmax}{arg\,max}
\DeclareMathOperator*{\argmin}{arg\,min}


\newcommand{\initOneLiners}{%
    \setlength{\itemsep}{0pt}
    \setlength{\parsep }{0pt}
    \setlength{\topsep }{0pt}
%
}
\newenvironment{oneLiners}[1][\ensuremath{\bullet}]
    {\begin{list}
        {#1}
        {\initOneLiners}}
    {\end{list}}



\begin{document}
\title{\huge The Heterogeneous Capacitated $k$-Center Problem}
\date{}
\author{Deeparnab Chakrabarty\thanks{Microsoft Research India, deeparnab@gmail.com} \and Ravishankar Krishnaswamy\thanks{Microsoft Research India, ravishankar.k@gmail.com} \and Amit Kumar\thanks{Comp. Sci. \& Engg., IIT Delhi, amitk@cse.iitd.ac.in}}
\institute{}
\maketitle
\begin{abstract}
	In this paper we initiate the study of the {\em heterogeneous capacitated $k$-center problem}: given a metric space $X = (F \cup C, d)$, and a collection of facilities $1, 2, \ldots, k$, with facility $i$ having a capacity of $c_i$. The goal is to open each facility $i$ facility at a location ${\rm loc}(i) \in F$, and also to assign clients to facilities using a map $\phi: C \rightarrow [k]$ so that the number of clients assigned to facility $i$ is at most $c_i$, and the objective is to minimize the maximum distance between a client and the location of its assigned facility, i.e., $\max_{c \in C} d(c,{\rm loc}(\phi(c)))$. If all the capacities $c_i$'s are identical, the problem is identical to the well-studied {\em uniform capacitated $k$-center problem} for which constant-factor approximations are known~\cite{XYZ}.
%However, the non-uniform generalization (called the {\em non-uniform capacitated $k$-center problem}) of this problem which has received much attention recently differs from our problem in the following sense: in the non-uniform problem, each facility location $f$ has a fixed capacity $c_f$, and the goal is to open $k$ centers and assign clients so that the capacity constraints are satisfied. On the other hand, in our {\em heterogeneous} problem, a set of $k$ different capacities is specified, and the goal is to suitably place them in facility locations.
The additional choice of determining which capacity should be installed in which location makes our problem technically substantially different from this problem, as well as the {\em non-uniform capacitated $k$-center problem} which has received much attention recently. In fact, one of our contributions is in relating the heterogeneous problem to a special-case of the classical {\em santa-claus problem}. Using this connection, we get the following results for heterogeneous capacitated $k$-center problem.
\begin{oneLiners}
\item A quasi-polynomial time $O(\log n/\epsilon)$-approximation where every capacity is violated by a factor of $O(1+\epsilon)$.
\item A polynomial time $O(1)$-approximation where every capacity is violated by an $O(\log n)$ factor.
\end{oneLiners}
We get improved results for the {\em soft-capacities} version where we can place multiple facilities in the same location, i.e., ${\rm loc}(\cdot)$ need not be injective.
\end{abstract}
%\thispagestyle{empty}
%\newpage
%\setcounter{page}{1}
\input{section1}
%\newpage
\section{Preliminaries}\label{sec:prelims}
Given an \mckc instance, we start by guessing $\opt$. We either prove $\opt$ is infeasible, or find an $(a,b)$-approximate allocation of clients to facilities.
	We define the bipartite graph $G = (F\cup C,E)$ where $(i,j)\in E$ iff $d(i,j) \leq \opt$. If $\opt$ is feasible, then the following assignment LP\eqref{eq:lp1}-\eqref{eq:lp6}
	must have a feasible solution.
In this LP, we  have opening  variables $y_{ip}$ for every $i\in F,p\in [P]$ indicating whether we open a facility with capacity $c_p$ at location $i$. Recall that the capacities available to us are $c_1, c_2, \ldots, c_P$ -- a facility with
capacity $c_p$ installed on it will be referred to as a {\em type $p$ facility.}
	We have connection variables $x_{ijp}$ indicating the fraction to which client $j\in C$ connects to a facility at location $i$ where a type $p$ facility has been opened.
	We force $x_{ijp} = 0$ for all pairs $i,j$ and type $p$ such that  $d(i,j) > \opt$.
	
		
		\begin{minipage}{0.49\textwidth}
			\begin{alignat*}{2}
				 \forall j\in C,   &\textstyle \sum_{i\in F} \sum_{p\in [P]}  x_{ijp} \geq 1 \label{eq:lp1} \tag{\small{L1}}  \\
				 \forall i\in F,p\in [P] ,  &\textstyle \sum_{j\in C}  x_{ijp} \leq c_py_{ip} \label{eq:lp2} \tag{\small{L2}} \\
				 \forall p\in [P], & \textstyle \sum_{i\in F} y_{iq}   \leq k_p \label{eq:lp3}  \tag{\small{L3}}
			\end{alignat*}
		\end{minipage}
		~\vline~
		\begin{minipage}{0.49\textwidth}
			\begin{alignat*}{3}
				&\forall i\in F, j\in C,p\in [P],  & x_{ijp} \leq y_{ip}\label{eq:lp4}   \tag{\small{L4}} \\
				& \forall i\in F, & \textstyle\sum_{p\in [P]} y_{ip} \leq 1 \label{eq:lp5}  \tag{\small{L5}} \\
				& \forall i\in F,j\in C,p\in [P], & x_{ijp},y_{ip} \geq 0\label{eq:lp6}\tag{\small{L6}}
			\end{alignat*}
		\end{minipage}
\smallskip

\noindent		
We say a solution $(x,y)$ is $(a,b)$-feasible if it satisfies \eqref{eq:lp1}, \eqref{eq:lp3}-\eqref{eq:lp6}, and \eqref{eq:lp2} with the RHS replaced by $bc_py^\mathsf{int}_{ip}$, and $x_{ijp} > 0$ only if $d(i,j) \leq a\cdot \opt$,
We desire to find an integral solution $(x^\mathsf{int},y^\mathsf{int})$ which is $(a,b)$-feasible.
The following claim shows that it suffices just to round the $y$-variables.
\begin{claim}
Given an $(a,b)$-feasible solution $(x,\y)$ where $\y_{ip}\in \{0,1\}$,
we can get  an $(a,b)$-approximate solution to the \mckc problem.
\end{claim}
\iffalse
\begin{proof}
Consider a bipartite graph with client nodes $C$ on one side, and nodes of the form $(i,p)$ with $\y_{ip} = 1$ on the other. The node $(i,p)$ has capacity $bc_p$.
Since $(x,\y)$ satisfies the conditions of the lemma, there is a fractional matching in this graph so that each client $j$  is fractionally matched to an $(i,p)$ so that $d(i,j)\leq a\cdot \opt$,
and the total fractional load on $(i,p)$ is $\leq bc_p$. The theory of matching tells us that there is an {\em integral} assignment of clients $j$ to nodes $(i,p)$ such that $d(i,j)\leq a\cdot\opt$
and the number of nodes matched to $(i,p)$ is $\leq \ceil{bc_p}$. Therefore opening a capacity $c_p$ facility at $i$ for all $(i,p)$ with $\y_{ip} = 1$ gives an $(a,b)$-approximate solution to \mckc.
\end{proof}
\fi

We remark that as is, the LP has an unbounded integrality gap for \mckc, and indeed, the gap instances also happen to be of the \cckp variety. So we strengthen it by adding some additional constraints which we explain later. However, since our strong decomposition theorem merely uses these $y_{ip}$ and $x_{ijp}$ values, we now present that first. %Henceforth, we focus on rounding the $y$-values. To this end, we make the following useful definition.
\begin{definition}[Roundable Sets]\label{def:rnding-mkc}
	A set of facilities $S\subseteq F$ is said to be $(a,b)$-roundable w.r.t $(x,y)$ if
	\begin{itemize}[noitemsep]
		\item[(a)] $\diam_G(S) \leq a$
		\item[(b)] there exists a rounding $\y_{ip} \in \{0,1\}$ for all $i \in S, p\in [P]$ such that
		\begin{enumerate}
			\item $\sum_{q \geq p} \sum_{i\in S} \y_{iq} ~\leq~ \floor{\sum_{q \geq p}\sum_{i\in S} y_{iq}}$ for all $p$, and
			\item $\sum_{j\in C} d_j \sum_{i\in S,p\in [P]} x_{ijp} \leq b\cdot \sum_{i\in S} \sum_{p\in [P]} c_p \y_{ip}$
		\end{enumerate}
	\end{itemize}
\end{definition}
\noindent

\iffalse
If $(x,y)$ were feasible, then for any $(a,b)$-roundable set, we can integrally open facilities to satisfy all the demand that was fractionally assigned to it taking a hit of $a$ in the cost and a factor of $b$ in the capacities. Furthermore, the number of open facilities is at most what the LP prescribes. Therefore, if we would be able to decompose the instance into roundable sets, we would be done.
Unfortunately, that is not possible, and in fact the above LP has a large integrality gap even when we allow arbitrary violation of capacities.

\begin{remark}[Integrality Gap for \mckc] \label{rem:ig}
Consider the following instance. The metric space $X$ is partitioned into $(F_1\cup C_1) \cup \cdots \cup (F_K\cup C_K)$, with $|F_k| = 2$ and $|C_k| = K$ for all $1\le k\le K$.
The distance between any two points in $F_i\cup C_i$ is $1$ for all $i$, while all other distances are $\infty$. The capacities available are $k_1 = K$ facilities with capacity $c_1 = 1$ and
$k_2= K-1$ facilities with capacity $c_2 = K$. It is easy to see that integrally any solution would violate capacities by a factor of $K/2$.
%It is easy to see that the above instance is not feasible with $OPT=1$: indeed, there is at least one client location where the optimal solution does not place a facility of capacity $H$ in its neighborhood, and it is not possible to serve the demand of this client using only capacity $1$ facilities, as there are only two locations where we can place facilities in its neighborhood.
On the other hand, there is a feasible solution for the above LP relaxation: for $F_k = \{a_k,b_k\}$, we set $y_{a_k2} = 1-1/K$ and $y_{b_k1} = 1$, and for all $j\in C_k$, we set $x_{a_kj2} = 1-1/K$ and $x_{b_kj1} = 1/K$.


 For the version with soft capacities, we do not have the constraint \eqref{eq:lp5} and the above integrality gap doesn't hold since we can install capacity $K$ facilities on $K-1$ of the sets $F_k$'s, $1\leq k\leq K-1$, and $K$ copies of the capacity $1$ facilities at $F_K$. Note that although $|F_K| = 2$, we have opened $K$ capacities.
\end{remark}

In particular, note that for the $(x,y)$ solution in the integrality gap example above there are no roundable sets. This motivates the definition of the second kind of sets.
\fi

So if we can partition the facilities into roundable sets with reasonable parameters, we would be done.   It turns out that sets which are not roundable have a \emph{non-expanding structure}, and indeed we define our \cckp instance over such sets. The following definition comes handy in this case.

\begin{definition}[Complete Neighborhood Sets] \label{def:comp-nbr}
	A subset $T\subseteq F$ of facilities is called a {\em complete neighborhood} if there exist clients $J\subseteq C$ such that $\Gamma(J) \subseteq T$.
	In this case $J$ is said to be {\em responsible} for $T$. Additionally, a complete neighborhood $T$ is said to be an $\alpha$-complete neighborhood if $\diam(T) \leq \alpha$.
\end{definition}

%\begin{remark}[Complete Neighborhood Sets to \cckp]\label{rem:red}
%	\emph{
If we find a complete neighborhood $T$ of facilities with a set $J$ of clients responsible for it, then we know that the optimal solution \emph{must satisfy} all the demand in $J$ by suitably opening facilities of sufficient capacity in $T$.
Thus, if we can partition the entire instance into a collection $\calT = (T_1,\ldots,T_m)$ of disjoint $\alpha$-complete neighborhood sets with $J_i$ responsible for $T_i$, we can define an instance $\calI$ of \cckp with $m$ machines with demands $D_i = |J_i|$ and cardinality constraint $f_i = |T_i|$, and there are $n_i$ jobs of capacities $c_i$ for $1 \leq i \leq P$.

However, in general, our decomposition theorem only ensures that we can partition the instance into sets which are either roundable or are complete neighborhoods, and the crux of the rounding lies in combining the two cases without opening more facilities of any type. Our final definition to describe our decomposition is that of $(\tau,\rho)$-{\em deletable clients} that can be removed from the instance since they can be ``$\rho$-charged'' to the remaining clients that are at most $\tau$-away.
\begin{definition}[Deletable Clients]\label{def:deletable}
	A subset $\Cd\subseteq C$ of clients is $\rho$-deletable if there exists a mapping $\phi_{j,j'}\in [0,1]$ for $j\in \Cd$ and $j'\in C\setminus \Cd$ satisfying (a) $\sum_{j'\in C\setminus \Cd} \phi_{j,j'} = 1$ for all $j\in \Cd$, and(b) $\sum_{j\in \Cd} \phi_{j,j'} \leq \rho$ for all $j'\in C\setminus \Cd$. Furthermore, $\phi_{j,j'} > 0$ only if $d(j,j') \leq \tau\cdot\opt$.



%The facilities opened by the $\opt $ solution corresponds to a valid solution for $\calI$; furthermore, any $\beta$-approximate solution for $\calI$ corresponds to a $(\alpha,\beta)$-approximate solution for the \mckc problem restricted to clients in $\cup_{\ell} J_\ell$. Finally note that for \mckc with soft-capacities, $\calI$ is an instance of the $Q||C_{min}$ problem.
%}

%\emph{
%Note that the above integrality gap  example is essentially a \cckp instance with $K$ machines of demand $K$ each having cardinality constraint $2$, and there are $K$ jobs of capacity $1$ and $K-1$ jobs with capacity $K$. This shows the assignment LP has bad integrality gap for the \cckp problem (but not for $Q||C_{min}$).}
%\end{remark}



\end{definition}
\iffalse
 The following claim shows we can remove $\Cd$ from consideration.
\begin{claim}\label{clm:prelim3}
	Let $\Cd$ be a $(\rho,\tau)$-deletable set.
	Given an $(a,b)$-approximate feasible solution $(x',\y)$ where $x'_{ijp}$ is defined only for $j\in C\setminus \Cd$, we can extend $x'$ to a general $(x,\y)$ solution
	which is $(a+\tau, b(1+\rho))$-approximate feasible.
\end{claim}
\begin{proof}
For any $j\in \Cd$, define $x_{ijp} = \sum_{j'\in C\setminus \Cd} x_{ij'p}\phi_{j,j'}$.
We get for all $j\in \Cd$,
$\textstyle \sum_{i\in F} \sum_{p\in [P]} x_{ijp} = \sum_{i,p} \sum_{j'\in C\setminus \Cd} x_{ij'p}\phi_{j,j'} = \sum_{j'\in C\setminus \Cd} \phi_{j,j'} \left(\sum_{i,p} x_{ij'p}\right) \geq \sum_{j'\in C\setminus \Cd} \phi_{j,j'} = 1$,
and for all $i\in F,p\in [P]$,
$\textstyle \sum_{j\in \Cd}  x_{ijp} = \sum_{j\in \Cd} \sum_{j'\in C\setminus \Cd} x_{ij'p}\phi_{j,j'} = \sum_{j'\in C\setminus \Cd} x_{ij'p}\left( \sum_{j\in \Cd} \phi_{j,j'}\right)  \leq \rho \sum_{j'\in C\setminus \Cd} x_{ijp} \leq b\rho c_p$. Therefore, in all we have $\sum_{j\in C} x_{ijp} \leq bc_p(1+\rho)$.
\end{proof}
\fi


Our final ingredient is that of supply polyhedra. Recall that an instance of  \cckp has $m$ machines $M$ with demands $D_1,\ldots,D_m$ and cardinality constraints $f_1,\ldots, f_m$, and $n$  types of jobs $J$ with capacities $c_1,\ldots,c_n$ respectively.
%In $Q||c_{min}$, there are no $f_i$'s, or equivalently $f_i = \infty$.
%In the version with cardinality constraints, that is $Q|f_i|C_{min}$ we are also given positive integers $f_1,\ldots, f_m$.
 A {\em supply vector} $(s_1,\ldots,s_n)$ where each $s_j$ is a non-negative integer
is called {\em feasible} for this instance if the ensemble formed by $s_j$ copies of jobs of capacity $c_j$ can be allocated feasibly to satisfy all the demands.
The {\em supply polyhedra} desires to capture these feasible supply vectors.

\begin{definition}[Supply Polyhedron]\label{def:supp-poly}
	Given an instance $\calI$ for a max-min allocation problem, a polyhedron $\calP(\calI)$ is called an $\alpha$-approximate supply polyhedron if
	(a) all feasible supply vectors lie in $\calP(\calI)$, and (b) given any non-negative integer vector $(s_1,\ldots,s_n)\in \calP(\calI)$ there exists an assignment
	of the $s_j$ jobs of capacity $c_j$ to the machines such that machine $i$ receives capacity $\geq D_i/\alpha$.
\end{definition}

Ideally, we would like {\em exactly} supply polyhedra, and one choice would be the convex hull of all the feasible supply vectors; indeed this is the tightest polytope satisfying condition (a).
Unfortunately, there are instances of \cckp where the convex hull contains infeasible integer points for which $\alpha = \Theta(\log n/\log \log n)$.


\section{\mckc via Supply Polyhedra}\label{sec:o1}
\def\yy{y^\calT}
In this section, we prove the following theorem. %One of the main engines will be our strong decomposition theorem (\Cref{thm:decomp}) which we will state here but will prove in the next section.
\begin{theorem}\label{thm:reduction}
Suppose there exists $\beta$-approximate supply polyhedra  for all instances of $Q|f_i|C_{min}$ (resp., $Q||C_{min}$) which have $\gamma$-approximate separation oracles.
Then for any $\delta\in(0,1)$, there is an $\left(\tilde{O}(1/\delta),\gamma\beta(1+5\delta)\right)$-bicriteria approximation algorithm for  \mckc (resp., with soft capacities).
\end{theorem}
\noindent
	
\noindent
\noindent
%
%As discussed in the Introduction, for the \mckc problem, the above LP has unbounded integrality gap (although, as we will see later, with soft capacities this is not the case).
%To obtain non-trivial algorithms we would need to strengthen the LP. Nevertheless, a feasible solution $(x,y)$ to LP~\eqref{eq:lp1}-\eqref{eq:lp6} gives us a way to decompose the problem
%into ``easily roundable parts'' and ``\cckp parts". To formalize this, we make two definitions.
%
%\begin{definition}\label{def:rnding-mkc}
%	A set of facilities $S\subseteq F$ is said to be $(a,b)$-roundable w.r.t feasible solution $(x,y)$ if
%	\begin{itemize}[noitemsep]
%		\item[(a)] $\diam_G(S) \leq a$
%		\item[(b)] there exists a rounding $Y_{ip} \in \{0,1\}$ for all $i \in S, p\in [P]$ such that
%		\begin{enumerate}
%			\item $\sum_{q \geq p} \sum_{i\in S} Y_{iq} ~\leq~ \floor{\frac{1}{2} \cdot \sum_{q \geq p}\sum_{i\in S} y_{iq}}$ for all $p$, and
%			\item $\sum_{j\in C} d_j \sum_{i\in S,p\in [P]} x_{ijp} \leq b\cdot \sum_{i\in S} \sum_{p\in [P]} c_p Y_{ip}$
%		\end{enumerate}
%	\end{itemize}
%\end{definition}
%\noindent
%For any $(a,b)$-roundable set, we can integrally open facilities to satisfy all the demand that was fractionally assigned to it taking a hit of $a$ in the cost and a factor of $b$ in the capacities.
%Furthermore, the number of open facilities is at most what the LP prescribes.
%%The idea behind the above definition should be clear.
%%Suppose $S$ is a set of facilities and consider a subset of demands $J$ which are being assigned (fractionally) by
%%the solution $(x,y)$ to $S$. If $S$ is $(a,b)$-roundable wrt $(x,y)$, then we can open facilities in $S$ integrally and reassign the demands in $J$ to these integral
%%facilities. By doing so, we shall increase the connection cost of demands in $J$ by an additive factor of $a$, and then we may violate the capacities of the integrally open
%%facilities by at most a factor $b$.
%\begin{definition} \label{def:comp-nbr}
%	A subset $S\subseteq F$ of facilities is called a {\em complete neighborhood} if there exists a client-set $J\subseteq C$ such that $\Gamma(J) \subseteq S$.
%	In this case the subset $J$ is said to be {\em responsible} for $S$. Additionally, a complete neighborhood $S$ is said to be an $a$-complete neighborhood if $\diam(S) \leq a$.
%\end{definition}
%\noindent
%%In a complete neighborhood set $S$ with $J$ responsible for it, we must open a total capacity of at least $|J|$ units among facilities in $S$.
%%A collection of disjoint complete neighborhood sets therefore correspond to a \cckp instance. If the diameter of each set is $\leq \alpha$, then $\beta$-approximate solutions to the
%%\cckp instance correspond to $(\alpha,\beta)$-bicriteria approximation.
%If we find a complete neighborhood $S$ of facilities with say a set $J$ of clients responsible for it, then we know that the optimal solution must satisfy all the demand in $J$ by suitably opening facilities of sufficient capacity in $S$. A disjoint collection of such complete neighborhoods leads to a \cckp problem instance. % described in Idea 1.
%\smallskip
%
%\noindent
Our results for \mckc follow from \Cref{thm:reduction} and results about supply polyhedra.
For example, Theorem~\ref{thm:2} follows from~\Cref{thm:reduction} (using $\delta=0.5$, say) and Theorem~\ref{thm:conflp}, and also noting that $D_\mathrm{max}/D_\mathsf{min} \leq n$ in our reduction. %Theorem~\ref{thm:2a} follows
%from~\Cref{thm:reduction,thm:asslp}.
We now state our decomposition result using which we prove~\Cref{thm:reduction}. %which essentially states that given an \mckc instance, we can partition the problem into roundable and complete neighborhood sets. The reader may want to recall the definitions of roundable sets (Definition~\ref{def:rnding-mkc}), complete neighborhood sets (Definition~\ref{def:comp-nbr}), deletable sets (Definition~\ref{def:deletable}), and the natural LP relaxation \eqref{eq:lp1}-\eqref{eq:lp6}.
%Our main techincal theorem is the following
%It is perhaps instructive to compare the below theorem with Theorem~\ref{thm:weakdecomp}.
%The proof of this theorem is rather technical, and we defer it to the next section.
\begin{theorem}[{\bf Decomposition Theorem}]\label{thm:decomp}
	Given a feasible solution $(x,y)$ to LP\eqref{eq:lp1}-\eqref{eq:lp6}, and $\delta > 0$, there is a polynomial time algorithm which finds a solution $\x$ satisfying \eqref{eq:lp2} and\eqref{eq:lp4}, and a
	decomposition as follows.
	\begin{enumerate}%[noitemsep]
		\item The facility set $F$ is partitioned into two families $\calS = (S_1, S_2, \ldots, S_K)$ and $\calT = (T_1, T_2, \ldots, T_L)$ of mutually disjoint subsets.
		The client set $C$ is partitioned into three disjoint subsets $C = \Cd \cup \Cbb \cup \Cb$ where $\Cd$ is a $(\tilde{O}(1/\delta),\delta)$-deletable subset.
		
			\item Each $S_k \in \calS$ is $(\tilde{O}(1/\delta),(1+\delta))$-roundable with respect to $(\x,y)$, and moreover, each client in $\Cb$ satisfies $\sum_{i \in \calS, p} \x_{ijp} \geq 1 - \frac{\delta}{100}$.
%\comment{deepc: maybe we should make 100 less arbitrary}
		\item Each $T_\ell$ is a $\tilde{O}(1/\delta)$-complete neighborhood with a corresponding set $J_\ell$ of clients responsible for it, and $\Cbb = \cup_{\ell = 1}^L J_\ell$.	
%		
%		
%		\item For the deleted clients $\Cd$, there is a mapping $\phi:\Cd \to \Cbb \cup \Cb$ such that
%		(a) $d(j,\phi(j)) \leq \tilde{O}(1/\delta)$ for all $j\in \Cd$, and
%		(b)	for all $j\in \Cbb \cup \Cb$, we have $\sum_{j' \in \Cd: \phi(j') = j} d_{j'} \leq (1+\delta)\cdot d_j$, i.e., the total demand mapped to $j$ is small.
%		
	\end{enumerate}
\end{theorem}


\begin{proof}[{\bf Proof of Theorem~\ref{thm:reduction}}]

Let us first describe an approach which fails. Let $(x,y)$ be a feasible solution to LP\eqref{eq:lp1}-\eqref{eq:lp6}, and apply~\Cref{thm:decomp}.
Although the sets in $\calS$ by definition are roundable which takes care of the clients in $\Cb$, the issue arises in assigning clients of $\Cbb$.
In particular, $\yy_p := \sum_{i\in \calT} y_{ip}$ for all $1\le p\le P$ which indicates the ``supply" of capacity $c_p$ available for the $\Cbb$ clients.
%Since the LP doesn't know $\calT$ beforehand, this supply mayn't be enough.
However, this may not be enough for serving all these clients (even with violation).
That is, the vector $\yy$ may not lie in the (approximate) supply polyhedra of the \cckp instance
defined by $\calT$.% as described in Remark~\ref{rem:red}.
 That we fail is not surprising; after all, we have so far only used the natural LP which has a bad integrality gap. To resolve this issue, we strengthen the LP by {\em explicitly requiring $\yy$ to be in the supply polyhedra}. Since we do not know $\calT$ before solving the LP (after all our LP rounding generated it), we enforce this for {\em all} collections of complete-neighborhood sets.
More precisely, for $\calT := (T_1,\ldots,T_L)$ of $L$ disjoint complete neighborhood sets, let $\calI_\calT$ denote the associated \cckp instance.
\begin{equation}\label{eq:lp7}
\forall \calT := (T_1,\ldots,T_L) \textrm{ disjoint neighborhood subsets}, \quad \yy \in \calP(\calI_\calT) \tag{\small{L7}}
\end{equation}
Note that this is a feasible constraint to add to LP\eqref{eq:lp1}-\eqref{eq:lp6}. In the $\opt$ solution, for any $\calT$ there must be enough supply dedicated for the clients responsible for these complete neighborhood sets.  We don't know how (and don't expect) to check feasibility of  \eqref{eq:lp7} for all collections $\calT$. However, we can still run ellipsoid method using the ``round-and-cut'' framework of \cite{CarrFLP00,ChakrabartyCKK11,Li15,Li16}.
To begin with, we start with the LP\eqref{eq:lp1}-\eqref{eq:lp6} and obtain feasible solution $(x,y)$. Subsequently, we apply the decomposition Theorem~\ref{thm:decomp} to obtain the collection $\calT = (T_1,\ldots,T_L)$.
We then check if $\yy \in \calP(\calI_\calT)$ or not. Since we have a $\gamma$-approximate separation oracle for $\calP(\calI_\calT)$, we  are either guaranteed that $\yy \in \calP(\calI'_\calT)$ where the $\ell^{th}$ demand is now  $D_\ell/\gamma$; or we get a hyperplane separating $\yy$ from $\calP(\calI_\calT)$ which also gives us a
hyperplane separating $y$ from  LP\eqref{eq:lp1}-\eqref{eq:lp7}. This can be fed to the ellipsoid algorithm to obtain a new  $(x,y)$ and the above process is repeated.

When this process stops, we will have a solution $(x,y)$ such that the supply $\{y_p^{\calT}\}$ lies in the supply polyhedra $\calP(\calI_\calT)$. So our overall algorithm is to simply round the roundable sets (by rounding down), and solve the instance $\calI_\calT$ with the supply vector $\{y_p^\calT\}$ using a suitable \cckp algorithm.
\end{proof}

We end the main body by noting that the configuration LP relaxation is in fact nearly the best possible supply polyhedra for \cckp.

\begin{theorem}\label{thm:conflp}
	For any instance $\calI$ of \cckp, the natural configuration LP for $\calI$ is an $O(\log D)$-approximate supply polyhedron with $(1+\epsilon)$-approximate separation oracle for any $\eps > 0$,
	where $D := D_{\mathrm{max}}/D_\mathrm{min}$. Moreover, there exists no supply polyhedra with approximation $o(\log D/\log \log D)$.
%	Given $(s_1,\ldots,s_n) \in \calP_\mathsf{conf}$ for an instance $\calI$ of $Q|f_i|C_{min}$, there is an of assignment of the $s_j$ jobs of capacity $c_j$  to the machines such that for all $i\in M$
%	receives a total capapcity $\geq D_i/\alpha$ for $\alpha = O(\log D)$ where $D = D_{max}/D_{min}$.
\end{theorem}


\iffalse
The analysis of the ellipsoid algorithm\comment{deepc: need to be careful and correct here}
tells us that in polynomial time we either prove infeasibility of the system \eqref{eq:lp1}-\eqref{eq:lp7} (implying the $\opt$  guess for \mckc is infeasible), or we
are have $(x,y)$ satisfying the premise of the following lemma.
%obtain a solution $(x,y)$
%which is feasible for LP\eqref{eq:lp1}-\eqref{eq:lp6}, such that for the $\calS, \calT$ obtained via Decomposition Theorem~\ref{thm:decomp}
%and the instance $\calI_\calT$ so obtained, the solution  $\yy \in \calP(\calI'_\calT)$ for the $\gamma$-shaded instance.

\begin{lemma}
	Given $(x,y)$ feasible for  LP\eqref{eq:lp1}-\eqref{eq:lp6}, let us apply the Decomposition Theorem~\ref{thm:decomp} to obtain the instance $\calS,\calT$.
	Suppose the solution $\yy_p := \sum_{i\in\calT} y_{ip}$ lies in $\calP(\calI'_\calT)$ for the \cckp (respectively, $Q||C_{min}$) instance $\calI'_\calT$ with $L$ machines with $D_\ell := |J_\ell|/\gamma$
	and $f_\ell := |T_\ell|$ (respectively, no cardinality constraints). Then we can obtain an $(\tilde{O}(1/\delta), \beta\gamma(1+\delta))$-approximate solution to the \mckc problem (respectively, with soft-capacities).
\end{lemma}
\begin{proof}
%Suppose the latter occurs. From this we can obtain the bicriteria solution to the \mckc problem as follows.
%Recall the notation in the Decomposition Theorem~\ref{thm:decomp}.
Since every set $S_k, 1\leq k\leq K,$ is $(\tilde{O}(1/\delta),(1+\delta))$-roundable, there exists a rounding $\y_{ip}$ for $i\in S_k$  such that
\begin{equation}\label{eq:repeat}
\textstyle \forall p, ~~~ \sum_{q\geq p} \sum_{i\in S_k} \y_{iq} \leq \floor{\sum_{q\geq p} \sum_{i\in S_k} y_{iq}}
\end{equation}
\def\s{\tilde{s}}
Ideally, we would like to open a facility of capacity $c_p$ at location $i$ whenever $\y_{ip} = 1$. Unfortunately, the decomposition theorem doesn't have capacity constraints for individual $p$'s but only their suffix sums. Instead we do the following. Define $y^{\calS}_p := \sum_{i\in \calS} y_{ip}$; LP\eqref{eq:lp3} implies that for all $p$, $y^\calS_p + y^\calT_p \leq k_p$.
For $1\leq p\leq P$, define $s_p := \sum_{i\in \calS} \y_{ip}$; \eqref{eq:repeat} implies for all $p$, $\sum_{q\geq p} s_q \leq \floor{\sum_{q\geq p} y^\calS_q}$ (since $\floor{a} + \floor{b} \leq \floor{a+b}$.)
\begin{claim}\label{clm:massmovement}
	Given $(s_1,\ldots,s_P)$ satisfying $\sum_{q\geq p} s_q \leq \floor{\sum_{q\geq p} y^\calS_q}$, there exists $(\s_1,\ldots,\s_P)$ satisfying
	for all $p$, (a) $\sum_{q\geq p} s_q \leq \sum_{q\geq p} \s_q \leq \sum_{q\leq p} y^\calS_q$, and (b) $\s_p \leq k_p$.
\end{claim}
\begin{proof}
Simply define $\s_p := \floor{\sum_{q\geq p} y^\calS_q} - \floor{\sum_{q> p} y^\calS_q}$. Therefore, $\sum_{q\geq p} \s_q = \floor{\sum_{q\geq p} y^\calS_q}$ implying (a).
To see (b), note $\s_p \leq \ceil{y^\calS_p} \leq k_p$, where we use the fact $\floor{a+b} \leq \floor{a} + \ceil{b}$ for any non-negative $a,b$.
\end{proof}

The first inequality in (a) implies that at every location with $\y_{ip} = 1$, we can open a facility of capacity $c_q \geq c_p$. This, along with condition (b) of roundable sets (Definition~\ref{def:rnding-mkc}),
implies we can find a fractional solution $x_{ijp}$ for $j\in \Cb$ and $(i,p)$ with $\y_{ip} = 1$
such that (a) $\sum_{i\in \calS, p\in [P]} x_{ijp} \geq 1$, (b) $x_{ijp} > 0$ only if $d(i,j) \leq \diam(S_k) \leq \tilde{O}(1/\delta)$, and (c) the capacity violation is $\leq (1+\delta)(1 - \delta/100)^{-1} \leq (1+2\delta)$. Note the second term arises since
from the decomposition theorem we have $\sum_{i\in \calS, p\in [P]} \x_{ijp} \geq 1-\delta/100$.
Thus we have fractionally assigned all $\Cb$ clients to open facilities in $\calS$.
%For all $j\in \Cb$, we assign it fractionally using $\frac{\x}{1-\delta/100}$ to


Define, for $p\in [P]$,  $t_p := k_p - \s_p$, the number of facilities of capacity $c_p$ we can open in $\calT$. Note, by Claim~\ref{clm:massmovement}, $t_p$'s are non-negative.
%
%
\begin{claim}
	$(t_1,\ldots,t_P) \in \calP(\calI'_\calT)$
\end{claim}
%	We now use the upward-feasibility property of $\calP$ to prove that the integral vector $(k^{(T)}_1,\ldots, k^{(T)}_P)$ also lies in $\calP(\calI'_\calT)$.
\begin{proof}
	By the Lemma premise, we have $\yy \in \calP(\calI'_\calT)$. %Using the upward-feasibility property of $\calP$, we get the following.
	%Recall that $\calI'_\calT$ has $L$ machines with demands $D_\ell := |J_\ell|/\beta$ and $f_\ell := |T_\ell|$.
	Now note that for all $p$,
	\[
\textstyle 	\sum_{q\geq p} t_q = \sum_{q\ge p} (k_q - \s_q) \ge \sum_{q\ge p} k_q - \sum_{q\geq p} y^\calS_q \geq  \sum_{q\geq p} y^\calT_q
	\]
	Since $\calP(\calI'_\calT)$ is upward-feasible, and $\yy\in \calP$, we get the claim.
\end{proof}
Since $\calP(\calI'_\calT)$ is $\beta$-approximate, we can find an allocation of the $t_p$ copies of jobs of capacity $c_p$ to the $L$ machines of $\calI'_\calT$ such that machine $\ell$ gets at most $f_\ell$ jobs and total capacity $\geq D_\ell/\beta = |J_\ell|/\beta\gamma$. We install these capacities on the facilities of $T_\ell$. Since the diameter of each $T_\ell$ is $\tilde{O}(1/\delta)$, we can find an $x_{ijp}$ assignment of $\Cbb$-clients  to these such that $\sum_{i\in \calT,p\in [P]} x_{ijp} \geq 1$ and $x_{ijp} > 0$ iff $d(i,j) = \tilde{O}(1/\delta)$, such that the capacity violation is at most $\alpha\beta$. This takes care of the clients in $\Cbb$. Finally, Claim~\ref{clm:prelim3} takes care of all the deleted clients $\Cd$ with an extra hit of $(1+\delta)$ on the capacity and additive $\tilde{O}(1/\delta)$ on the distance.
\end{proof}
\noindent
This completes the proof of Theorem~\ref{thm:reduction} for the general \mckc problem. For the problem with soft capacities, the proof is exactly the same, except in the end, the instance $\calI_\calT$ is a $Q||C_{min}$ instance rather than a \cckp one.
\end{proof}


We end this section by noting that for the \mckc problem with soft-capacities, if we use the assignment supply polyhedra described in Section~\ref{sec:supplypolyhedra}, then we do not need to run the ellipsoid algorithm.
In particular, the inequality \eqref{eq:lp7} is implied \eqref{eq:lp1}-\eqref{eq:lp6} for $\calP_\mathsf{ass}$ defined in \eqref{eq:asslp1}-\eqref{eq:asslp3}.
\begin{lemma}\label{lem:implied}
Given any $(x,y)$ feasible for LP\eqref{eq:lp1}-\eqref{eq:lp6} and any $\calT = (T_1,\ldots,T_m)$, we have $\yy \in \calP_\mathsf{ass}(\calI_\calT)$.
\end{lemma}
\begin{proof}
	Fix $\calT = (T_1,\ldots,T_m)$ to be a collection of complete neighborhood sets. In the instance $\calI_\calT$ of $Q||C_{min}$, we have $m$ machines with demands
	$D_\ell = |J_\ell|$, where $J_\ell$ is the client set responsible for $T_\ell$. Recall, $y^\calT_p := \sum_{i \in \calT} y_{ip}$, and we need to find $z_{\ell,p}$ which satisfy the constraints
	\eqref{eq:asslp1}-\eqref{eq:asslp3} where $s_p := y^\calT_p$.
	
	The definition is natural: $z_{\ell,p} := \sum_{i\in T_\ell} y_{ip}$. Clearly it satisfies \eqref{eq:asslp1} (indeed with equality). We now show it satisfies \eqref{eq:asslp2}.
	To this end, define for any $j\in J_\ell$, $x_{jp} := \sum_{i\in T_\ell} x_{ijp}$. Since $\Gamma(J_\ell) \subseteq T_\ell$, we get from \eqref{eq:lp1} that $\sum_p x_{jp} \geq 1$.
	In particular,
	\begin{equation}
\textstyle 	\label{eq:hohum1}\sum_p \sum_{j\in J_\ell}x_{jp} \geq D_\ell
	\end{equation}
	
	From \eqref{eq:lp4}, we know $x_{ijp} \leq y_{ip}$ and summing over all $i\in T_\ell$, we get for all $j\in T_\ell$, $x_{jp} \leq \sum_{i\in T_\ell} y_{ip} = z_{\ell,p}$.
	In particular, $\sum_{j\in J_\ell} x_{jp} \leq z_{\ell,p} D_\ell$. From \eqref{eq:lp2} we know for all $i\in T_\ell, p\in [P]$, $\sum_{j\in J_\ell} x_{ijp} \leq c_p y_{ip}$.
	Summing over all $i\in T_\ell$, gives $\sum_{j\in J_\ell} x_{jp} \leq c_p z_{\ell,p}$. Putting together, we get
	\begin{equation}
	\textstyle 	\label{eq:hohum2}\sum_{j\in J_\ell}x_{jp} \leq z_{\ell,p} \min(D_\ell,c_p)
	\end{equation}
	\eqref{eq:hohum1} and \eqref{eq:hohum2} imply that $z$ satisfies \eqref{eq:asslp2}.
\end{proof}
Therefore, one can use the natural LP relaxation to obtain for any $\delta > 0$, a $\left(\tilde{O}(1/\delta),(2+\delta)\right)$-bicriteria approximation for the \mckc problem with soft capacities.
As it should be clear, this is a much more efficient algorithm.

\fi



% described in Idea 1.






\section{Conclusion}
Blah Blah

\bibliographystyle{abbrv}
\bibliography{mckc}
\newpage
\appendix

\section{Full Version: Roadmap}
For completeness and a cleaner presentation, we present all the details completely in the full version, including all the definitions and theorems already stated in the main body of the paper. 
\comment{rk: give roadmap.}
%\input{}
%\newpage
\section{Technical Preliminaries}\label{fsec:prelims}
Given an \mckc instance, we start by guessing $\opt$. We either prove $\opt$ is infeasible, or find an $(a,b)$-approximate allocation of clients to facilities.
	We define the bipartite graph $G = (F\cup C,E)$ where $(i,j)\in E$ iff $d(i,j) \leq \opt$. If $\opt$ is feasible, then the following assignment LP\eqref{feq:lp1}-\eqref{feq:lp6}
	must have a feasible solution.
In this LP, we  have opening  variables $y_{ip}$ for every $i\in F,p\in [P]$ indicating whether we open a facility with capacity $c_p$ at location $i$. Recall that the capacities available to us are $c_1, c_2, \ldots, c_P$ -- a facility with
capacity $c_p$ installed on it will be referred to as a {\em type $p$ facility.}
	We have connection variables $x_{ijp}$ indicating the fraction to which client $j\in C$ connects to a facility at location $i$ where a type $p$ facility has been opened.
	We force $x_{ijp} = 0$ for all pairs $i,j$ and type $p$ such that  $d(i,j) > \opt$.
	
		
		\begin{minipage}{0.49\textwidth}
			\begin{alignat}{3}
				& \forall j\in C,   &\textstyle \sum_{i\in F} \sum_{p\in [P]}  x_{ijp} \geq 1 \label{feq:lp1} \tag{\small{L1}}  \\
				& \forall i\in F,p\in [P] ,  &\textstyle \sum_{j\in C}  x_{ijp} \leq c_py_{ip} \label{feq:lp2} \tag{\small{L2}} \\
				& \forall p\in [P], & \textstyle \sum_{i\in F} y_{iq}   \leq k_p \label{feq:lp3}  \tag{\small{L3}}
			\end{alignat}
		\end{minipage}
		~\vline~
		\begin{minipage}{0.49\textwidth}
			\begin{alignat}{3}
				& \forall i\in F, j\in C,p\in [P],  &  x_{ijp} \leq y_{ip}\label{feq:lp4}   \tag{\small{L4}} \\
				& \forall i\in F, &  \textstyle\sum_{p\in [P]} y_{ip} \leq 1 \label{feq:lp5}  \tag{\small{L5}} \\
				& \forall i\in F,j\in C,p\in [P], &  x_{ijp},y_{ip} \geq 0\label{feq:lp6}\tag{\small{L6}}
			\end{alignat}
		\end{minipage}
\smallskip

\noindent		
We say a solution $(x,y)$ is $(a,b)$-feasible if it satisfies \eqref{feq:lp1}, \eqref{feq:lp3}-\eqref{feq:lp6}, and \eqref{feq:lp2} with the RHS replaced by $bc_py^\mathsf{int}_{ip}$, and $x_{ijp} > 0$ only if $d(i,j) \leq a\cdot \opt$,
We desire to find an integral solution $(x^\mathsf{int},y^\mathsf{int})$ which is $(a,b)$-feasible.
The following lemma shows that it suffices just to round the $y$-variables.
\begin{claim}
Given an $(a,b)$-feasible solution $(x,\y)$ where $\y_{ip}\in \{0,1\}$,
we can get  an $(a,b)$-approximate solution to the \mckc problem.
\end{claim}
\begin{proof}
Consider a bipartite graph with client nodes $C$ on one side, and nodes of the form $(i,p)$ with $\y_{ip} = 1$ on the other. The node $(i,p)$ has capacity $bc_p$.
Since $(x,\y)$ satisfies the conditions of the lemma, there is a fractional matching in this graph so that each client $j$  is fractionally matched to an $(i,p)$ so that $d(i,j)\leq a\cdot \opt$,
and the total fractional load on $(i,p)$ is $\leq bc_p$. The theory of matching tells us that there is an {\em integral} assignment of clients $j$ to nodes $(i,p)$ such that $d(i,j)\leq a\cdot\opt$
and the number of nodes matched to $(i,p)$ is $\leq \ceil{bc_p}$. Therefore opening a capacity $c_p$ facility at $i$ for all $(i,p)$ with $\y_{ip} = 1$ gives an $(a,b)$-approximate solution to \mckc.
\end{proof}
\noindent
Henceforth, we focus on rounding the $y$-values. To this end, we make the following useful definition.
\begin{definition}[Roundable Sets]\label{fdef:rnding-mkc}
	A set of facilities $S\subseteq F$ is said to be $(a,b)$-roundable w.r.t $(x,y)$ if
	\begin{itemize}[noitemsep]
		\item[(a)] $\diam_G(S) \leq a$
		\item[(b)] there exists a rounding $\y_{ip} \in \{0,1\}$ for all $i \in S, p\in [P]$ such that
		\begin{enumerate}
			\item $\sum_{q \geq p} \sum_{i\in S} \y_{iq} ~\leq~ \floor{\sum_{q \geq p}\sum_{i\in S} y_{iq}}$ for all $p$, and
			\item $\sum_{j\in C} d_j \sum_{i\in S,p\in [P]} x_{ijp} \leq b\cdot \sum_{i\in S} \sum_{p\in [P]} c_p \y_{ip}$
		\end{enumerate}
	\end{itemize}
\end{definition}
\noindent
If $(x,y)$ were feasible, then for any $(a,b)$-roundable set, we can integrally open facilities to satisfy all the demand that was fractionally assigned to it taking a hit of $a$ in the cost and a factor of $b$ in the capacities. Furthermore, the number of open facilities is at most what the LP prescribes. Therefore, if we would be able to decompose the instance into roundable sets, we would be done.
Unfortunately, that is not possible, and in fact the above LP has a large integrality gap even when we allow arbitrary violation of capacities.

\begin{remark}[Integrality Gap for \mckc] \label{frem:ig}
Consider the following instance. The metric space $X$ is partitioned into $(F_1\cup C_1) \cup \cdots \cup (F_K\cup C_K)$, with $|F_k| = 2$ and $|C_k| = K$ for all $1\le k\le K$.
The distance between any two points in $F_i\cup C_i$ is $1$ for all $i$, while all other distances are $\infty$. The capacities available are $k_1 = K$ facilities with capacity $c_1 = 1$ and
$k_2= K-1$ facilities with capacity $c_2 = K$. It is easy to see that integrally any solution would violate capacities by a factor of $K/2$.
%It is easy to see that the above instance is not feasible with $OPT=1$: indeed, there is at least one client location where the optimal solution does not place a facility of capacity $H$ in its neighborhood, and it is not possible to serve the demand of this client using only capacity $1$ facilities, as there are only two locations where we can place facilities in its neighborhood.
On the other hand, there is a feasible solution for the above LP relaxation: for $F_k = \{a_k,b_k\}$, we set $y_{a_k2} = 1-1/K$ and $y_{b_k1} = 1$, and for all $j\in C_k$, we set $x_{a_kj2} = 1-1/K$ and $x_{b_kj1} = 1/K$.


 For the version with soft capacities, we do not have the constraint \eqref{feq:lp5} and the above integrality gap doesn't hold since we can install capacity $K$ facilities on $K-1$ of the sets $F_k$'s, $1\leq k\leq K-1$, and $K$ copies of the capacity $1$ facilities at $F_K$. Note that although $|F_K| = 2$, we have opened $K$ capacities.
\end{remark}

In particular, note that for the $(x,y)$ solution in the integrality gap example above there are no roundable sets. This motivates the definition of the second kind of sets.

\begin{definition}[Complete Neighborhood Sets] \label{fdef:comp-nbr}
	A subset $T\subseteq F$ of facilities is called a {\em complete neighborhood} if there exists a client-set $J\subseteq C$ such that $\Gamma(J) \subseteq T$.
	In this case $J$ is said to be {\em responsible} for $T$. Additionally, a complete neighborhood $T$ is said to be an $\alpha$-complete neighborhood if $\diam(T) \leq \alpha$.
\end{definition}
\begin{remark}[Complete Neighborhood Sets to \cckp]\label{frem:red}
	\emph{
If we find a complete neighborhood $T$ of facilities with say a set $J$ of clients responsible for it, then we know that the optimal solution must satisfy all the demand in $J$ by suitably opening facilities of sufficient capacity in $S$. Given a collection $\calT = (T_1,\ldots,T_m)$ of disjoint $\alpha$-complete neighborhood sets with $J_i$ repsonsible for $T_i$, we can define an instance $\calI$ of the \cckp problem with $m$ machines with demands $D_i = |T_i|$ and cardinality constraint $f_i = |T_i|$, and $P$ jobs of capacities $c_1,\ldots,c_P$. The facilities opened by the $\opt $ solution corresponds to a valid solution for $\calI$; furthermore, any $\beta$-approximate solution for $\calI$ corresponds to a $(\alpha,\beta)$-approximate solution for the \mckc problem restricted to clients in $\cup_{\ell} J_\ell$. Finally note that for \mckc with soft-capacities, $\calI$ is an instance of the $Q||C_{min}$ problem.
}

\emph{
Note that the above integrality gap  example is essentially a \cckp instance with $K$ machines of demand $K$ each having cardinality constraint $2$, and there are $K$ jobs of capacity $1$ and $K-1$ jobs with capacity $K$. This shows the assignment LP has bad integrality gap for the \cckp problem (but not for $Q||C_{min}$).}
\end{remark}

Our final definition is that of $(\tau,\rho)$-{\em deletable clients} who can be removed from the instance since they can be ``$\rho$-charged'' to the remaining clients no further than $\tau$-away.
\begin{definition}[Deletable Clients]\label{fdef:deletable}
	A subset $\Cd\subseteq C$ of clients is $\rho$-deletable if there exists a mapping $\phi_{j,j'}\in [0,1]$ for $j\in \Cd$ and $j'\in C\setminus \Cd$ satisfying (a) $\sum_{j'\in C\setminus \Cd} \phi_{j,j'} = 1$ for all $j\in \Cd$, and(b) $\sum_{j\in \Cd} \phi_{j,j'} \leq \rho$ for all $j'\in C\setminus \Cd$. Furthermore, $\phi_{j,j'} > 0$ only if $d(j,j') \leq \tau\cdot\opt$.
\end{definition}
 The following claim shows we can remove $\Cd$ from consideration.
\begin{claim}\label{fclm:prelim3}
	Let $\Cd$ be a $(\rho,\tau)$-deletable set.
	Given an $(a,b)$-approximate feasible solution $(x',\y)$ where $x'_{ijp}$ is defined only for $j\in C\setminus \Cd$, we can extend $x'$ to a general $(x,\y)$ solution
	which is $(a+\tau, b(1+\rho))$-approximate feasible.
\end{claim}
\begin{proof}
For any $j\in \Cd$, define $x_{ijp} = \sum_{j'\in C\setminus \Cd} x_{ij'p}\phi_{j,j'}$.
We get for all $j\in \Cd$,
$\textstyle \sum_{i\in F} \sum_{p\in [P]} x_{ijp} = \sum_{i,p} \sum_{j'\in C\setminus \Cd} x_{ij'p}\phi_{j,j'} = \sum_{j'\in C\setminus \Cd} \phi_{j,j'} \left(\sum_{i,p} x_{ij'p}\right) \geq \sum_{j'\in C\setminus \Cd} \phi_{j,j'} = 1$,
and for all $i\in F,p\in [P]$,
$\textstyle \sum_{j\in \Cd}  x_{ijp} = \sum_{j\in \Cd} \sum_{j'\in C\setminus \Cd} x_{ij'p}\phi_{j,j'} = \sum_{j'\in C\setminus \Cd} x_{ij'p}\left( \sum_{j\in \Cd} \phi_{j,j'}\right)  \leq \rho \sum_{j'\in C\setminus \Cd} x_{ijp} \leq b\rho c_p$. Therefore, in all we have $\sum_{j\in C} x_{ijp} \leq bc_p(1+\rho)$.
\end{proof}


% described in Idea 1.






\section{Region Growing: Reduction to $Q|k_i|C_{min}$}
In this section, we give a reduction to $Q|k_i|C_{min}$ when we allow logarithmic approximations. The main technique is region growing which was first used in the context of sparsest and multi cut problems~\cite{LeightonRao,GVY}.
\begin{theorem}\label{thm:red}
	Given an $\alpha$-approximation algorithm for $Q|k_i|C_{min}$, for any $\epsilon>0$ there exists an \\ $\left(O(\log n/\epsilon), \alpha(1+\epsilon)\right)$-approximate algorithm for the \mckc problem.
\end{theorem}
\begin{proof}
	We start with a guess of $\opt$ and the $\opt$-induced graph $G$. We assume this is a correct guess.
	Given $G$ we given an algorithm construct an instance $\calI$ of $Q|k_i|C_{min}$.
	Initially $\calI$ is empty. We maintain a set of alive clients $C'$ which is initially $C$. We maintain a working graph $H$ which is initialized to $G$ and is always a subgraph of $G$.
	Given a client $j$ and an integer $t$, let $N_t(j)$ denote all the clients $j'$ s.t. $d_H(j,j') < 2t$ and $\Gamma_t(j)$ denote all the clients $j'$ with $d_H(j,j')  = 2t$.\smallskip
	
	
	Till $C'$ is empty, we perform the following operation.
	Select an arbitrary active client $j\in C'$. Find the smallest $t$ such that $|\Gamma_t(j)| < \eps\cdot |N_t(j)|$. Since for all $s < t$ we have $|N_{s+1}(j)|\geq (1+\epsilon)|N_s(j)|$ which implies $|N_{s+1}(j)| > (1+\epsilon)^s$, 
	we get  $t \leq (\ln n)/\epsilon$ where $n=|C'|$. Let us use $A_j$ to denote $N_t(j)$ and $B_j$ to denote $\Gamma_{t}(j)$. Also let us use $F_j$ to denote all the facilities neighboring any client in $A_j$. Note that any facility $i\in F_j$ has neighbors in $A_j \cup B_j$.
	We now add a machine to $\calI$ and attribute it to the client $j$. Abusing notation, we call the machine $j$ as well.
	We add the  cardinality constraint $k_j = |F_j|$. The speed of the machine is set to be $s_j := 1/|A_j|$.
	Finally,  we delete $A_j\cup B_j$ from $C'$ and $A_j\cup B_j\cup F_j$ from $H$ and repeat. \smallskip
	
	At the end of the above operation, we have an instance $\calI$ where the machines and their cardinality constraints have been specified.
	We are given $P$ jobs where the processing time of the $p$th job is $c_p$. This completes the description of the $Q|k_i|C_{min}$ instance.
	
	\begin{claim}
		There exists an allocation where the total processing time of each machine is at least $1$.
	\end{claim}
	\begin{proof}
		In the optimal solution to \mckc, a total capacity of $|A_j|$ must be installed among the facilities $F_j$. 
		We mimic this solution for $\calI$, and since the speed of machine $j$ is $1/|A_j|$, the claim follows.
	\end{proof}
	\begin{claim}
		Given an allocation to $\calI$ where each machine gets processing time $\geq \alpha$, we can construct a solution for the \mckc problem
		which is $O(\log n/\epsilon)$-approximate and violates the capacities by at most $(1+\epsilon)/\alpha$-factor.
	\end{claim}
	\begin{proof}
		For machine $j$ in $\calI$, let $S_j$ be the jobs allocated; we have $\sum_{p\in S_j} c_p \ge \alpha\cdot |A_j|$ and $|S_j| \leq k_j = |F_j|$.
		We arbitrarily open $|S_j|$ facilities in $F_j$ to which we assign all the clients in $A_j \cup B_j$. Since $|B_j|<\eps |A_j|$, we have the total capacity opened is at least $\alpha/(1+\epsilon)$ times 
		the total number of clients in $A_j\cup B_j$. Furthermore, every client in $A_j\cup B_j$ is at most $O(\log n/\epsilon)$-away from any facility in $F_j$. This implies the assignment.
	\end{proof}	
\end{proof}


\newpage
\section{Max-Min Allocation Problems and Supply Polyhedra}\label{sec:regiongrowing}
An instance of the non-uniform max-min allocation problem $Q||C_{min}$, is described  $m$ machines $M$ with demands $D_1,\ldots,D_m$ and $n$  types of jobs $J$ with capacities $c_1,\ldots,c_n$. 
In the version with cardinality constraints, that is $Q|f_i|C_{min}$ we are also given positive integers $f_1,\ldots, f_m$.

A {\em supply vector} $(s_1,\ldots,s_n)$ where each $s_j$ is a non-negative integer
is called {\em feasible} for instances of these problems if the ensemble formed by $s_j$ copies of jobs of capacity $c_j$ can be allocated feasibly to satisfy all the demands.
The {\em supply polyhedra} of these instances desire to capture these feasible supply vectors.

\begin{definition}[Supply Polyhedron]
	Given an instance $\calI$ for a max-min allocation problem, a polyhedron $\calP(\calI)$ is called an $\alpha$-approximate supply polyhedron if 
	(a) all feasible supply vectors lie in $\calP(\calI)$, and (b) given any non-negative integer vector $(s_1,\ldots,s_n)\in \calP(\calI)$ there exists an assignment
	of the $s_j$ jobs of capacity $c_j$ to the machines such that machine $i$ receives a total capacity of $\geq D_i/\alpha$.
\end{definition}

Ideally, we would like {\em exactly} supply polyhedra. One guess would be the convex hull of all the supply vectors; indeed this is the tightest polytope satisfying condition (a).
Unfortunately, there are instances of $Q||C_{min}$ (and even for the uniform case $P||C_{min}$), the convex hull of supply vectors contain infeasible integer points. This instance is
motivated by integrality gap examples for machine scheduling~\cite{bibid}.
\begin{theorem}
	There cannot exist $\alpha$-approximate supply polyhedra (or convex sets) for $\alpha < 1.001$ for all  $Q||C_{min}$ instances.
\end{theorem}
\begin{proof}
\comment{	Needs to be written }.
\end{proof}
\begin{remark}\emph{
At this point, we should underscore the difference between supply polyhedra and say LP relaxations for solving  these allocation problems.
Given an instance of say $Q||C_{max}$ {\em along with} the supply vector (which is one standard way the problems are stated), there does exist a polytope capturing all the feasible allocations. It is the integer hull.
However, in general, the description of this integer hull uses the supply vector in describing these constraints and therefore are non-linear when the supplies are variables. Nevertheless, as we discuss below, many LP relaxations
studied in the literature imply supply polyhedra, and their integrality gaps imply the approximation factor for the polyhedra as well.
}
\end{remark}

For our purposes, we need more technical conditions from the supply polyhedra. The first is a natural condition which states if one moves the supply to higher capacity jobs, then feasibility remains.
The second is related to polynomial time algorithms.

\begin{definition}
	An supply polyhedra $\calP(\calI)$ for is said to be {\em useful} if it satisfies the following two conditions.
	\begin{asparaitem}
		\item[\emph{(Upward Feasibility.)}] Reorder the jobs so that $c_1\le c_2 \le \cdots \le c_n$.
		If $(s_1,\ldots,s_n)\in \calP$ and $(\bar{s}_1,\ldots,\bar{s}_n)$ is a vector satisfying $\sum_{k\geq i} \bar{s}_k \geq \sum_{k\geq i} s_k$, then $(\bar{s}_1,\ldots,\bar{s}_n)\in \calP$ as well.
		\item[\emph{($\beta$-Approximate Separation.)}] Given any $y\in \R^n_{\geq 0}$,  there is a polynomial time procedure which either returns a hyperplane separating $y$ from $\calP$, or asserts that 
		$y\in \calP(\calI')$ for the supply polyhedra of the instance $\calI'$ where all demands have been reduced by a factor $\beta$.
	\end{asparaitem}
\end{definition}

\begin{itemize}[noitemsep]
	\item Re-state natural LP for \mckc.
	\item Supply Polyhedra for $Q||C_{min}$.
	\item Supply Polyhedra for $Q|f_i|C_{min}$.
	\item Connection to Integrality Gaps ... how $\alpha$-appx supply polyhedra imply $\alpha$-approximation.
	\item Rounding Assignment  for Small Jobs.
\end{itemize}
\newpage \section{\mckc via Supply Polyhedra}
In this section, we prove the following theorem.
\begin{theorem}\label{thm:reduction}
Suppose there exists $\alpha$-approximate supply polyhedra  for all instances of $Q|f_i|C_{min}$ (respectively, $Q||C_{min}$) which are useful with $\beta$-approximate separation oracles .
Then for any $\epsilon> 0$, there is an $\left(\tilde{O}(1/\epsilon),\alpha\beta(1+\eps)\right)$-bicriteria approximation algorithm for the \mckc problem (respectively, with soft capacities).
\end{theorem}
\begin{proof}{\bf deepc: write about graph etc. somewhere it has been written.}\smallskip
	
\noindent
Recall the natural LP relaxation for the \mckc problem.

\begin{minipage}{0.45\textwidth}
	\begin{alignat}{4}
		& \quad \forall j\in C,   &&\quad  \textstyle \sum_{i\in F} \sum_{p\in [P]}  x_{ijp} \geq 1 \label{eq:lp1} \\
		& \quad \forall i\in F,p\in [P] ,  &&\quad  \textstyle \sum_{j\in C}  x_{ijp} \leq c_py_{ip} \label{eq:lp2} \\
		& \quad \forall p\in [P], && \quad \textstyle \sum_{q \geq p} \sum_{i\in F} y_{iq}   \leq \sum_{q\geq p} k_q \label{eq:lp3}  
	\end{alignat}
\end{minipage}
~\vline~
\begin{minipage}{0.45\textwidth}
	\begin{alignat}{4}
		& \quad \forall i\in F, j\in C,p\in [P],  && \quad x_{ijp} \leq y_{ip}\label{eq:lp4}   \\
		& \quad \forall i\in F, && \quad \textstyle\sum_{p\in [P]} y_{ip} \leq 1 \label{eq:lp5}  \\
		& \quad \forall i\in F,j\in C,p\in [P], && \quad x_{ijp},y_{ip} \geq 0\label{eq:lp6}
	\end{alignat}
\end{minipage}
\smallskip

\noindent
For the version with soft capacities, we do not have the constraint \eqref{eq:lp5}. \smallskip

For the \mckc problem, as stated, the above LP has unbounded integrality gap. We strengthen the LP as follows.
Recall Definition~\ref{def:comp-nbr} of complete neighborhoods: a subset $T\subseteq F$ is a complete neighborhood 
if there exists $J\subseteq C$ with $\Gamma(J) \subseteq T$. We add a constraint for every collection $\calT := (T_1,\ldots,T_L)$ of $L$ disjoint complete neighborhood sets. 

\def\yy{\bar{y}}
Given $\calT$, we define an instance $\calI_\calT$ of the $Q|f_i|C_{min}$ problem as follows. There are $L$ machines with demand $D_\ell := |J_\ell|$ for $1\leq\ell\leq L$ where $J_\ell$ is the subset of $C$ responsible for $T_\ell$.
The cardinality constraint for this demand is $f_\ell := |T_\ell|$. The capacities available are $c_1\leq \cdots \leq c_P$.
Define  $\yy_p := \sum_{i\in \calT} y_{ip}$ for all $1\le p\le P$. If the \mckc instance is feasible, then there is an integral solution where $(\yy_1,\ldots,\yy_n)$ is a feasible supply vector for $\calI_\calT$.
This is the constraint we add.

\begin{equation}\label{eq:lp7}
\forall \calT := (T_1,\ldots,T_L) \textrm{ disjoint neighborhood subsets}, \quad \yy \in \calP(\calI_\calT)
\end{equation}
\noindent
We don't know how (and don't expect) to check feasibility of  \eqref{eq:lp7} for all collections $\calT$. However, we can still run ellipsoid method using the ``round-and-cut'' framework of \cite{bibid}.
To begin with, we start with the LP\eqref{eq:lp1}-\eqref{eq:lp6} and obtain feasible solution $(x,y)$. Subsequently, we apply the decomposition Theorem~\ref{thm:decomp} to obtain the collection $\calT = (T_1,\ldots,T_L)$.
We then check if $\yy \in \calP(\calI_\calT)$ or not. Since we have a $\beta$-approximate separation oracle for $\calP(\calI_\calT)$, we either are guaranteed that $\yy \in \calP(\calI'_\calT)$ where the $\ell$th demand is now 
$D_\ell/\beta$. Or we get a hyperplane separating $\yy$ from $\calP(\calI_\calT)$ which also gives us a
hyperplane separating $y$ from  LP\eqref{eq:lp1}-\eqref{eq:lp7}. This can be fed to the ellipsoid algorithm to obtain a new iterate $(x,y)$ and the above process repeated. The analysis of the ellipsoid algorithm\comment{deepc: need to be careful and correct here}
tells us that in polynomial time we either prove infeasibility of the system \eqref{eq:lp1}-\eqref{eq:lp7} (implying the $\opt$  guess for \mckc is infeasible), or we obtain a solution  $(x,y)$ such that for the $\calS, \calT$ obtained via Decomposition Theorem~\ref{thm:decomp}
and the instance $\calI_\calT$ so obtained, the solution  $\yy \in \calP(\calI'_\calT)$ for the $\beta$-shaded instance.

Suppose the latter occurs. From this we can obtain the bicriteria solution to the \mckc problem as follows. 
Recall the notation in the Decomposition Theorem~\ref{thm:decomp}. Since every set $S_k, 1\leq k\leq K$ is $(\tilde{O}(\frac{1}{\eps}),(1+\eps))$-roundable, there exists a rounding $Y_{ip}$ for $i\in S_k$  such that
\begin{equation}\label{eq:repeat}
\textstyle \sum_{q\geq p} \sum_{i\in S_k} Y_{iq} \leq \floor{\sum_{q\geq p} \sum_{i\in S_k} y_{iq}} 
\end{equation}
Install a capacity of $c_p$ wherever $Y_{ip} = 1$, and assign all the clients in $j\in C_\mathsf{blue}$ to a facility $\psi(j) \in \calS$ such that $d(j,\psi(j)) \leq \tilde{O}(\frac{1}{\eps})$. This can be done because each $S_k$ is a locally roundable set.
Furthermore, if any $j\in \Cd$ has $\phi(j) \in \Cb$, then assign $j$ to $\psi(\phi(j))$. Therefore, we have taken care of clients in $\Cb$ and some clients of $\Cd$ violating capacities by $(1+\epsilon)^2$-factor.
Let $k^{(S)}_p := \sum_{i\in \calS} Y_{ip}$ be the number of facilities of type $p$ opened, and let $k^{(T)}_p := k_p - k^{(S)}_p$.

Now we come to $\calT$. Note that $\yy_p := \sum_{i\in \calT} y_{ip}$, and we have $\yy \in \calP(\calI'_\calT)$. Recall that $\calI'_\calT$ has $L$ machines with demands $D_\ell := |J_\ell|/\beta$ and $f_\ell := |T_\ell|$.
\begin{claim}
	$(k^{(T)}_1,\ldots, k^{(T)}_P) \in \calP(\calI'_\calT)$
\end{claim}
%	We now use the upward-feasibility property of $\calP$ to prove that the integral vector $(k^{(T)}_1,\ldots, k^{(T)}_P)$ also lies in $\calP(\calI'_\calT)$. 
\begin{proof}To see this note that for all $p$, from \eqref{eq:repeat} and \eqref{eq:lp3}
\[
\textstyle \sum_{q\ge p} \sum_{i\in \calS} y_{iq} - \sum_{q\geq p} k^{(S)}_q \geq 0 ~~~\textrm{and} ~~~~ \sum_{q\geq p} k_q - \sum_{q\ge p}\sum_{i\in F} y_{iq} \geq 0
\]
which upon adding gives for all $p$, $\sum_{q\geq p} k^{(T)}_q \geq \sum_{q\ge p} \sum_{i\in \calT} y_{iq} = \sum_{q\geq p} \yy_q$.
By the upward-feasibility property of $\calP$, we get the claim.
\end{proof}
Since $\calP(\calI'_\calT)$ is $\alpha$-approximate, we can find an allocation of the $k^{(T)}_p$ copies of jobs of capacity $c_p$ to the $L$ machines such that machine $\ell$ gets at most $f_\ell$ jobs
and total capacity $\geq D_\ell/\alpha = |J_\ell|/\alpha\beta$. We install these capacities on the facilities of $T_\ell$. Since the diameter of each $T_\ell$ is $\tilde{O}(1/\epsilon)$, we can find an assignment $\psi$ 
of $\Cbb$ to these open facilities such that $d(j,\psi(j)) = \tilde{O}(1/\epsilon)$ and capacity violation is at most $\alpha\beta$. Furthermore, when we assign $j\in \Cd$ for which $\phi(j) \in \Cbb$ to $\psi(\phi(j))$, 
the total capacity violation is at most $(1+\eps)\alpha\beta$. 
This completes the proof of Theorem~\ref{thm:reduction}.
\end{proof}

\input{section4}
%\newpage
\section{Supply Polyhedra of $Q||C_{min}$: Proof of Theorem~\ref{fthm:asslp}}\label{fsec:asslp}
\def\pv{\mathbf{b}}
\newcommand{\dem}{\mathsf{cap}}
  Throughout the proof we fix $\calI$ to be the instance of $Q||C_{min}$ % and we rename machines and jobs so that $D_1 \geq \cdots \geq D_m$ and $n$ types of  jobs $J$ with capacities $c_1 \geq \cdots \geq c_n$.
  %We also fix
  and the supply vector $(s_1,\ldots,s_n)$. For simplicity of presentation, given the supply vector, abusing notation let $J$ denote the multiset of jobs where job $j$ appears $s_j$ times. We know that the LP\eqref{feq:asslp1}-\eqref{feq:asslp3} is feasible with the $s_j$ replaced by $1$. %Let $N = \sum_j s_j$.
  We want to find an assignment where machine $i$ gets at least $D_i/2$ capacity.
%Let  $z$ be a feasible solution to \eqref{feq:asslp1}-\eqref{feq:asslp3}.
%Let the instance $\calI$ of $Q||C_{min}$  have $m$ machines $M$ with demands
%A supply vector $(s_1,\ldots,s_n)$ indicates the number of jobs of each type available; a supply vector is feasible if together they can satisfy all the demands.
%We wish to find a convex set/polyhedra which captures all the feasible supply vectors. In particular, any feasible supply vector should be in the set, and given any (integer) supply vector in the set
%there should be an allocation which satisfies the demands to an $\alpha$-factor.
% \smallskip
%
%\noindent
%A feasible supply vector $(s_1,\ldots,s_n)$ must lie in the following polytope.
%%We assume we have a guess $D$ for the optimum value which is certified by a feasible solution to the following assignment LP. Below, $D_i := Ds_i$.
%	\begin{alignat}{4}
%		\calP_\mathsf{ass} && = \{(s_1,\ldots,s_n):  && \notag \\
%		&& \quad \forall j \in J,   &\quad  \textstyle \sum_{i\in M} z_{ij}  \leq  s_j \label{feq:asslp1} \\
%		&& \quad \forall i\in M ,  &\quad  \textstyle \sum_{j\in C}  z_{ij}  \min(c_j,D_i) \geq D_i \label{feq:asslp2} \\
%		&& \quad \forall i\in M, j\in J, & \quad z_{ij}   \geq 0 \label{feq:asslp3}  \}
%	\end{alignat}
%Not all integral $(s_1,\ldots,s_n) \in \calP_\mathsf{ass}$ need be feasible; but the following theorem shows given such a supply vector, there exists an assignment satisfying the demands up to a factor $2$.
%%The following theorem shows that $\calP_\mathsf{ass}$ captures the
%%the assignment LP has integrality $\leq 2$.
%\begin{theorem}\label{fthm:asslprounding}
%Given $(s_1,\ldots,s_n) \in \calP_\mathsf{ass}$, there is an of assignment $\phi$ of the $s_j$ jobs of capacity $c_j$  to the machines such that for all $i\in M$,
%$\sum_{j:\phi(j) = i} c_j \geq D_i/2$.
%\end{theorem}
%\begin{proof}
%For simplicity, given the supply vector, abusing notation let $J$ denote the multiset of jobs where job $j$ appears $s_j$ times. We know that the LP\eqref{feq:asslp1}-\eqref{feq:asslp3} is feasible with the $s_j$ replaced by $1$. Let $N = \sum_j s_j$.
%

The algorithm is a  very simple greedy algorithm which doesn't look at the LP solution , and the feasibility of LP\eqref{feq:asslp1}-\eqref{feq:asslp3} is only used for analysis.
Rename the jobs (with multiplicities) in decreasing order of capacities $c_1\geq c_2 \geq \cdots \geq c_N$, and rename  the machines in decreasing order of $D_i$'s, that is, $D_1 \geq D_2 \geq \ldots \geq D_m$.
Starting with machine $i=1$ and job $j=1$, assign jobs $j$ to $i$ if the total capacity filled in machine $i$ is $< D_i/2$ and move to the next job. Otherwise, call machine $i$ happy and move to the next machine. Obviously, if all machines are happy at the end we have found our assignment.

The non-trivial part is to  prove that if some machine is unhappy, then the LP\eqref{feq:asslp1}-\eqref{feq:asslp3} is infeasible (with $s_j$ replaced by $1$).
To do so, we take the Farkas dual of the LP; the following LP is feasible iff LP\eqref{feq:asslp1}-\eqref{feq:asslp3} is infeasible. We describe a feasible solution to the system below if we obtain some unhappy agent.
	\begin{alignat}{4}
		&&   & \quad \textstyle \sum_{i=1}^m \beta_i D_i > \sum_{j=1}^n\alpha_j \label{feq:assdual1}  \tag{F1} \\
		&& \quad \forall i\in M,j\in J & \quad \textstyle \beta_i\min(c_j,D_i) \leq \alpha_j \label{feq:assdual2} \tag{F2}  \\
		&& \quad \forall i\in M, &\quad  \beta_i \geq 0\label{feq:assdual3}\tag{F3}
	\end{alignat}
	\def\i{i^\star}
Suppose machine $\i$ is the first machine which is unhappy. Let $S_1,\ldots,S_{\i-1}$ be the jobs assigned to machines $1$ to $(\i-1)$ and $S_{i^\star}$ be the remainder of jobs.
We have $\sum_{j\in S_{\i}} c_j < D_{\i}/2$. We also have for all $1\leq i\leq \i$, $\sum_{j\in S_i} \min(c_j,D_i) \leq D_i$. If not, then the machine must receive at least two jobs and would have capacity $> D_i/2$
from all but the last.
We now describe a feasible solution to \eqref{feq:assdual1}-\eqref{feq:assdual3}.
%{\bf deepc: this turns out to be trickier than what meets the eye}.

Given the assignment $S_i$'s, call a machine $i$ {\em overloaded} if $S_i$ contains a single jobs $j_i$ with $c_{j_i} \geq D_i$.
We let $\beta_1 = 1$. For $1\leq i <\i$, we have the following three-pronged rule
\begin{itemize}[noitemsep]
	\item If $i+1$ is not overloaded, $\beta_{i+1} = \beta_i$.
	\item If $i+1$ is overloaded, and so is $i$, then $\beta_{i+1} = \beta_i \cdot D_i/D_{i+1}$.
	\item If $i+1$ is overloaded but $i$ is not, then $\beta_{i+1} = \beta_i \cdot c_{j_{i+1}}/D_{i+1}$, where $j_{i+1}$ is the job assigned to $i+1$.
\end{itemize}
For any job $j$ assigned to machine $i$, we set $\alpha_j = \beta_i \min(c_j,D_i)$. Since for any $S_i$, we have $\sum_{j\in S_i} \min(c_j,D_i) \leq D_i$ and $\sum_{j\in S_{\i}} c_j < D_{\i}/2$,  the given $(\alpha,\beta)$ solution satisfies \eqref{feq:assdual1}. We now prove that it satisfies \eqref{feq:assdual2}.
From the construction of the $\beta$'s the following claims follow.
\begin{claim}\label{fclm:c1}
$\beta_1\leq \beta_2 \leq \cdots \leq \beta_m$.
\end{claim}
\begin{claim}\label{fclm:c2}
$\beta_1D_1 \geq \beta_2D_2 \geq \cdots \geq \beta_mD_m$.
\end{claim}
\begin{proof}
	The only non-obvious case is if $i+1$ is overloaded but $i$ is not: in this case $\beta_{i+1}D_{i+1} = \beta_ic_{j_{i+1}}$. But since $i$ is not overloaded, let $j$ be some job assigned to $i$ with $c_j \leq D_i$.
	By the greedy rule, $c_j \geq c_{j_{i+1}}$, and so $\beta_{i+1}D_{i+1} \leq \beta_iD_i$.
\end{proof}
\noindent
Now fix a job $j$ and let $i$ be the machine it is assigned to. Note \eqref{feq:assdual2} holds for $(i,j)$ and we need to show \eqref{feq:assdual2} holds for all $(i',j)$ too.
I don't see any more glamorous way than case analysis. \smallskip

\noindent
{\bf Case 1: $c_j \leq D_i$.} In this case $\alpha_j = \beta_ic_j$ and $i$ is not overloaded.
Let $i' < i$.  Then we have $\beta_{i'}\min(c_j,D_{i'}) \le \beta_{i'}c_j \leq \beta_ic_j$, where the last inequality follows from Claim~\ref{fclm:c1}.

Now let $i' > i$. If $c_j \leq D_{i'}$, then none of the machines from $i$ to $i'$ can be overloaded. Therefore, $\beta_{i'} = \beta_i$, and so $\beta_{i'}c_j = \beta_ic_j = \alpha_j$.
So, we may assume $c_j > D_{i'}$ and we need to upper bound $\beta_{i'}D_{i'}$. Let $i'' > i$ be the first machine which is overloaded with job $j''$ say.
By Claim~\ref{fclm:c2}, we have $\beta_{i'}D_{i'} \leq \beta_{i''}D_{i''}$. Now note that
$\beta_{i''}D_{i''}  = \beta_{i''-1}c_{j''} = \beta_ic_{j''} \leq \beta_ic_j = \alpha_j$ where the second equality follows since none of the machines from $i$ to $i''-1$ were overloaded. \smallskip

\noindent
{\bf Case 2: $c_j > D_i$.} In this case $\alpha_j = \beta_iD_i$ and $i$ is overloaded. Let $i' > i$. Then, $\beta_{i'}\min(c_j,D_{i'}) = \beta_{i'}D_{i'} \leq \beta_iD_i$ where the last inequality follows from Claim~\ref{fclm:c2}.


Let $i' < i$. Let $i'\leq i'' < i$ be the smallest entry such that $c_j > D_{i''}$. Note that all machines from $i''$ to $i$ must be overloaded implying $\beta_{i''}D_{i''} = \beta_iD_i$.
Since $c_j \leq D_{i'}$ (in case $i' < i''$), we need to upper bound $\beta_{i'}c_j$.
By Claim~\ref{fclm:c1}, $\beta_{i'}c_j \leq \beta_{i''-1}c_j$. Now, if $i''-1$ were overloaded,
then $\beta_{i''}D_{i''} = \beta_{i''-1}D_{i''-1} \geq \beta_{i''-1}c_j$ where the last inequality follows from definition of $i''$. Together, we get $\beta_{i'}c_j \leq \beta_iD_i$.
%For every machine $1\leq i\leq \i$, define $\beta_i := D_\i/D_i$. For every job $j\in S_i$ for $1\leq i\leq \i-1$, define $\alpha_j = \beta_i\min(c_j,D_i)$ and for $j\in S_\i$, define $\alpha_j = c_j$.
%\end{proof}
\def\y{\bar{y}}
\def\z{\bar{z}}
\def\yy{\bar{\bar{y}}}
\begin{lemma}\label{flem:conf-is-uf}
	$\calP_\mathsf{ass}$ is upward-feasible.
\end{lemma}
\begin{proof}
	%\comment{\bf \Large Needs to be written}
	Let $s := (s_1,\ldots,s_n)\in \calP_\mathsf{ass}$ for a certain instance of $Q||C_{min}$ where the jobs have been renamed so that $c_1\leq \cdots \leq c_n$. We need to prove any non-negative vector $t := (t_1,\ldots,t_n)$ s.t. $t\succeq_\suff s$also lies in $\calP_\mathsf{ass}$.
	By the ``hybridization argument'', it suffices to prove the lemma for  $s$ and $t$ differing only in coordinates $\{j-1,j\}$ and $ t_j\ge s_j$ and $t_{j-1} \geq \max(0,s_{j-1} + (s_j - t_j))$.
	Given that, we can move from $s$ to $t$ by changing pairs of coordinates each time maintaining feasibility in $\calP_\mathsf{ass}$.
	
	Let $z$ be the solution for the supply vector $s$; we construct a solution $\z$  for the supply vector $t$ starting with $\z = z$.
	If $\z$ is not already feasible, then it must be because $s_{j-1} \geq \sum_{i\in M} \z_{i,j-1} > t_{j-1}$.
	We select an arbitrary $i\in M$ with $\z_{i,j-1} > 0$ and increase $\z_{ij}$ and decrease $\z_{i,j-1}$ by $\delta$. Since $c_j \geq c_{j-1}$, \eqref{feq:asslp2} remains valid.
	Since the total increase of fractional load of job $j$ is exactly the same as the decrease in that of job $j-1$, and we only need total  decrease $(s_{j-1} -t_{j-1}) \leq t_j - s_j$, at the end we get that $\z$ is feasible wrt supply vector $t$.
	
\end{proof}
%\begin{lemma}
%Suppose $(y_1,\ldots,y_n)\in \calP_\mathsf{ass}$. Let $(\y_1,\ldots,\y_n)$ be a vector such that for all $1\leq i\leq n$, $\sum_{j\leq i} \y_j \geq \sum_{j\leq i} y_i$. Then
%$(\y_1,\ldots,\y_n)\in \calP_{\mathsf{ass}}$.
%\end{lemma}
%\begin{proof}
%By induction, let us assume the lemma is true for all $\y$ with $\y_1 = y_1$ which satisfies the prefix-sum condition.
%Let $\yy$ be the vector with $\yy_1 = y_1$, $\yy_2 = \y_2 + \y_1 - y_1$, and $\yy_i = \y_i$ otherwise.
%Since $\yy\in \calP_\mathsf{ass}$, there is an assignment $z_{ij}$ satisfying \eqref{feq:asslp1}-\eqref{feq:asslp3} with $s_j = \yy_j$.
%We now describe a feasible solution $\z_{ij}$ with $s_j = \y_j$.
%
%Let $\theta := \y_2/\yy_2 \le 1$ since $\y_1 \geq y_1$. Define $\z_{i2} = \theta z_{i2}$ for all $i$, and define $\z_{i1} = z_{i1} + (1-\theta)z_{i2}$.
%For $j=2$, we have $\sum_{i\in M} \z_{i2} = \theta \sum_{i\in M} z_{i2} \leq \theta \yy_2 = \y_2$.
%For $j=1$, we have $\sum_{i\in M} \z_{i1} = \sum_{i\in M} z_{i1} + (1-\theta) \sum_{i\in M} z_{i2} \leq \yy_1 + (1-\theta)\yy_2 = \yy_1 +\yy_2 - \y_2 = \y_1$.
%Since the other $z_{ij}$'s and $\y_j$'s are untouched, $\z_{ij}$ satisfies \eqref{feq:asslp1} with $\y_j$'s.
%
%Now fix a machine $i$. The `increase' in the LHS of \eqref{feq:asslp2} is  $\sum_{j\in J} (\z_{ij} - z_{ij}) \min(c_j,D_i) = (1-\theta)z_{i2}\min(c_1,D_i) -  (1-\theta) z_{i2}\min(c_2,D_i) \geq 0$ since $c_1 \geq c_2$.
%
%\end{proof}
%

\def\Supp{\mathsf{Supp}\xspace}
\newcommand{\barcalS}{\bar{\cal S}\xspace}
\def\cckp{$Q|k_i|C_{min}$\xspace}
\renewcommand{\brp}{{(p)}}
\renewcommand{\br}[1]{{(#1)}}
\renewcommand{\brp}{{(p)}}
\renewcommand{\br}[1]{{(#1)}}
\renewcommand{\bc}{{\bar c}}
\newcommand{\brt}{{(t)}}
\newpage
\section{Approximate Supply Polyhedra for $Q|f_i|C_{min}$}
Let the instance $\calI$ of $Q||C_{min}$  have $m$ machines $M$ with demands $D_1 \geq \cdots \geq D_m$ and cardinality constraints $f_1,\ldots,f_m$, and $n$ types of  jobs $J$ with capacities $c_1 \geq \cdots \geq c_n$.  We assume $D_1/D_m \leq n^C$\comment{deepc: i think this can be made wlog with some std trick}
A supply vector $(s_1,\ldots,s_n)$ indicates the number of jobs of each type available; a supply vector is feasible if together they can satisfy all the demands.
An $\alpha$-approximate supply polyhedra $\calP$ has the following properties: any feasible supply vector lies in $\calP$, and given any integral vector $(s_1,\ldots,s_n) \in \calP$ there is an allocation 
of the $s_j$ jobs of capacity $c_j$ which satisfies every demand up to an $\alpha$-factor.
%We wish to find a convex set/polyhedra which captures all the feasible supply vectors. In particular, any feasible supply vector should be in the set, and given any (integer) supply vector in the set
%there should be an allocation which satisfies the demands to an $\alpha$-factor.
\smallskip

\noindent
A feasible supply vector $(s_1,\ldots,s_n)$ must lie in the following polytope. Let $\Supp$ be a set indicating infinitely many copies of all jobs.
For every machine $i$, let $\calF_i :=\{S\in \Supp: |S| \leq f_i ~\textrm{ and } \sum_{j\in S} c_j \geq D_i\}$ denote all the feasible sets that can satisfy machine $i$. 
Let $n(S,j)$ denote the number of copies of job of type $j$.
	\begin{alignat}{4}
		\calP_\mathsf{conf} && = \{(s_1,\ldots,s_n):  && \notag \\
		&& \quad \forall i \in M,   &\quad  \textstyle \sum_{S\in \calF_i} z(i,S)  =  1 \label{eq:conflp1} \\
		&& \quad \forall j\in J ,  &\quad  \textstyle \sum_{i\in M,S\in \calF_i}  z(i,S)n(S,j) \le  s_j \label{eq:conflp2} \\
		&& \quad \forall i\in M, S\in \calF_i, & \quad z(i,S)   \geq 0 \label{eq:conflp3}  \}
	\end{alignat}
	
\begin{theorem}\label{thm:conflprounding}
	Given $(s_1,\ldots,s_n) \in \calP_\mathsf{conf}$, there is an of assignment $\phi$ of the $s_j$ jobs of capacity $c_j$  to the machines such that for all $i\in M$, 
	$\sum_{j:\phi(j) = i} c_j \geq D_i/\alpha$ for $\alpha = O(\log n)$.
\end{theorem}
\def\calFr{\calF^{\mathsf{rel}}}
\begin{proof}
		Given a feasible fractional solution $\{z(i,S)\}$ to the configuration LP above, we want to  efficiently round it to obtain an integer solution which is an $\alpha$-approximation for the given \cckp instance.
		\medskip 
		
		We call a job of capacity $c_j$ {\em large} for machine $i$ if $c_j \geq \frac{D_i}{16C\log n}$, otherwise it is said to be \emph{small} for machine $i$. 
		For a machine $i$, we define a relaxed collection of feasible sets $\calFr_i$ where $S\in \calFr_i$ if either (a) $S = \{j\}$ and $j$ is large for $i$, or (b) $c_j < \frac{D_i}{8C\log n}$ for all $j\in S$, $|S| \leq f_i$, and $\sum_{j\in S} c_j \geq D_i/2$.
	
		

%		
%			Let $\Delta$ be a large enough constant.
%				For every machine $i$, let $\calFr_i$ denote the subsets $S\subseteq \Supp$ with $|S|\leq f_i$ but $\sum_{j\in S} c_j \geq D_i/4\Delta^2$.
%				
%		%First we round the quantities $c_p$ and $D_j$ down to nearest power of $\Delta$, and let the rounded quantities be $\bc_p$ and $\barD_j$ respectively.
%	   We call a configuration $S\subseteq \Supp$ large for $i$ if $z(i,S) > 0$ and $S$ contains a large job for $i$. Otherwise it is called small.
\paragraph{Partitioning Configurations  and Bucketing Demands.}	
Our first step is  to modify $z$ such that its support $z(i,S) > 0$ for only $S\in \calFr_i$ for all $i$.
For every machine $i$, if $z(i,S) > 0$ and $S$ contains any large job $j$ for $i$, then we replaces $S$ by $\{j\}$. To be precise, we set $z(i,\{j\}) = z(i,S)$ and $z(i,S) = 0$.
We call such singleton configurations {\em large} for $i$; all others are small. Note that after this step, $z(i,S) > 0$ only for $S\in \calFr_i$.
Let $\calF^L_i$ be the collection of large configurations for $i$; the rest $\calF^S_i$ being small configurations.
		Define $z^L(i) := \sum_{S\in \calF^L_i} z(i,S)$ be the total large contribution to $i$, and let $z^S(i) := 1 - z^L(i)$ the small contribution.


 	
	
%	Our algorithm starts with a feasible fractional solution to the configuration LP, and over time modifies the solution to make \emph{most} of the \emph{large configuration assignments} integral $0/1$ while remaining feasible to a slightly \emph{relaxed} configuration LP. Finally, it rounds the small item types to $0/1$ using the classical algorithm of Lenstra, Shmoys and Tardos~\cite{LST}. Overall, the rounding algorithm requires several steps which we next present one by one.
%	

%	We also slightly expand the set of small and large configurations to satisfy some weaker conditions involving $\bc_p$ and $\barD_j$ as follows: we define a set $\barcalS^L_j$ of \emph{relaxed} large configurations to be $\barcalS^L_j = \{ \{p\} \text{ s.t } \bc_p \geq \frac{\barD_j}{4 \Delta^2}  \}$. Similarly, the expanded collection of \emph{relaxed} small configurations for demand $j$ is denoted by $\barcalS^S_j = \{ ((1,n'_1), (2,n'_2), \ldots, (P,n'_P)) \text{ s.t } \sum_{p \in [P]} \bc_p n'_p \geq \frac{\barD_j}{\Delta} \text{ and } \sum_{p \in [P]} n'_p \leq f_j \text{ and } n'_p \leq n_p \text{ for all } 1 \leq p \leq P \}$.
%	
%	\begin{claim}
%		The relaxed large and small configurations satisfy $\calS^L_j \subseteq \barcalS^L_j$ and $\calS^S_j \subseteq \barcalS^S_j$.
%	\end{claim}
%	
%	\begin{proof}
%		Since we scale down the demands and capacities by a factor of $\Delta$, it is easy to see.
%	\end{proof}
%	
%	\begin{corollary}
%		The solution $\{y(S,j)\}$ is feasible to the relaxed configuration LP where we allow relaxed large and small configurations.
%	\end{corollary}
%	
%	
%	\medskip % \noindent {\bf Step 2: ``Bucketing'' Demand Requirements.}
	
	The next step of our algorithm is to partition the demands into buckets depending on their requirement values $D_i$. To this end, we say that demand $i$ belongs to \emph{bucket $t$} if
	$2^{t-1} \leq D_i < 2^t$  (we assume wlog by scaling that  the smallest demand is $1$). We let $B^\brt$ to denote the bucket $t$. 
		Note that the number of buckets $K \leq C\log n$; this drives the approximation factor.
We make one observation. 
\begin{claim}\label{clm:c001}
	For any $t$, let $i$ and $i'$ be two machines in $B^\brt$ and let $f_i \leq f_{i'}$. 
	Let $z(i,T) > 0$ for some small configuration for $i$.
Then $T\in \calFr_{i'}$ and $\sum_{j\in T} c_j \geq D_{i'}/2$.
\end{claim}
\begin{proof}
Note that since $z(i,T) > 0$, we have $\sum_{j\in T} c_j \geq D_{i} \geq 2^{t-1} \ge D_{i'}/2$. Furthermore, for any $j\in T$, we have $c_j \leq \frac{D_{i}}{8C\log n} \le \frac{2^t}{8C\log n}$.
Therefore any other machine $i'\in B^\brt$, $T$ satisfies two conditions of being in $\calFr_{i'}$.
Now if $f_{i'} \geq f_i$, we get $|T| \leq f_{i'}$ as well. 
\end{proof}
Before describing our subroutines, we make a few definitions. All of these are with respect to a solution $z$. 
A machine $i$ is called {\em rounded} if there exists $S\in \calFr_i$ with $z(i,S) = 1$. We let $\calR$ denote the rounded demands.
The remaining machines are of three kinds:  {\em large} ones with $z^L(i) = 1$, {\em hybrid} ones with $z^L(i) \in (0,1)$ and {\em small} ones with $z^L(i) = 0$. 
Let $\calL,\calH,\calS$ denote these respectively. 


%Every job $j$ participates in large configurations $z(i,\{j\})$ and small configurations $z(i,S)$ with $j\in S$. We let $z^L(j) := \sum_{i: j \textrm{ is large for } i} z(i,\{j\})$. 
%We let $\calL$ be the set of {\em super-large} jobs with $z^L(j) = s_j$. %This set can only increase in our algorithm. 
%Finally, we maintain a set of {\em rounded demands} $\calR$ which consists of $(i,S)$ with $z(i,S) = 1$ and $S\in \calFr_i$. That is, the jobs in $S$ are assigned to machine $i$ and both $i$ and $S$ can be removed from consideration.
%	We say that a job $j$ is \emph{large} with respect to bucket $t$ if its size $\bc_p$ is at least $\frac{\Delta^{t-2}}{4}$, i.e., it can belong to a relaxed large configuration for a bucket-$t$ demand.  %For a demand $j$, let $\barcalS^L_j$ and $\barcalS^S_j$ denote the large and the small configurations corresponding to $j$ respectively.
	
%	We now further define some useful variables which can be derived from $\{y(S,j)\}$. Indeed, fix a solution $y$ to the relaxed configuration LP. For a demand $j$, define $y^L(j)$ to be $\sum_{(S,j) \in \barcalS^L_j} y(S,j)$, i.e., the total fractional extent to which $j$ is satisfied by large configurations. Define $y^S(j)$ similarly corresponding to small configurations. Likewise, we extend this definition to buckets of demands as well. For a bucket $t$, define $y^L(B^\brt)$ to be $\sum_{j \in B^\brt} y^L(j)$, and define $y^S(B^\brt)$ analogously. Now, for a \emph{type $p$} of items, define $y(p)$ to be $\sum_{(S,j)} n(S,p) y(S,j)$, the fractional extent to which this type is assigned across all items in the solution $y$. From the LP feasibility constraint, we know that $y(p) \leq n_p$. Finally, given a bucket $t$, let $I^\brt$ denote the set of item types which are large for bucket $t$. For a bucket $t$ and item type $p \in I^\brt$, define $y^L(p,t)$ as $\sum_{(\{p\}, j): j \in B^\brt} y(\{p\}, j)$, which is the extent to which items of type $p$ are assigned as large demands to demands in bucket $t$.
%	
\paragraph{Subroutine: {\sf FixBucket}($t$).} This takes a bucket $t$ with more than one hybrid machine, and modifies the $z$-solution such that
there is at most one hybrid machine in $t$. Other machines in other buckets are unaffected.

%Let 
%	{\bf Step 2: Rounding Large Demand Assignments.} In this step, we modify the LP solution 
% such that for each bucket $t$, there is at most one hybrid machine $i\in B^{(t)}$ with  $z^L(i) \in (0,1)$.
%% and at most one $j$ with $c_j \geq D_i/4\Delta^2$  with $z(i,\{j\}) \in (0,1)$. 
%% The flip side is that we may introduce variables $z(i,S)$ for sets $S$ where $|S|\leq f_i$ but $\sum_{j\in S} c_j \geq D_i/4\Delta^2$.
% %\emph{at most one strictly fractional variable} $0 < y_{S,j} < 1$ over all $j \in B^\brt$ and $S \in \barcalS^L_j$. 
% To this end, we repeatedly perform the following steps, starting from the smallest bucket $t$ onwards. 
 
% 
% Before we start the steps for bucket $t$, we always maintain the following invariant holds for every bucket $t' < t$ (which is vacuously true for the smallest ($t=1$) bucket):
%	\begin{framed}
%		\begin{itemize}
%			\item[({\bf I})] For each bucket $t' < t$, there is at most one  demand $i_{t'} \in B^{(t')}$ %\setminus \calR$ 
%			such that $z^L(i_{t'}) \in (0,1)$. Further, if such a demand $i_{t'}$ exists, then $f_{i_{t'}} \leq f_{i}$ for all $i \in B^{(t')} \setminus \calR$.
%		\end{itemize}
%	\end{framed}
	
%	Suppose this invariant holds for all buckets upto $t-1$. Now we describe the iteration for bucket $t$.
	
Among the hybrid machines $ B^\brt$, let $i$  be the one with the smallest $f_i$. Let $i'$ be any other hybrid machine. We know there is at least one more.
%	
%	
%	%\medskip \noindent {\bf Step 2a: Intra-Bucket Rounding.}  
%	Define a total order $\prec$ on $B^\brt \setminus \calR$, where $i \prec i'$ if $f_{i} \leq f_{i'}$.  
%	%We now ensure that the demands with the smallest $f_j$ value get preference when it comes to being satisfied by large configurations. 
%	Suppose there exists $i,i' \in B^\brt \setminus \calR$ such that the following conditions hold: (i) $i\prec i'$, (ii) $z^L(i)$ and $z^L(i')$ are both in $(0,1)$. 
	We now \emph{modify} $z$ as follows.
	Since $z^L(i') > 0$, there exists a large configuration $\{j'\}$ for $i'$ with $z(i',\{j'\}) > 0$. Similarly, since $z^L(i) < 1$, there must exist a {\em small} configuration $T\in \calFr_i$ such that $z(i,T) > 0$.
	%To this end, let $(S,j') \in \barcalS^L_{j'}$ be a large configuration  with $y(S,j') > 0$. Now, since $y^l(j) < 1$, we know that there is a small configuration $(T,j) \in \barcalS^S_j$ with $y(T,j) > 0$. 
	By Claim~\ref{clm:c001}, note that $T\in \calFr_{i'}$ as well.
	We then perform the following change: increase $z(i,\{j'\})$ and $z(i',T)$ by $\delta$,  and decrease $z(i',\{j'\})$ and $z(i,T)$ by $\delta$, for a $\delta > 0$ such that one of the variables becomes 0 or 1.
	Note that this keeps \eqref{eq:conflp1} and \eqref{eq:conflp2} maintained. 	%In particular, no job $j$ leaves $\calL$.
	
	
	We keep on performing this process as long as possible; since we always transfer large configuration assignments to demands which appear earlier in the total order, this process will stop at some point. At this point, we add whichever demands are integrally assigned by large configurations to 
	the set $\calR$, i.e., all $i \in B^\brt$ for which   $z(i,S) = 1$ for some $S\in \calFr_i$ are added to $\calR$.
%	
%	\begin{claim} \label{cl:swap1}
%		The modified solution above is feasible for the relaxed LP where support $z(i,S)$ is allowed for $S\in \calFr_i$.
%	\end{claim}
%	\begin{proof}
%		Since $i$ and $i'$ are in the same bucket, $c_{j'} \geq D_i/4\Delta^2$ as well, and so $\{j'\} \in \calFr_i$. Similartly, $\sum_{j\in T} c_j \geq D_{i'}/\Delta$. Furthermore, since $i\prec i'$, we have $|T| \leq f_{i} \leq f_{i'}$. Thus, the support of $z$ stays in the relaxed support. Finally, the LHS of \eqref{eq:conflp2} remains unchanged in the above operation for all jobs.
%%		
%%		
%%		To see why the modified solution is feasible for the relaxed configuration LP, observe that $(S,j)$ is also a valid relaxed large configuration because $j$ and $j'$ are of the same bucket $t$ and hence have the same rounded demand requirement of $\Delta^t$. Similarly, observe that $(T,j')$ is a valid small configuration for $j'$ because $|T| \leq f_{j} \leq f_{j'}$, and the remaining constraints for being a small configuration (i.e., the total capacity of items in $T$ should exceed the $\barD_{j'}/\Delta$, and the numbers $n(T,p) \leq n_p$ for all types) are satisfied trivially since they enforce the same constraints for $j$ and $j'$. As a result, since we increase $y(S,j)$ and decrease $y(T,j)$ at the same rate, constraint~\cref{eq:config1} is satisfied for $j$. Similarly, since we increase $y(T,j')$ and decrease $y(S,j')$ at the same rate, constraint~\cref{eq:config1} is satisfied for $j'$. Likewise, since we increase $y(S,j)$ and decrease $y(S,j')$ at the same rate, constraint~\cref{eq:config2} is satisfied for the item type corresponding to the large item in $S$. Finally, since we increase $y(T,j')$ and decrease $y(T,j)$ at the same rate, constraints~\cref{eq:config2} are satisfied for all item types corresponding to items in $T$.
%	\end{proof}
%	
\begin{claim}\label{clm:step2}
	Fix Bucket on $t$ produces a solution with at most one hybrid machine in $B^\brt$.
\end{claim}
	




%	The following claim encapsulates the state of affairs after taking care of the $t$th bucket.
%	
%	\begin{claim} \label{cl:step3a}
%		Let $\alpha := z^L(B^\brt \setminus \calR)$ denote the total fractional assignment of large configurations to the non-integrally rounded demands in $B^\brt$. Then the following condition holds at the termination of the above iterative process: suppose we arrange the demands in $B^\brt \setminus \calR$ according to the order $\prec$. Let this ordering be $i_1, i_2, \ldots,i_r$, and let $k$ denote $\lfloor \alpha \rfloor$. Then we have that $z^L(i_u) = 1$ for $u=1, \ldots, k$, $z^L(i_u) = 0$ for $u > k+1$, and lastly, $z^l(i_{k+1})$ is equal to the fractional part of $\alpha = \alpha - \floor{\alpha}$.
%	\end{claim}
%	
%	\begin{proof}
%		Complete...
%	\end{proof}
	
%	\begin{claim}
%		After step (3a), the invariant (I) continues to hold for all buckets $t' < t$.
%	\end{claim}
%	
%	\begin{proof}
%		This is true because we do not alter the assignments in buckets $p' < p$ in the step.
%	\end{proof}
	
	
%	To summarize, after the above step, we have ensured that there is at most one demand (i.e., $j_{k+1}$ from the claim above) of bucket $t$
%	with strictly fractional $y^l(j)$ value in the modified LP solution. However, even the demands with $y^l(j) = 1$, i.e.,  $j_1, \ldots, j_k$,  may currently be satisfied by many large configurations. In the next step of the algorithm, we iteratively perform a \emph{second transformation} to ensure that each such demand is in fact \emph{integrally satisfied by a single large item}, and hence we can add such demands to $\calR$ and proceed.
%	
%	
	%{\Huge NEED TO FIX BELOW}
\paragraph{Subroutine: {\sf FixLargeMachine}($i$).}
This takes input a large machine $i\in B^\brt\setminus \calR$ with  $z^L(i) = 1$ and modifies $z$ such that  at the end $i$ enters  $\calR$.
Since $i\notin \calR$ in the beginning, there must exist then two large configurations with $z(i,\{j_1\}) \in (0,1)$ and $z(i,\{j_2\}) \in (0,1)$. 
	Let $j_1$ be the job with the smallest capacity among all large configurations $(i,\{j\})$ with $z(i,\{j\}) > 0$.
%	Wlog, assume $c_{j_1} \le c_{j_2}$.
	Two cases arise. In the simple case, there exists no $i'\notin \calR$ and $S'\in \calFr_{i'}$ with $z(i',S') > 0$ and $j_1 \in S'$. That is, no other machine fractionally claims the job $j_1$.
	Since $s_{j_1}$ is an integer, we have slack in \eqref{eq:conflp2} and therefore we can round up $z(i,\{j_1\}) = 1$ (zeroing out all other $i$'s $z(i,S)$'s ) without violating \eqref{eq:conflp2}. We then  add $(i,\{j_1\})$ to $\calR$.
	
	Otherwise, there exists a machine $i'$ (which could be in a different bucket) and a set $S\in \calFr_{i'}$ such that $z(i',S) \in (0,1)$ and $j_1 \in S$. 
	Now define the set $T$ as follows. If $c_{j_2} > \frac{D_{i'}}{8C\log n}$, then $T = \{j_2\}$; otherwise $T = S - j_1 + j_2$. Note that in either case $T \in \calFr_{i'}$.
	In the first case, $j_2$ is large for $i'$. In the second case, $|T| = |S|$ and $\sum_{j\in T} c_j \ge \sum_{j\in S} c_j$ since $c_{j_2} \geq c_{j_1}$ by choice of $j_1$.
	
	We modify $z$-as follows. We decrease $z(i,\{j_2\})$ and $z(i',S)$ by $\delta$ and increase $z(i,\{j_1\})$ and $z(i',T)$ by $\delta$ till one of the values becomes $0$ or $1$. 
	As before, this preserves the LHS of \eqref{eq:conflp1} and can only decrease the LHS of \eqref{eq:conflp2} (for jobs $j\in S\setminus j_1$ if $T = \{j_2\}$).
%	Also note that $z^L(j)$ can only increase for any job; in particular no job leaves $\calL$. {\Huge NOT CORRECT}
   This process ends with assigning $z(i,\{j_1\}) = 1$ and we add $(i,\{j_1\})$ to $\calR$.
   
   \begin{claim}\label{clm:002}
Subroutine Fix Large  Machine $i$ modifies the LP solution and adds $i$ to $\calR$. 
   	\end{claim}
   \noindent
   Note that Fix Large Machine can make a small machine $i'$ hybrid or large for its bucket since $z^L(i')$ could potentially increase.  
   We run the following while loop in Step 1 of te algorithm.
   
   \paragraph{Step 1: Taking care of large machines}
   \begin{itemize}[noitemsep]
   	\item[] While $\calL$ is non-empty: 
   	\begin{itemize}[noitemsep]
   		\item If $i\in \calL$, then {\sf Fix-Large-Machine}($i$). Note that $i$ enters $\calR$ after this. This can increase the number of hybrid machines across buckets.
   		\item For all $1\leq t\leq K$, if $B^\brt$ contains more than one hybrid machine, then {\sf Fix-Bucket}($t$). This can increase the number of machines in $\calL$.
   	\end{itemize}
   \end{itemize}
   The above while loop terminates in at most $m$ iterations, since the first bullet point adds a machine to $\calR$.
%   If there are more than one hybrid in any bucket, we apply Step 2 again. Since Step 3 moves machines to $\calR$, this process can go on for at most $m$ steps. After doing this sequence of steps, we have the following scenario.
%   
   \begin{claim}\label{clm:003}
   	At the end of {\bf Step 1}, we have for every bucket $t$, at most one  $i\in B^\brt\setminus\calR$ has $z^L(i) \in (0,1)$ and the rest have $z^S(i) = 1$. Furthermore, for every $i\in \calS$ and $z(i,S) > 0$, 
   	we have $\sum_{j\in S} c_j \geq D_i/2$.
   	\end{claim}
   	

   
   
%	We need to make sure that $S - j_1 + j_2$  lies in 
%	Note that $S$ could already have one (or more) copies of $j_2$; we have just increases $n(S,j_2)$ by additive $1$. Since $c_{j_1} \leq c_{j_2}$, the total capacity of $S-j_1+j_2$ exceeds that of $S$. The cardinality of both sets are the same. Therefore, they are valid sets to put $z$-mass on. We keep doing this modification till one of the fractional values become integral. Since for $i$, mass moves from large configurations $(i,\{j\})$ of larger $c_j$-capacity to smaller $c_j$-capacity, this process terminates at one point. At this point, we must have $z(i,\{j_1\}) = 1$; we add $(i,\{j_1\})$ to $\calR$.
%	
	%We perform Step 3 for all large machines. After Step 3, every bucket $B^\brt\setminus \calR$ contains at most one hybrid machine $i_t$ with $z^L(i_t) \in (0,1)$ and small machines $i$ with $z^S(i) = 1$.
%	At the end of it, for every machine $i$ there can be at most one large configuration $\{j\}$ with $z(i,\{j\}) > 0$.\medskip
%	
%	We repeat this step while there  exists a  demand $j \in B^\brt$ for which there are two large configurations, say $(\{p_1\}, j), (\{p_2\}, j)$ with positive $y$ assignments, and suppose $\bc_{p_1} < \bc_{p_2}$ without loss of generality. There are now two cases depending on whether item type $p_1$ is contained in any other non-integrally assigned configuration or not.
%	
%	\medskip \noindent {\bf Case 3b(i): There exists demand $j' \notin \calR$ and configuration $(T,j')$ such that $n(T,p_1) \geq 1$ and $y(T,j') > 0$.}
%	In this case, note that $j'$ may not belong to bucket $t$. Now, if item type $p_2$ is a small item type for demand $j'$, then let $T'$ denote the multi-set obtained from $T$ be adding one item of type $p_2$ and removing one item of type $p_1$. Since $p_1$ has smaller size than $p_2$, clearly the total size of items in $T'$ will exceed that of $T$. On the other hand, if $p_2$ is a large item type for $j'$, then let $T' = \{p_2\}$ and we use the large configuration of $(T',j')$. In both cases, we perform the following updates at a uniform rate: increase $y(\{p_1\},j)$ and $y(T',j')$ and decrease $y(\{p_2\},j)$ and $y(T,j')$, and stop when one of them reaches $0$ or $1$. If in the process some demand is integrally assigned, we include it in $\calR$.
%	
%	\medskip \noindent {\bf Case 3b(ii): There exists no demand $j \notin \calR$ and configuration $(T,j')$ such that $n(T,p_1) \geq 1$ and $y(T,j') > 0$.}
%	In this case, we have at least $1$ free capacity of item type $p_1$, and so we set $y(\{p_1\},j)= 1$ and all other $y(S,j) = 0$ for $S \neq \{i_1\}$. We add $j$ to the rounded set $\calR$ and repeat this step (3b).
%	
%	
%	\begin{claim} \label{cl:step3b}
%		If $\{y(S,j)\}$ is a feasible to the relaxed configuration LP, then after a single execution of either case 3b(i) or case 3b(ii), the modified solution remains feasible for the relaxed configuration LP, and moreover, invariant~(I) continues holds for all buckets $\leq t-1$.
%	\end{claim}
%	\begin{proof}
%		Again, it is easy to check the feasibility of LP solution much like~\Cref{cl:swap1}. We also claim that the invariant~[(I1)] continues to hold (for bucket $t-1$) during this process. Indeed, we only need to be worried for the case when $j'$ is of bucket less than $t$, as otherwise we are not modifying the configurations in lower buckets. In this case it will be the unique fractional demand of this bucket guaranteed by invariant~[I1]. Further $(T',j')$ will also be a large configuration for this demand $j'$. Therefore, $y^l(j')$ does not change, and so, the invariant still holds.
%	\end{proof}
%	
%	To summarize, when the process ends, all demands $j \in B^\brt$  except perhaps for one demand satisfy the property that $y^l(j)$ is either 0 or 1. Further, for any demand $j$, there is at most one large configuration $(S,j)$ with positive $y(S,j)$ value. If a large configuration $(\{i\},j)$ satisfies $y(\{i\},j)=1$, then we include $j$ in $\calR$. Thus, for bucket $t$, there is at most one large configuration $(S,j), j \in B^\brt \setminus \calR$  such that $y(S,j)$ strictly between $0$ and $1$. Moreover, since this fractional demand is the demand $j_{k+1}$ identified in Claim~\ref{cl:step3a}, it has the smallest $f$ among all $j \in B^\brt \setminus \calR$. Thus, we satisfy the invariant for bucket $t$ as well. Observe that right there is exactly one large configuration with positive $y$ value serving $j_k$. However, when we carry out the step (3b) for a bucket larger than $t$, we may redistribute the large assignment of $j$ over several large configurations, but $y^l(j)$ will remain unchanged.
%	

%	\paragraph{Step 2:} This takes input a hybrid machine $i\in B^\brt\setminus \calR$ with the property there exists at least two jobs $j_1$ and $j_2$
%	which are large for $i$, $z(i,\{j_1\}) > 0$ and $z(i,\{j_2\}) > 0$, and neither $j_1$ or $j_2$ are in $\calL$. This subroutine, which is very similar to Fix Large Machine, modifies the LP solution and at the end at least one new job enters $\calL$.
%	
%Let $j_1$ be the smallest capacity job among jobs $j$ with $z(i,\{j\}) > 0$ and $j\notin \calL$. Let $j_2$ be another such job which is guaranteed to exist.
%		%Pick any hybrid  machine $i$ and, after renaming, let $j_1,j_2,\ldots,j_r $ be the jobs in non-decreasing capacity order,  with $z(i,\{j_t\})>0$ for all $1\leq t\leq r$. 
%%	Starting with $j_1$, we check if there exists any other machine $i'$ (small or hybrid) and {\em small} configuration $S$ for $i'$ containing $j_1$.
%Since $j_1\notin \calL$, there must exist some machine $i'$  with $z(i',S) > 0$ and $j_1\in S$ for some small configuration $S$. We now move mass precisely as in Step 3. 
%Define a set $T$ as follows. If $c_{j_2} > D_{i'}/4\log n$, then $T = \{j_2\}$; otherwise $T = S - j_1 + j_2$. Note that in either case $T \in \calFr_{i'}$.
%	In the first case, $j_2$ is large for $i'$. In the second case, $|T| = |S|$ and $\sum_{j\in T} c_j \ge \sum_{j\in S} c_j$ since $c_{j_2} \geq c_{j_1}$ by choice of $j_1$. We decrease $z(i,\{j_2\})$ and $z(i',S)$ by $\delta$ and increase $z(i,\{j_1\})$ and $z(i',T)$ by $\delta$ till one of the values becomes $0$ or $1$. 
%	As before, this preserves the LHS of \eqref{eq:conflp1} and can only decrease the LHS of \eqref{eq:conflp2} (for jobs $j\in S\setminus j_1$ if $T = \{j_2\}$).
%	Unlike Step 3, once we are done with job $j_1$, we won't have $z(i,\{j_1\}) = 1$; either $j_1$ enters $\calL$ or all other jobs enter $\calL$.
%	
%	We keep performing this procedure; once again this procedure increases large-configuration mass on lower capacity configurations, and therefore it will terminate.
%	After termination, we have the following property: for any hybrid machine $i$ and jobs $j_1,\ldots,j_r$ with $z(i,\{j_t\}) > 0$, there can be at most one job $j_t$ which appears
%	in a small configuration of some other machine $i'$ (hybrid or small). For the hybrid machine $i_t$, let this job be $j_t$. Note that $i_t$ and $i_{t'}$ could have the same $j_t$. \smallskip
%	
%	Next, for every hybrid machine $i$ with $z^L(i) < 1 -1/2K$, we simply zero-out their large configuration mass. Note that $z^S(i) > 1/2K$ for all such machines. We rename these machines small.
%	Let $i_1,\ldots,i_{K'}$ be the hybrid machines that remain with $K' \leq K$. Let $J'$ be the jobs which are large for machine in $K'$.
%	Note that $|J'| \geq K'$.

\paragraph{Step 2: Taking care of hybrid machines.} 
Let $\calH$ be the set of hybrid machines at this point. We know that $|\calH| \leq K \leq C\log n$ since each bucket has at most one hybrid machine.
For any machine $i\in \calH$ with $z^L(i) \le 1-1/K$, we zero-out all its large contribution. More precisely, for all $j$ large for $i$ we set $z(i,\{j\})= 0$.
Note that \eqref{eq:conflp1} is no longer true, but it holds with RHS $\geq 1/K$. Note that these machines enter $\calS$.

At this point, we have $\calH$ where every $i\in \calH$ has $z^L(i) > 1-1/K$. Let $K' := |\calH|$. Let $J'$ be the set of jobs $j$ which are large for some machine $i\in \calH$ and $z(i,\{j\}) > 0$.
Let $G$ be a bipartite graph with $\calH$ on one side and $J'$ on the other and we draw an edge $(i,j)$ iff $j$ is large for $i$.
\begin{claim}
	There is a matching in $G$ saturating all $i\in \calH$.
\end{claim}
\begin{proof}
Pick a subset $\calH' \subseteq \calH$ and let $J''$ be its neighborhood in $G$. We need to show $\sum_{j\in J''} s_j \geq \calH'$.
Since $z$ satisfies \eqref{eq:conflp2}, we get
\[
\sum_{j\in J''} s_j \geq \sum_{j\in J''} \sum_{i\in \calH'} z(i,\{j\}) = \sum_{i\in \calH'} \sum_{j\in J''} z(i,\{j\})  > (1-1/K)|\calH'| \geq |\calH'| - 1
\]
The inequality follows since $J''$ is the neighborhood of $\calH'$ and the fact that $z^L(i) > 1-1/K$ for all $i\in \calH$.
The claim follows since $s_j$'s are integers.
\end{proof}
If machine $i\in \calH$ is matched to job $j$, then we assign $i$ this job and add $i$ to $\calR$. Let $J_M \subseteq J'$ be the subset of jobs allocated; note $|J_M| \leq K$.
After this point we have only small machines remaining. For every $i\in \calS$ and every small configuration $S$ with $z(i,S) > 0$, we move this mass to $z(i,S\setminus J_M)$.
Note that $\sum_{j\in S\setminus J_M} c_j \geq  \sum_{j\in S} c_j - |J_M|\cdot \frac{D_i}{8C\log n} \geq 3D_i/8$ where we use the fact that $\sum_{j\in S} c_j \geq D_i/2$ (by Claim~\ref{clm:003}) and $K\leq C\log n$.


\begin{claim}\label{clm:007}
At the end of {\bf  Step 2}, we have a set of residual machines $\calS$ and a set of residual jobs $J_{res}$  and a solution $z(i,S)$ where 
\begin{enumerate} [noitemsep]
	\item For all $i\in \calS$ we have $z(i,S) > 0$ iff $|S| \leq f_i$, $\sum_{j\in S} c_j \geq 3D_i/8$, and $c_j < \frac{D_i}{8C\log n}$ for all $j\in S$.
	\item $\forall i \in \calS, ~ \textstyle 1 \ge \sum_{S} z(i,S)  \geq   1/K \geq \frac{1}{C\log n}$.
	\item $\forall j\in J_{res}, ~ \textstyle \sum_{i} z(i,S)n(S,j)  \leq  s_j$.
	
\end{enumerate}
\end{claim}
\paragraph{Step 3: Taking care of Small Machines.} We convert the LP solution in Claim~\ref{clm:007} to an assignment LP solution in the standard way.
For every $i\in \calS$ and $j\in J_{res}$ define $z_{ij} = \sum_{S} z(i,S)n(S,j)$. Note that this satisfies the constraint of the assignment LP:
	\begin{alignat}{4}
		&& \quad \forall j \in J_{res},   &\quad  \textstyle \sum_{i\in \calS} z_{ij}  \leq  s_j \notag  \\
		&& \quad \forall i\in \calS ,      &\quad  \textstyle \sum_{j\in J_{res}}  z_{ij}c_j \geq \frac{3D_i}{8C\log n} \notag\\
	&& \quad \forall i\in \calS ,      &\quad  \textstyle \sum_{j\in J_{res}}  z_{ij} \leq f_i \notag\\ 
		&& \quad \forall i\in \calS, j\in J_{res} ~\textrm{with}~ c_j \geq \frac{D_i}{8C\log n}, & \quad z_{ij}   =  0   \notag
	\end{alignat}
	The last equality follows from point $1$ of Claim~\ref{clm:007}.
The first inequality follows from point 3 of Claim~\ref{clm:007}. To see the second and third point, note 
that for any $i\in \calS$,
\[
\sum_{j\in J_{res}} z_{ij}c_j = \sum_j \sum_S z(i,S)n(S,j)c_j = \sum_S z(i,S) \sum_j n(S,j)c_j \geq \frac{1}{C\log n}\cdot \frac{3D_i}{8}
\]
	and,
\[
\sum_{j\in J_{res}} z_{ij} = \sum_j \sum_S z(i,S)n(S,j) = \sum_S z(i,S) \sum_j n(S,j) \leq f_i
\]
since for any $S$, $\sum_{j\in S} n(S,j) \leq f_i$. \smallskip

Now we use Theorem~\ref{thm:shmoystardos} to find an allocation of $J_{res}$ to machines $\calS$ such that every machine gets capacity $\geq \frac{D_i}{4C\log n}$.

\end{proof}


\subsection{Assignment LP for $Q|f_i|C_{min}$}
Suppose we are given $m$ machines $M$ with cardinality constraints $f_1,\ldots,f_m$, and $n$ types of  jobs $J$ with capacities $c_1 \geq \cdots \geq c_n$. 
Let $(s_1,\ldots, s_n)$ be a supply vector, that is, there are $s_j$ copies of job $j$. 
Suppose there exists a feasible solution to the following LP.
\begin{alignat}{4}
	&& \quad \forall j\in J,  &\quad  \textstyle \sum_{i\in M} z_{ij} \leq s_j \label{eq:asslp1} \\
	&& \quad \forall i\in M,  &\quad  \textstyle \sum_{j\in J} c_jz_{ij}  \geq D_i \label{eq:asslp2}\\
	&&\quad \forall i\in M, & \quad \textstyle \sum_{j\in J} z_{ij}  \leq  f_i \label{eq:asslp3}\\
	&& \quad \forall i\in \calS, j\in J_{res} ~\textrm{with}~ c_j \geq C_i, & \quad z_{ij}   =  0   \label{eq:asslp4}
\end{alignat}
\def\zz{z^{\mathsf{int}}}
%For the correct guess, the above LP is feasible. If so, using the fact that $c_p \leq T_i$ for all $i\sim p$, we can get a feasible solution with a slight hit in the demand.
\begin{theorem}\label{thm:shmoystardos}
	If \eqref{eq:asslp1}-\eqref{eq:asslp4} is feasible, then there is an integral assignment $\zz_{ij}$ which satisfies \eqref{eq:asslp1}, \eqref{eq:asslp3} and \eqref{eq:asslp4}, and 
	$\quad \forall i\in M, ~~ \sum_{j\in J} c_j\zz_{ij}  \geq D_i - C_i$.
\end{theorem}
\begin{proof} {\large NEEDS BETTER WRITING} We repeat the argument of Shmoys and Tardos~\cite{bibid}.
Form $\floor{\sum_{j\in J} z_{ij}} \le f_i$ copies of every machine; let $N_i$ be the copies of machine $i$. Order the jobs with multiplicities s.t. $c_1 \geq c_2 \geq \cdots \geq c_N$ where $N = \sum_j s_j$.
Modify $z_{ij}$ to get an assigment $z_{ij}$ for $i\in \cup N_i$ and $j\in [N]$ as follows. We do this for one machine $i$.

Given $z_{ij}$'s we form $|N_i| + 1$ groups $S_1,\ldots, S_{|N_i|},S_{|N_i|+1}$ with $\sum_{j\in S_t} z_{ij} =1$ for all $1\leq t\leq |N_i|$ and $\sum_{j\in S_t} z_{ij} < 1$ for $t = |N_i|+1$.
Note that $\sum_{j\in S_t} z_{ij}c_j < c_{j'}$ for $j'\in S_{t-1}$. When we do this modification for all machines, we get a fractional matching solution where all the $N_i$ copies get fractional value $1$ but the jobs are at most $1$.
So, there is an integral matching. The total integral load on machine $i$ is at least $\sum_{t>1} \sum_{j\in S_t} z_{ij}c_j \geq D_i - C_i$ since $z_{ij} = 0$ for $c_j > C_i$.
	
	Cardinality constraint vacuously satisfied.
\end{proof}

\subsection{Lower Bound}
We show that the configuration LP for the problem of allocating jobs to machines of different speeds to maximize the minimum processing time has super-constant integrality gap.
\begin{theorem}
	The integrality gap of CLP is $\Omega(\log n/\log\log n)$.
\end{theorem}
\begin{proof}
	\def\M{\mathcal{M}}
	Fix $K$. We present an instance $\calI_K$ for which configuration LP is feasible but any integral allocation must violate the demand of some machine by factor $K$.
	
	First we describe the machines in $\calI_K$.
	\begin{enumerate}
		\item There is $1$ machine $M_0$ with $D(M_0) = 1$ and $f(M_0) = 1$.
		\item There are $K$ machines $M_1,\ldots,M_K$ with $D(M_i) = K^{-i}$ and $f_i := f(M_i) = K^{2K + 1}\cdot K^{-2i}$.
		\item There are $K$ {\bf classes} of machines $\M_1,\M _2,\ldots, \M _K$. Machines in the same class are equivalent. 
		There are $f_i$ machines in $\M_i$ and they are numbered $N^{(i)}_1,\ldots,N^{(i)}_{f_i}$.
		Each machine $N$  in class $i$ has $D(N) = \frac{1}{f_iK^i} = K^{-(2K+ 1)}\cdot K^i$ and $f(N) = 1$.
	\end{enumerate}
	Now we describe the jobs.
	\begin{enumerate}
		\item There are $K$ ``big jobs'' $J_1,\ldots,J_K$ with $c(J_i) = 1$.
		\item There are $K$ other types of jobs of the same capacity. Job $J$ of type $i$ has capacity $c(J) = c_i :=  \frac{1}{f_iK^i} = K^{-(2K+1)}K^i$  and there are $n_i := f_i (1+1/K) = (K+1)K^{2K}K^{-2i}$ of them.
		We divide these $n_i$ jobs into two sets $S_i \cup T_i$ where $|S_i| = f_i$ and $|T_i| = f_i/K$. We order the jobs in $S_i$ arbitrarily and call them $P^{(i)}_1,\cdots,P^{(i)}_{f_i}$.
	\end{enumerate}
	So, the total number of machines in $\calI_K$ are $1 + K + \sum_{i=1}^K f_i  \leq K^{2K}$ and the number of jobs is $K + (1+1/K)\sum_{i=1}^K f_i \approx K^{2K}$. 
	
	\begin{lemma}
		The Configuration LP is feasible.
	\end{lemma}
	\begin{proof}
		We describe a fractional solution.
		\begin{enumerate}
			\item For machine $M_0$ we satisfy as follows: set  $y(M_0,J_i) = 1/K$ for $i=1,\ldots,K$. Note $c(J_i) \geq D(M_0)$ and $|J_i| = 1 = f(M_0)$. 
			\item For machine $M_i$ we satisfy as follows: set $y(M_i,J_i) = 1-1/K$ and $y(M_i,S_i) = 1/K$. Recall $S_i$ are the $f_i$ jobs of type $i$. 
			\begin{itemize}
				\item Note $c(J_i) = 1 \geq D(M_i) = K^{-i}$ and 	$|J_i| = 1 \leq f(M_i) = K^{2K+1}K^{-2i}$ since $i\leq K$.
				\item Note $c(S_i) = |S_i|\cdot \frac{1}{f_iK^i} = \frac{1}{K^i} = D(M_i)$. Note $|S_i| = f_i = f(M_i)$.
			\end{itemize}
			\item For $1\leq i\leq K$, for a machine $N^{(i)}_j$ in class $i$, where $1\leq j\leq f_i$, we satisfy it as follows: $y(N^{(i)}_j, P^{(i)}_j) = 1-1/K$ and $y(N^{(i)}_j, t) = 1/f_i$ for all $t\in T_i$.
			Since $|T_i| = f_i/K$, the total fractional $y$-amount that $N^{(i)}_j$ gets is $1$. Also note that $N^{(i)}_j$ gets singleton jobs of type $i$ whose capacity is $\frac{1}{f_iK_i} = D(N^{(i)}_j)$.
		\end{enumerate}
		We need to show that no job is over allocated.
		\begin{enumerate}
			\item The big jobs $J_i$ is given $1/K$ to $M_0$ and $(1-1/K)$ to $M_i$.
			\item For $1\leq i\leq K$, $1\leq j\leq f_i$, job $P^{(i)}_j \in S_i$ is given $1/K$ to $M_i$ and $(1-1/K)$ to $N^{(i)}_j\in \M_i$.
			\item For $1\leq i\leq K$, job $t\in T_i$ is given $1/f_i$ to the $f_i$ machines of $\M_i$.
		\end{enumerate}
		This completes the description of the feasible solution.
	\end{proof}
	\begin{lemma}
		The integral optimum must violate some machine by factor $\Omega(K)$.
	\end{lemma}
	\begin{proof}
		Lets take machines in $\M_i$. Recall all machines here have demand of $\frac{1}{f_iK^i}$ and cardinality constraint of $1$.
		Thus in the integral optimum, they {\bf must} get one job which is either big, or of type $i$ or larger. 
		Now, the total number of jobs of type $j > i$ are 
		\[
		\sum_{j>i} f_j(1+1/K) = (K+1)K^{2K} \sum_{j > i} K^{-2j} \leq  (K+1)K^{2K}K^{-2i} \sum_{\ell=1}^\infty K^{-2\ell} = O\left(f_i/K\right)
		\]
		So, at least $(1 - \Theta(1/K))f_i$ of the machines in $\M_i$ get a job of type $i$ (or a big job but lets assume for now this don't happen -- can be ma).
		Therefore, the number of type $i$ jobs left after satisfying machines $(M_0,\ldots,M_K)$ are only $\Theta(f_i/K)$.
		
		
		Now take a machine $M_i$. We have $f(M_i) = f_i$ and $D(M_i) = 1/K^i$. 
		First note that jobs of type $j < i$ are ``useless'' for $M_i$. Any $f_i$ of them (best to take them of type $(i-1)$)  gives capacity $f_i\cdot c_{i-1}  = \frac{f_i}{f_{i-1}K^{i-1}} = \frac{1}{K^{i+1}} =  \frac{1}{K}\cdot D(M_i)$. So any subset of these jobs that can fit in $M_i$ gives capacity $\leq D(M_i)/K$.
		On the other hand, the total capacity of jobs remaining from type $j \geq i$ is  $\sum_{j\geq i} \Theta(f_j/K)\cdot \frac{1}{f_jK^j} = \Theta(1/K)\sum_{j\geq i} \frac{1}{K^j} = \Theta(D(M_i)/K)$.
		
		Therefore, any machine $M_i$ can't get more than $D(M_i)/K$ from the ``small'' jobs. But then they all can't get big jobs. 
	\end{proof}
	The above two lemmas prove the theorem after noting that $K = \Theta(\log n/\log\log n)$ where $n$ is either the number of machines of jobs.
\end{proof}

\input{section8}
%\section{Conclusion}

%\newpage

\section{Integrality Gap for Non-Uniform Santa Claus Problem}\label{sec:app-bsig}
We reproduce the integrality gap example for the configuration LP by Bansal and Sviridenko~\cite{BansalS06} for the general max-min allocation problem, and point out how their instance is in fact a $Q|restr|C_{min}$ instance.
Fix integer $K$. There are $K$ machines with demand $D_i = K$; these are the large machines $L = \{M_1,M_2,\ldots,M_K\}$. There are $K-1$ large jobs with $c_j = K$ which can only be assigned to the machines in $L$.
Let $J_B$ be the set of large jobs. There are $K^2$ small machines each with $D_i = 1$; these machines are distributed in $K$ classes where the $i$th class $\cC_i$ contains $K$ small machines. We let $m^{(i)}_k$ denote the $k$th machine in $\cC_i$, for $1\leq k\leq K$.
There are $K^2 + K$ small jobs with $c_j = 1$. These jobs are partitoned into $K$ classes with $i$th class $\cJ_i$ containing $K+1$ small jobs. Each class $\cJ_i$ has one ``public'' job $j^{(i)}_0$ which can be assigned to any machine $m^{(i)}_k \in \cC_i$
 and $K$ ``private'' jobs $j^{(i)}_k$, $1\leq k\leq K$ where $j^{(i)}_k$ can be assigned to only $m^{(i)}_k \in \cC_i$. Furthermore all the private jobs $j^{(i)}_k\in \cJ_i$ can be assigned to the large machine $M_i \in L$. This completes the description of the instance.
Note that the number of machines and jobs are $\Theta(K^2)$.

The integral optimum solution must give one machine $i$ capacity $\leq D_i/K$. Indeed, at least one large machine $M_i$ will not receive a job in $J_B$.
The only other jobs available to $M_i$ are the private jobs in $\calJ_i$. Suppose we allocate two such jobs to $M_i$; wlog these are $j^{(i)}_1$ and $j^{(i)}_2$. Now note
that the machines $m^{(i)}_1$ and $m^{(i)}_2$ have only job $j^{(i)}_0$ which can be assigned to them; and so one of them would get capacity $0$.
Therefore, the machine $M_i$ can receive only one job $j^{(i)}_k$ giving it total capacity $\leq D_i/K$.

On the other hand the configuration LP is feasible. Every large machine $M_i$ gets $z(M_i, j) = 1/K$ for all large jobs $j\in J_B$ and $z(M_i,\{j^{(i)}_1,\ldots, j^{(i)}_K\}) = 1/K$ for the set of private jobs in $\cJ_i$.
For all $1\leq i,k\leq K$, every machine $m^{(i)}_k$ receives $z(m^{(i)}_k,j^{(i)}_k) = 1-1/K$ and $z(m^{(i)}_k,j^{(i)}_0) = 1/K$. One can check all the jobs are fractionally assigned exactly.

\end{document}
