%\newpage 
\section{\mckc via Supply Polyhedra}\label{sec:o1}
\def\yy{y^\calT}
In this section, we prove the following theorem. One of the main engines will be a strong decomposition theorem (\Cref{thm:decomp}) which we will state here but will prove in the next section.
\begin{theorem}\label{thm:reduction}
Suppose there exists $\beta$-approximate, upward feasible supply polyhedra  for all instances of $Q|f_i|C_{min}$ (respectively, $Q||C_{min}$) which have $\gamma$-approximate separation oracles.
Then for any $\delta\in(0,1)$, there is an $\left(\tilde{O}(1/\delta),\gamma\beta(1+5\delta)\right)$-bicriteria approximation algorithm for the \mckc problem (respectively, with soft capacities).
\end{theorem}
\noindent
	
\noindent
\noindent
%
%As discussed in the Introduction, for the \mckc problem, the above LP has unbounded integrality gap (although, as we will see later, with soft capacities this is not the case).
%To obtain non-trivial algorithms we would need to strengthen the LP. Nevertheless, a feasible solution $(x,y)$ to LP~\eqref{eq:lp1}-\eqref{eq:lp6} gives us a way to decompose the problem 
%into ``easily roundable parts'' and ``\cckp parts". To formalize this, we make two definitions.
%
%\begin{definition}\label{def:rnding-mkc}
%	A set of facilities $S\subseteq F$ is said to be $(a,b)$-roundable w.r.t feasible solution $(x,y)$ if
%	\begin{itemize}[noitemsep]
%		\item[(a)] $\diam_G(S) \leq a$
%		\item[(b)] there exists a rounding $Y_{ip} \in \{0,1\}$ for all $i \in S, p\in [P]$ such that
%		\begin{enumerate}
%			\item $\sum_{q \geq p} \sum_{i\in S} Y_{iq} ~\leq~ \floor{\frac{1}{2} \cdot \sum_{q \geq p}\sum_{i\in S} y_{iq}}$ for all $p$, and
%			\item $\sum_{j\in C} d_j \sum_{i\in S,p\in [P]} x_{ijp} \leq b\cdot \sum_{i\in S} \sum_{p\in [P]} c_p Y_{ip}$
%		\end{enumerate}
%	\end{itemize}
%\end{definition}
%\noindent
%For any $(a,b)$-roundable set, we can integrally open facilities to satisfy all the demand that was fractionally assigned to it taking a hit of $a$ in the cost and a factor of $b$ in the capacities.
%Furthermore, the number of open facilities is at most what the LP prescribes. 
%%The idea behind the above definition should be clear.  
%%Suppose $S$ is a set of facilities and consider a subset of demands $J$ which are being assigned (fractionally) by
%%the solution $(x,y)$ to $S$. If $S$ is $(a,b)$-roundable wrt $(x,y)$, then we can open facilities in $S$ integrally and reassign the demands in $J$ to these integral
%%facilities. By doing so, we shall increase the connection cost of demands in $J$ by an additive factor of $a$, and then we may violate the capacities of the integrally open
%%facilities by at most a factor $b$. 
%\begin{definition} \label{def:comp-nbr}
%	A subset $S\subseteq F$ of facilities is called a {\em complete neighborhood} if there exists a client-set $J\subseteq C$ such that $\Gamma(J) \subseteq S$.
%	In this case the subset $J$ is said to be {\em responsible} for $S$. Additionally, a complete neighborhood $S$ is said to be an $a$-complete neighborhood if $\diam(S) \leq a$.
%\end{definition}
%\noindent
%%In a complete neighborhood set $S$ with $J$ responsible for it, we must open a total capacity of at least $|J|$ units among facilities in $S$. 
%%A collection of disjoint complete neighborhood sets therefore correspond to a \cckp instance. If the diameter of each set is $\leq \alpha$, then $\beta$-approximate solutions to the 
%%\cckp instance correspond to $(\alpha,\beta)$-bicriteria approximation.
%If we find a complete neighborhood $S$ of facilities with say a set $J$ of clients responsible for it, then we know that the optimal solution must satisfy all the demand in $J$ by suitably opening facilities of sufficient capacity in $S$. A disjoint collection of such complete neighborhoods leads to a \cckp problem instance. % described in Idea 1.
%\smallskip
%
%\noindent
The above theorem and results about supply polyhedra imply the bicriteria algorithms for the \mckc problem.
Theorem~\ref{thm:2} follows from the above theorem (instantialted with $\delta = 0.5$, say) and Theorem~\ref{thm:conflp} after noting that $D_\mathrm{max}/D_\mathsf{min} \leq n$ in the reduction we describe below. Theorem~\ref{thm:2a} follows
from the above theorem and Theorem~\ref{thm:asslp}. 

Before moving to the proof of Theorem~\ref{thm:reduction}, we state our main technical result which is a decomposition theorem which essentially states that given an \mckc instance, we can partition the problem into roundable and complete neighborhood sets. The reader may want to recall the definitions of roundable sets (Definition~\ref{def:rnding-mkc}), complete neighborhood sets (Definition~\ref{def:comp-nbr}), deletable sets (Definition~\ref{def:deletable}), and the natural LP relaxation \eqref{eq:lp1}-\eqref{eq:lp6}.
%Our main techincal theorem is the following 
It is perhaps instructive to compare the below theorem with Theorem~\ref{thm:weakdecomp}.
The proof of this theorem is rather technical, and we defer it to the next section.
\begin{theorem}[{\bf Decomposition Theorem}]\label{thm:decomp}
	Given a feasible solution $(x,y)$ to LP\eqref{eq:lp1}-\eqref{eq:lp6}, and $\delta > 0$, there is a polynomial time algorithm which finds a solution $\x$ satisfying \eqref{eq:lp2} and\eqref{eq:lp4}, and a 
	decomposition as follows.
	\begin{enumerate}%[noitemsep]
		\item The facility set $F$ is partitioned into two families $\calS = (S_1, S_2, \ldots, S_K)$ and $\calT = (T_1, T_2, \ldots, T_L)$ of mutually disjoint subsets.
		The client set $C$ is partitioned into three disjoint subsets $C = \Cd \cup \Cbb \cup \Cb$ where $\Cd$ is a $(\tilde{O}(1/\delta),\delta)$-deletable subset.
		
			\item Each $S_k \in \calS$ is $(\tilde{O}(1/\delta),(1+\delta))$-roundable with respect to $(\x,y)$, and moreover, each client in $\Cb$ satisfies $\sum_{i \in \calS, p} \x_{ijp} \geq 1 - \frac{\delta}{100}$.
\comment{deepc: maybe we should make 100 less arbitrary}
		\item Each $T_\ell$ is a $\tilde{O}(1/\delta)$-complete neighborhood with a corresponding set $J_\ell$ of clients responsible for it, and $\Cbb = \cup_{\ell = 1}^L J_\ell$.	
%		
%		
%		\item For the deleted clients $\Cd$, there is a mapping $\phi:\Cd \to \Cbb \cup \Cb$ such that
%		(a) $d(j,\phi(j)) \leq \tilde{O}(1/\delta)$ for all $j\in \Cd$, and
%		(b)	for all $j\in \Cbb \cup \Cb$, we have $\sum_{j' \in \Cd: \phi(j') = j} d_{j'} \leq (1+\delta)\cdot d_j$, i.e., the total demand mapped to $j$ is small.
%		
	\end{enumerate}
\end{theorem}


\begin{proof}[{\bf Proof of Theorem~\ref{thm:reduction}}]

Let us first describe an approach which fails. Let $(x,y)$ be a feasible solution to LP\eqref{eq:lp1}-\eqref{eq:lp6}, and apply~\Cref{thm:decomp}. 
Although the sets in $\calS$ by definition are roundable which takes care of the clients in $\Cb$, the issue arises in assigning clients of $\Cbb$.
In particular, $\yy_p := \sum_{i\in \calT} y_{ip}$ for all $1\le p\le P$ which indicates the ``supply" of capacity $c_p$ available for the $\Cbb$ clients.
%Since the LP doesn't know $\calT$ beforehand, this supply mayn't be enough. 
However, this may not be enough for serving all these clients (even with violation).
That is, the vector $\yy$ may not lie in the (approximate) supply polyhedra of the \cckp instance
defined by $\calT$ as described in Remark~\ref{rem:red}. 

That we fail is not surprising; after all, the LP has a bad integrality gap (Remark~\ref{rem:ig}) and we need to strengthen it.
We strengthen the LP by {\em explicitly requiring $\yy$ to be in the supply polyhedra}. Since we do not know $\calT$ before solving the LP (after all the LP generated it), 
we go ahead and require this for {\em all} collection of complete-neighborood sets. 
More precisely, for $\calT := (T_1,\ldots,T_L)$ of $L$ disjoint complete neighborhood sets, let $\calI_\calT$ denote the \cckp instance a la Remark~\ref{rem:red}.
\begin{equation}\label{eq:lp7}
\forall \calT := (T_1,\ldots,T_L) \textrm{ disjoint neighborhood subsets}, \quad \yy \in \calP(\calI_\calT) \tag{\small{L7}}
\end{equation}
Note that this is a feasible constraint to add to LP\eqref{eq:lp1}-\eqref{eq:lp6}. In the $\opt$ solution, for any $\calT$ there must be enough supply dedicated for the clients responsible for these complete neighborhood sets. So we have the following claim.
\begin{claim}\label{clm:trivial}
	If $\opt$ is feasible, then there is a feasible solution to LP\eqref{eq:lp1}-\eqref{eq:lp6} along with \eqref{eq:lp7}.
\end{claim}

\noindent
We don't know how (and don't expect) to check feasibility of  \eqref{eq:lp7} for all collections $\calT$. However, we can still run ellipsoid method using the ``round-and-cut'' framework of \cite{CarrFLP00,ChakrabartyCKK11,Li15,Li16}.
To begin with, we start with the LP\eqref{eq:lp1}-\eqref{eq:lp6} and obtain feasible solution $(x,y)$. Subsequently, we apply the decomposition Theorem~\ref{thm:decomp} to obtain the collection $\calT = (T_1,\ldots,T_L)$.
We then check if $\yy \in \calP(\calI_\calT)$ or not. Since we have a $\gamma$-approximate separation oracle for $\calP(\calI_\calT)$, we  are either guaranteed that $\yy \in \calP(\calI'_\calT)$ where the $\ell^{th}$ demand is now  $D_\ell/\gamma$; or we get a hyperplane separating $\yy$ from $\calP(\calI_\calT)$ which also gives us a
hyperplane separating $y$ from  LP\eqref{eq:lp1}-\eqref{eq:lp7}. This can be fed to the ellipsoid algorithm to obtain a new iterate $(x,y)$ and the above process is repeated. The analysis of the ellipsoid algorithm\comment{deepc: need to be careful and correct here}
tells us that in polynomial time we either prove infeasibility of the system \eqref{eq:lp1}-\eqref{eq:lp7} (implying the $\opt$  guess for \mckc is infeasible), or we 
are have $(x,y)$ satisfying the premise of the following lemma. 
%obtain a solution $(x,y)$ 
%which is feasible for LP\eqref{eq:lp1}-\eqref{eq:lp6}, such that for the $\calS, \calT$ obtained via Decomposition Theorem~\ref{thm:decomp}
%and the instance $\calI_\calT$ so obtained, the solution  $\yy \in \calP(\calI'_\calT)$ for the $\gamma$-shaded instance.

\begin{lemma}
	Given $(x,y)$ feasible for  LP\eqref{eq:lp1}-\eqref{eq:lp6}, let us apply the Decomposition Theorem~\ref{thm:decomp} to obtain the instance $\calS,\calT$.
	Suppose the solution $\yy_p := \sum_{i\in\calT} y_{ip}$ lies in $\calP(\calI'_\calT)$ for the \cckp (respectively, $Q||C_{min}$) instance $\calI'_\calT$ with $L$ machines with $D_\ell := |J_\ell|/\gamma$
	and $f_\ell := |T_\ell|$ (respectively, no cardinality constraints). Then we can obtain an $(\tilde{O}(1/\delta), \beta\gamma(1+\delta))$-approximate solution to the \mckc problem (respectively, with soft-capacities).
\end{lemma}
\begin{proof}
%Suppose the latter occurs. From this we can obtain the bicriteria solution to the \mckc problem as follows. 
%Recall the notation in the Decomposition Theorem~\ref{thm:decomp}. 
Since every set $S_k, 1\leq k\leq K,$ is $(\tilde{O}(1/\delta),(1+\delta))$-roundable, there exists a rounding $\y_{ip}$ for $i\in S_k$  such that
\begin{equation}\label{eq:repeat}
\textstyle \forall p, ~~~ \sum_{q\geq p} \sum_{i\in S_k} \y_{iq} \leq \floor{\sum_{q\geq p} \sum_{i\in S_k} y_{iq}} 
\end{equation}
\def\s{\tilde{s}}
Ideally, we would like to open a facility of capacity $c_p$ at location $i$ whenever $\y_{ip} = 1$. Unfortunately, the decomposition theorem doesn't have capacity constraints for individual $p$'s but only their suffix sums. Instead we do the following. Define $y^{\calS}_p := \sum_{i\in \calS} y_{ip}$; LP\eqref{eq:lp3} implies that for all $p$, $y^\calS_p + y^\calT_p \leq k_p$.
For $1\leq p\leq P$, define $s_p := \sum_{i\in \calS} \y_{ip}$; \eqref{eq:repeat} implies for all $p$, $\sum_{q\geq p} s_q \leq \floor{\sum_{q\geq p} y^\calS_q}$ (since $\floor{a} + \floor{b} \leq \floor{a+b}$.)
\begin{claim}\label{clm:massmovement}
	Given $(s_1,\ldots,s_P)$ satisfying $\sum_{q\geq p} s_q \leq \floor{\sum_{q\geq p} y^\calS_q}$, there exists $(\s_1,\ldots,\s_P)$ satisfying
	for all $p$, (a) $\sum_{q\geq p} s_q \leq \sum_{q\geq p} \s_q \leq \sum_{q\leq p} y^\calS_q$, and (b) $\s_p \leq k_p$.
\end{claim}
\begin{proof}
Simply define $\s_p := \floor{\sum_{q\geq p} y^\calS_q} - \floor{\sum_{q> p} y^\calS_q}$. Therefore, $\sum_{q\geq p} \s_q = \floor{\sum_{q\geq p} y^\calS_q}$ implying (a).
To see (b), note $\s_p \leq \ceil{y^\calS_p} \leq k_p$, where we use the fact $\floor{a+b} \leq \floor{a} + \ceil{b}$ for any non-negative $a,b$.
\end{proof}

The first inequality in (a) implies that at every location with $\y_{ip} = 1$, we can open a facility of capacity $c_q \geq c_p$. This, along with condition (b) of roundable sets (Definition~\ref{def:rnding-mkc}),
implies we can find a fractional solution $x_{ijp}$ for $j\in \Cb$ and $(i,p)$ with $\y_{ip} = 1$
such that (a) $\sum_{i\in \calS, p\in [P]} x_{ijp} \geq 1$, (b) $x_{ijp} > 0$ only if $d(i,j) \leq \diam(S_k) \leq \tilde{O}(1/\delta)$, and (c) the capacity violation is $\leq (1+\delta)(1 - \delta/100)^{-1} \leq (1+2\delta)$. Note the second term arises since
from the decomposition theorem we have $\sum_{i\in \calS, p\in [P]} \x_{ijp} \geq 1-\delta/100$.
Thus we have fractionally assigned all $\Cb$ clients to open facilities in $\calS$.
%For all $j\in \Cb$, we assign it fractionally using $\frac{\x}{1-\delta/100}$ to 


Define, for $p\in [P]$,  $t_p := k_p - \s_p$, the number of facilities of capacity $c_p$ we can open in $\calT$. Note, by Claim~\ref{clm:massmovement}, $t_p$'s are non-negative.
%
%
\begin{claim}
	$(t_1,\ldots,t_P) \in \calP(\calI'_\calT)$
\end{claim}
%	We now use the upward-feasibility property of $\calP$ to prove that the integral vector $(k^{(T)}_1,\ldots, k^{(T)}_P)$ also lies in $\calP(\calI'_\calT)$. 
\begin{proof}
	By the Lemma premise, we have $\yy \in \calP(\calI'_\calT)$. %Using the upward-feasibility property of $\calP$, we get the following.
	%Recall that $\calI'_\calT$ has $L$ machines with demands $D_\ell := |J_\ell|/\beta$ and $f_\ell := |T_\ell|$.
	Now note that for all $p$, 
	\[
\textstyle 	\sum_{q\geq p} t_q = \sum_{q\ge p} (k_q - \s_q) \ge \sum_{q\ge p} k_q - \sum_{q\geq p} y^\calS_q \geq  \sum_{q\geq p} y^\calT_q 
	\]
	Since $\calP(\calI'_\calT)$ is upward-feasible, and $\yy\in \calP$, we get the claim.
\end{proof}
Since $\calP(\calI'_\calT)$ is $\beta$-approximate, we can find an allocation of the $t_p$ copies of jobs of capacity $c_p$ to the $L$ machines of $\calI'_\calT$ such that machine $\ell$ gets at most $f_\ell$ jobs and total capacity $\geq D_\ell/\beta = |J_\ell|/\beta\gamma$. We install these capacities on the facilities of $T_\ell$. Since the diameter of each $T_\ell$ is $\tilde{O}(1/\delta)$, we can find an $x_{ijp}$ assignment of $\Cbb$-clients  to these such that $\sum_{i\in \calT,p\in [P]} x_{ijp} \geq 1$ and $x_{ijp} > 0$ iff $d(i,j) = \tilde{O}(1/\delta)$, such that the capacity violation is at most $\alpha\beta$. This takes care of the clients in $\Cbb$. Finally, Claim~\ref{clm:prelim3} takes care of all the deleted clients $\Cd$ with an extra hit of $(1+\delta)$ on the capacity and additive $\tilde{O}(1/\delta)$ on the distance.
\end{proof}
\noindent
This completes the proof of Theorem~\ref{thm:reduction} for the general \mckc problem. For the problem with soft capacities, the proof is exactly the same, except in the end, the instance $\calI_\calT$ is a $Q||C_{min}$ instance rather than a \cckp one.
\end{proof}

We end this section by noting that for the \mckc problem with soft-capacities, if we use the assignment supply polyhedra described in Section~\ref{sec:supplypolyhedra}, then we do not need to run the ellipsoid algorithm.
In particular, the inequality \eqref{eq:lp7} is implied \eqref{eq:lp1}-\eqref{eq:lp6} for $\calP_\mathsf{ass}$ defined in \eqref{eq:asslp1}-\eqref{eq:asslp3}.
\begin{lemma}\label{lem:implied}
Given any $(x,y)$ feasible for LP\eqref{eq:lp1}-\eqref{eq:lp6} and any $\calT = (T_1,\ldots,T_m)$, we have $\yy \in \calP_\mathsf{ass}(\calI_\calT)$.
\end{lemma}
\begin{proof}
	Fix $\calT = (T_1,\ldots,T_m)$ to be a collection of complete neighborhood sets. In the instance $\calI_\calT$ of $Q||C_{min}$, we have $m$ machines with demands
	$D_\ell = |J_\ell|$, where $J_\ell$ is the client set responsible for $T_\ell$. Recall, $y^\calT_p := \sum_{i \in \calT} y_{ip}$, and we need to find $z_{\ell,p}$ which satisfy the constraints
	\eqref{eq:asslp1}-\eqref{eq:asslp3} where $s_p := y^\calT_p$.
	
	The definition is natural: $z_{\ell,p} := \sum_{i\in T_\ell} y_{ip}$. Clearly it satisfies \eqref{eq:asslp1} (indeed with equality). We now show it satisfies \eqref{eq:asslp2}.
	To this end, define for any $j\in J_\ell$, $x_{jp} := \sum_{i\in T_\ell} x_{ijp}$. Since $\Gamma(J_\ell) \subseteq T_\ell$, we get from \eqref{eq:lp1} that $\sum_p x_{jp} \geq 1$.
	In particular, 
	\begin{equation}
\textstyle 	\label{eq:hohum1}\sum_p \sum_{j\in J_\ell}x_{jp} \geq D_\ell
	\end{equation} 
	
	From \eqref{eq:lp4}, we know $x_{ijp} \leq y_{ip}$ and summing over all $i\in T_\ell$, we get for all $j\in T_\ell$, $x_{jp} \leq \sum_{i\in T_\ell} y_{ip} = z_{\ell,p}$.
	In particular, $\sum_{j\in J_\ell} x_{jp} \leq z_{\ell,p} D_\ell$. From \eqref{eq:lp2} we know for all $i\in T_\ell, p\in [P]$, $\sum_{j\in J_\ell} x_{ijp} \leq c_p y_{ip}$.
	Summing over all $i\in T_\ell$, gives $\sum_{j\in J_\ell} x_{jp} \leq c_p z_{\ell,p}$. Putting together, we get
	\begin{equation}
	\textstyle 	\label{eq:hohum2}\sum_{j\in J_\ell}x_{jp} \leq z_{\ell,p} \min(D_\ell,c_p)
	\end{equation} 
	\eqref{eq:hohum1} and \eqref{eq:hohum2} imply that $z$ satisfies \eqref{eq:asslp2}.
\end{proof}
Therefore, one can use the natural LP relaxation to obtain for any $\delta > 0$, a $\left(\tilde{O}(1/\delta),(2+\delta)\right)$-bicriteria approximation for the \mckc problem with soft capacities.
As it should be clear, this is a much more efficient algorithm.
