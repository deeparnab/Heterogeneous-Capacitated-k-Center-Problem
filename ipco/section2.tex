%\newpage
\section{Reduction to Max-Min Allocation via Region Growing}\label{fsec:regiongrowing}
In this section, we give a reduction to \cckp  when we allow logarithmic approximations. We then show how we get Theorem~\ref{fthm:1} using this result.
\begin{theorem}\label{fthm:weakred}
	Given an $\beta$-approximation algorithm for \cckp (respectively, $Q||C_{min}$), for any $\epsilon>0$ there exists an $\left(O(\log n/\epsilon), \beta(1+\epsilon)\right)$-approximate algorithm for the \mckc problem (respectively, for the \mckc problem with soft capacities).
\end{theorem}

The main crux of the above proof is the following decomposition theorem obtained by the technique of region growing which was first used in the context of sparsest and multi cut problems~\cite{LeightonR99,GargVY96}.

\begin{theorem}\label{fthm:weakdecomp}
	Given a guess $\opt$ for \mckc problem and any $\epsilon>0$,  there is an algorithm which partitions the facilities $F$ into a collection $\calT = (T_1,\ldots,T_L)$ of
	$O(\log n/\epsilon)$-complete neighborhoobd sets with $J_\ell$ responsible for $T_\ell$, and the client set $C = \Cd \cup \bigcup_{\ell=1}^L J_\ell$ such that
	$\Cd$ is an $(O(\log n/\epsilon),\epsilon)$-deletable set.
\end{theorem}
\begin{proof}
Recall $G$ is the graph with $d(i,j) \leq \opt$ for $(i,j)\in G$. %	We start with a guess of $\opt$ and the $\opt$-induced graph $G$. We assume this is a correct guess.
%	Given $G$ we given an algorithm construct an instance $\calI$ of $Q|k_i|C_{min}$.
	Initially $\calT$ and $\Cd$ are empty. We maintain a set of alive clients $C'$ which is initially $C$. We maintain a working graph $H$ which is initialized to $G$ and is always a subgraph of $G$.
	Given a node $j$ and an integer $t$, let $N^{(t)}_H(j)$ denote all the modes $j'$ s.t. $d_H(j,j') < t$ and $\Gamma^{(t)}_H(j)$ denote all the nodes $j'$ with $d_H(j,j')  = t$.
	Note that for even $t$, we have $\Gamma^{(t)}_H(j) \subseteq C$, and for odd $t$, $\Gamma^{(t)}_H(j) \subseteq F$.
	\smallskip
	
	
	Till $C'$ is empty, we perform the following operation.
	Select an arbitrary active client $j\in C'$. Find the smallest even $t$ such that $|\Gamma^{(t)}_H(j)| < \eps\cdot |N^{(t)}_H(j)\cap C|$. Since for all $s < t$ we have $|N^{(s+2)}(j)\cap C|\geq (1+\epsilon)|N^{(s)}_H(j)\cap C|$,  $|N^{(s+2)}(j)\cap C| > (1+\epsilon)^{\frac{s}{2}}$. Therefore,  $t \leq (2\ln n)/\epsilon,$ where $n=|C'|$. %Let us use $A_j$ to denote $N_t(j)$ and $B_j$ to denote $\Gamma_{t}(j)$.
	We define $T_\ell := N^{(t)}_H(j) \cap F$ and $J_\ell := N^{(t)}_H \cap C$; note that $T_\ell$ is an $O(\log n/\epsilon)$-complete neighborhood which is responsible for $J_\ell$. Furthermore, we add $J_\mathsf{ext} := \Gamma^{(t)}_H(j)$ to $\Cd$, and since
	$|J_\mathsf{ext}| < \eps |J_\ell|$ and $\diam(J_\ell) = O(\log n/\epsilon)$, there exists a mapping $\phi_{j,j'}$ for $j\in J_\mathsf{ext}$ and $j'\in J_\ell$ such that $\sum_{j'\in J_\ell} \phi_{j,j'} = 1$ for all $j\in J_\mathsf{ext}$, and
	$\sum_{j\in J_\mathsf{ext}} \phi_{j,j'} \leq \epsilon$ for all $j\in J_\ell$, and $\phi_{j,j'} > 0$ only if $d(j,j') = O(\log n/\epsilon)$. That is, $J_\mathsf{ext}$ is a valid $(O(\log n/\epsilon),\epsilon)$-deletable set.
	Finally, we delete $T_\ell \cup J_\ell \cup J_\mathsf{ext}$ from $H$ and $J_\ell \cup J_\mathsf{ext}$ from $C'$. We continue this procedure till $C'$ is empty. \end{proof}
\begin{proof}[Proof of Theorem~\ref{fthm:weakred}]
	Given $\calT$ we form the instance $\calI$  of \cckp (or $Q||C_{min}$ in case of soft-capacities) described in Remark~\ref{frem:red}. We provide $k_p$ copies of job with capacity $c_p$. If $\opt$ is feasible, then there must exist a feasible solution to $\calI$. Furthermore, a $\beta$-approximate solution to $\calI$ gives an $\left(O(\log n/\epsilon),\beta\right)$-approximate solution to the clients in $C\setminus \Cd$. The theorem follows from Claim~\ref{fclm:prelim3}.
%	
%	\begin{claim}
%		There exists an allocation where the total processing time of each machine is at least $1$.
%	\end{claim}
%	\begin{proof}
%		In the optimal solution to \mckc, a total capacity of $|A_j|$ must be installed among the facilities $F_j$.
%		We mimic this solution for $\calI$, and since the speed of machine $j$ is $1/|A_j|$, the claim follows.
%	\end{proof}
%	\begin{claim}
%		Given an allocation to $\calI$ where each machine gets processing time $\geq \alpha$, we can construct a solution for the \mckc problem
%		which is $O(\log n/\epsilon)$-approximate and violates the capacities by at most $(1+\epsilon)/\alpha$-factor.
%	\end{claim}
%	\begin{proof}
%		For machine $j$ in $\calI$, let $S_j$ be the jobs allocated; we have $\sum_{p\in S_j} c_p \ge \alpha\cdot |A_j|$ and $|S_j| \leq k_j = |F_j|$.
%		We arbitrarily open $|S_j|$ facilities in $F_j$ to which we assign all the clients in $A_j \cup B_j$. Since $|B_j|<\eps |A_j|$, we have the total capacity opened is at least $\alpha/(1+\epsilon)$ times
%		the total number of clients in $A_j\cup B_j$. Furthermore, every client in $A_j\cup B_j$ is at most $O(\log n/\epsilon)$-away from any facility in $F_j$. This implies the assignment.
%	\end{proof}	
\end{proof}

As a corollary to Theorem~\ref{fthm:weakred}, and using the fact that $Q||C_{min}$ has a PTAS ~\cite{AzarE98},  and our result (Theorem~\ref{fthm:q} proved in~\Cref{fsec:qptas}) that \cckp has a quasipolynomial time approximation scheme (QPTAS), we get Theorem~\ref{fthm:1}.\medskip


In Section~\ref{fsec:o1} we state a much stronger decomposition theorem than Theorem~\ref{fthm:weakdecomp} which exploits the LP solution.
To exploit it for \mckc problem, however, and prove an analogous theorem as Theorem~\ref{fthm:weakred}, we need to understand certain polyhedra with respect to the \cckp problem.
We first do this in the next section.
