%\newpage
\section{Supply Polyhedra of $Q||C_{min}$: Proof of Theorem~\ref{thm:asslp}}\label{sec:asslp}
\def\pv{\mathbf{b}}
\newcommand{\dem}{\mathsf{cap}}
  Throughout the proof we fix $\calI$ to be the instance of $Q||C_{min}$ % and we rename machines and jobs so that $D_1 \geq \cdots \geq D_m$ and $n$ types of  jobs $J$ with capacities $c_1 \geq \cdots \geq c_n$.  
  %We also fix 
  and the supply vector $(s_1,\ldots,s_n)$. For simplicity of presentation, given the supply vector, abusing notation let $J$ denote the multiset of jobs where job $j$ appears $s_j$ times. We know that the LP\eqref{eq:asslp1}-\eqref{eq:asslp3} is feasible with the $s_j$ replaced by $1$. %Let $N = \sum_j s_j$. 
  We want to find an assignment where machine $i$ gets at least $D_i/2$ capacity. 
%Let  $z$ be a feasible solution to \eqref{eq:asslp1}-\eqref{eq:asslp3}. 
%Let the instance $\calI$ of $Q||C_{min}$  have $m$ machines $M$ with demands 
%A supply vector $(s_1,\ldots,s_n)$ indicates the number of jobs of each type available; a supply vector is feasible if together they can satisfy all the demands.
%We wish to find a convex set/polyhedra which captures all the feasible supply vectors. In particular, any feasible supply vector should be in the set, and given any (integer) supply vector in the set
%there should be an allocation which satisfies the demands to an $\alpha$-factor.
% \smallskip
%
%\noindent
%A feasible supply vector $(s_1,\ldots,s_n)$ must lie in the following polytope.
%%We assume we have a guess $D$ for the optimum value which is certified by a feasible solution to the following assignment LP. Below, $D_i := Ds_i$.
%	\begin{alignat}{4}
%		\calP_\mathsf{ass} && = \{(s_1,\ldots,s_n):  && \notag \\
%		&& \quad \forall j \in J,   &\quad  \textstyle \sum_{i\in M} z_{ij}  \leq  s_j \label{eq:asslp1} \\
%		&& \quad \forall i\in M ,  &\quad  \textstyle \sum_{j\in C}  z_{ij}  \min(c_j,D_i) \geq D_i \label{eq:asslp2} \\
%		&& \quad \forall i\in M, j\in J, & \quad z_{ij}   \geq 0 \label{eq:asslp3}  \}
%	\end{alignat}
%Not all integral $(s_1,\ldots,s_n) \in \calP_\mathsf{ass}$ need be feasible; but the following theorem shows given such a supply vector, there exists an assignment satisfying the demands up to a factor $2$.
%%The following theorem shows that $\calP_\mathsf{ass}$ captures the 
%%the assignment LP has integrality $\leq 2$.
%\begin{theorem}\label{thm:asslprounding}
%Given $(s_1,\ldots,s_n) \in \calP_\mathsf{ass}$, there is an of assignment $\phi$ of the $s_j$ jobs of capacity $c_j$  to the machines such that for all $i\in M$, 
%$\sum_{j:\phi(j) = i} c_j \geq D_i/2$.
%\end{theorem}
%\begin{proof}
%For simplicity, given the supply vector, abusing notation let $J$ denote the multiset of jobs where job $j$ appears $s_j$ times. We know that the LP\eqref{eq:asslp1}-\eqref{eq:asslp3} is feasible with the $s_j$ replaced by $1$. Let $N = \sum_j s_j$. 
%

The algorithm is a  very simple greedy algorithm which doesn't look at the LP solution , and the feasibility of LP\eqref{eq:asslp1}-\eqref{eq:asslp3} is only used for analysis.  
Rename the jobs (with multiplicities) in decreasing order of capacities $c_1\geq c_2 \geq \cdots \geq c_N$, and rename  the machines in decreasing order of $D_i$'s, that is, $D_1 \geq D_2 \geq \ldots \geq D_m$. 
Starting with machine $i=1$ and job $j=1$, assign jobs $j$ to $i$ if the total capacity filled in machine $i$ is $< D_i/2$ and move to the next job. Otherwise, call machine $i$ happy and move to the next machine. Obviously, if all machines are happy at the end we have found our assignment. 

The non-trivial part is to  prove that if some machine is unhappy, then the LP\eqref{eq:asslp1}-\eqref{eq:asslp3} is infeasible (with $s_j$ replaced by $1$).
To do so, we take the Farkas dual of the LP; the following LP is feasible iff LP\eqref{eq:asslp1}-\eqref{eq:asslp3} is infeasible. We describe a feasible solution to the system below if we obtain some unhappy agent.
	\begin{alignat}{4}
		&&   & \quad \textstyle \sum_{i=1}^m \beta_i D_i > \sum_{j=1}^n\alpha_j \label{eq:assdual1}  \tag{F1} \\
		&& \quad \forall i\in M,j\in J & \quad \textstyle \beta_i\min(c_j,D_i) \leq \alpha_j \label{eq:assdual2} \tag{F2}  \\
		&& \quad \forall i\in M, &\quad  \beta_i \geq 0\label{eq:assdual3}\tag{F3}
	\end{alignat}
	\def\i{i^\star}
Suppose machine $\i$ is the first machine which is unhappy. Let $S_1,\ldots,S_{\i-1}$ be the jobs assigned to machines $1$ to $(\i-1)$ and $S_{i^\star}$ be the remainder of jobs. 
We have $\sum_{j\in S_{\i}} c_j < D_{\i}/2$. We also have for all $1\leq i\leq \i$, $\sum_{j\in S_i} \min(c_j,D_i) \leq D_i$. If not, then the machine must receive at least two jobs and would have capacity $> D_i/2$ 
from all but the last. 
We now describe a feasible solution to \eqref{eq:assdual1}-\eqref{eq:assdual3}.
%{\bf deepc: this turns out to be trickier than what meets the eye}.

Given the assignment $S_i$'s, call a machine $i$ {\em overloaded} if $S_i$ contains a single jobs $j_i$ with $c_{j_i} \geq D_i$. 
We let $\beta_1 = 1$. For $1\leq i <\i$, we have the following three-pronged rule
\begin{itemize}[noitemsep]
	\item If $i+1$ is not overloaded, $\beta_{i+1} = \beta_i$.
	\item If $i+1$ is overloaded, and so is $i$, then $\beta_{i+1} = \beta_i \cdot D_i/D_{i+1}$.
	\item If $i+1$ is overloaded but $i$ is not, then $\beta_{i+1} = \beta_i \cdot c_{j_{i+1}}/D_{i+1}$, where $j_{i+1}$ is the job assigned to $i+1$.
\end{itemize}
For any job $j$ assigned to machine $i$, we set $\alpha_j = \beta_i \min(c_j,D_i)$. Since for any $S_i$, we have $\sum_{j\in S_i} \min(c_j,D_i) \leq D_i$ and $\sum_{j\in S_{\i}} c_j < D_{\i}/2$,  the given $(\alpha,\beta)$ solution satisfies \eqref{eq:assdual1}. We now prove that it satisfies \eqref{eq:assdual2}.
From the construction of the $\beta$'s the following claims follow.
\begin{claim}\label{clm:c1}
$\beta_1\leq \beta_2 \leq \cdots \leq \beta_m$.
\end{claim}
\begin{claim}\label{clm:c2}
$\beta_1D_1 \geq \beta_2D_2 \geq \cdots \geq \beta_mD_m$.
\end{claim}
\begin{proof}
	The only non-obvious case is if $i+1$ is overloaded but $i$ is not: in this case $\beta_{i+1}D_{i+1} = \beta_ic_{j_{i+1}}$. But since $i$ is not overloaded, let $j$ be some job assigned to $i$ with $c_j \leq D_i$.
	By the greedy rule, $c_j \geq c_{j_{i+1}}$, and so $\beta_{i+1}D_{i+1} \leq \beta_iD_i$.
\end{proof}
\noindent
Now fix a job $j$ and let $i$ be the machine it is assigned to. Note \eqref{eq:assdual2} holds for $(i,j)$ and we need to show \eqref{eq:assdual2} holds for all $(i',j)$ too.
I don't see any more glamorous way than case analysis. \smallskip

\noindent
{\bf Case 1: $c_j \leq D_i$.} In this case $\alpha_j = \beta_ic_j$ and $i$ is not overloaded. 
Let $i' < i$.  Then we have $\beta_{i'}\min(c_j,D_{i'}) \le \beta_{i'}c_j \leq \beta_ic_j$, where the last inequality follows from Claim~\ref{clm:c1}.

Now let $i' > i$. If $c_j \leq D_{i'}$, then none of the machines from $i$ to $i'$ can be overloaded. Therefore, $\beta_{i'} = \beta_i$, and so $\beta_{i'}c_j = \beta_ic_j = \alpha_j$.
So, we may assume $c_j > D_{i'}$ and we need to upper bound $\beta_{i'}D_{i'}$. Let $i'' > i$ be the first machine which is overloaded with job $j''$ say.
By Claim~\ref{clm:c2}, we have $\beta_{i'}D_{i'} \leq \beta_{i''}D_{i''}$. Now note that
$\beta_{i''}D_{i''}  = \beta_{i''-1}c_{j''} = \beta_ic_{j''} \leq \beta_ic_j = \alpha_j$ where the second equality follows since none of the machines from $i$ to $i''-1$ were overloaded. \smallskip

\noindent
{\bf Case 2: $c_j > D_i$.} In this case $\alpha_j = \beta_iD_i$ and $i$ is overloaded. Let $i' > i$. Then, $\beta_{i'}\min(c_j,D_{i'}) = \beta_{i'}D_{i'} \leq \beta_iD_i$ where the last inequality follows from Claim~\ref{clm:c2}.


Let $i' < i$. Let $i'\leq i'' < i$ be the smallest entry such that $c_j > D_{i''}$. Note that all machines from $i''$ to $i$ must be overloaded implying $\beta_{i''}D_{i''} = \beta_iD_i$.
Since $c_j \leq D_{i'}$ (in case $i' < i''$), we need to upper bound $\beta_{i'}c_j$. 
By Claim~\ref{clm:c1}, $\beta_{i'}c_j \leq \beta_{i''-1}c_j$. Now, if $i''-1$ were overloaded, 
then $\beta_{i''}D_{i''} = \beta_{i''-1}D_{i''-1} \geq \beta_{i''-1}c_j$ where the last inequality follows from definition of $i''$. Together, we get $\beta_{i'}c_j \leq \beta_iD_i$.
%For every machine $1\leq i\leq \i$, define $\beta_i := D_\i/D_i$. For every job $j\in S_i$ for $1\leq i\leq \i-1$, define $\alpha_j = \beta_i\min(c_j,D_i)$ and for $j\in S_\i$, define $\alpha_j = c_j$.
%\end{proof}
\def\y{\bar{y}}
\def\z{\bar{z}}
\def\yy{\bar{\bar{y}}}
\begin{lemma}\label{lem:conf-is-uf}
	$\calP_\mathsf{ass}$ is upward-feasible.
\end{lemma}
\begin{proof}
	%\comment{\bf \Large Needs to be written}
	Let $s := (s_1,\ldots,s_n)\in \calP_\mathsf{ass}$ for a certain instance of $Q||C_{min}$ where the jobs have been renamed so that $c_1\leq \cdots \leq c_n$. We need to prove any non-negative vector $t := (t_1,\ldots,t_n)$ s.t. $t\succeq_\suff s$also lies in $\calP_\mathsf{ass}$. 
	By the ``hybridization argument'', it suffices to prove the lemma for  $s$ and $t$ differing only in coordinates $\{j-1,j\}$ and $ t_j\ge s_j$ and $t_{j-1} \geq \max(0,s_{j-1} + (s_j - t_j))$.
	Given that, we can move from $s$ to $t$ by changing pairs of coordinates each time maintaining feasibility in $\calP_\mathsf{ass}$.
	
	Let $z$ be the solution for the supply vector $s$; we construct a solution $\z$  for the supply vector $t$ starting with $\z = z$.
	If $\z$ is not already feasible, then it must be because $s_{j-1} \geq \sum_{i\in M} \z_{i,j-1} > t_{j-1}$.
	We select an arbitrary $i\in M$ with $\z_{i,j-1} > 0$ and increase $\z_{ij}$ and decrease $\z_{i,j-1}$ by $\delta$. Since $c_j \geq c_{j-1}$, \eqref{eq:asslp2} remains valid.
	Since the total increase of fractional load of job $j$ is exactly the same as the decrease in that of job $j-1$, and we only need total  decrease $(s_{j-1} -t_{j-1}) \leq t_j - s_j$, at the end we get that $\z$ is feasible wrt supply vector $t$.
	
\end{proof}
%\begin{lemma}
%Suppose $(y_1,\ldots,y_n)\in \calP_\mathsf{ass}$. Let $(\y_1,\ldots,\y_n)$ be a vector such that for all $1\leq i\leq n$, $\sum_{j\leq i} \y_j \geq \sum_{j\leq i} y_i$. Then
%$(\y_1,\ldots,\y_n)\in \calP_{\mathsf{ass}}$.
%\end{lemma}
%\begin{proof}
%By induction, let us assume the lemma is true for all $\y$ with $\y_1 = y_1$ which satisfies the prefix-sum condition.
%Let $\yy$ be the vector with $\yy_1 = y_1$, $\yy_2 = \y_2 + \y_1 - y_1$, and $\yy_i = \y_i$ otherwise.
%Since $\yy\in \calP_\mathsf{ass}$, there is an assignment $z_{ij}$ satisfying \eqref{eq:asslp1}-\eqref{eq:asslp3} with $s_j = \yy_j$.
%We now describe a feasible solution $\z_{ij}$ with $s_j = \y_j$.
%
%Let $\theta := \y_2/\yy_2 \le 1$ since $\y_1 \geq y_1$. Define $\z_{i2} = \theta z_{i2}$ for all $i$, and define $\z_{i1} = z_{i1} + (1-\theta)z_{i2}$.
%For $j=2$, we have $\sum_{i\in M} \z_{i2} = \theta \sum_{i\in M} z_{i2} \leq \theta \yy_2 = \y_2$.
%For $j=1$, we have $\sum_{i\in M} \z_{i1} = \sum_{i\in M} z_{i1} + (1-\theta) \sum_{i\in M} z_{i2} \leq \yy_1 + (1-\theta)\yy_2 = \yy_1 +\yy_2 - \y_2 = \y_1$. 
%Since the other $z_{ij}$'s and $\y_j$'s are untouched, $\z_{ij}$ satisfies \eqref{eq:asslp1} with $\y_j$'s.
%
%Now fix a machine $i$. The `increase' in the LHS of \eqref{eq:asslp2} is  $\sum_{j\in J} (\z_{ij} - z_{ij}) \min(c_j,D_i) = (1-\theta)z_{i2}\min(c_1,D_i) -  (1-\theta) z_{i2}\min(c_2,D_i) \geq 0$ since $c_1 \geq c_2$.
%
%\end{proof}
%
