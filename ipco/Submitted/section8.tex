%\newpage
\section{QPTAS for \cckp: Proof of Theorem~\ref{fthm:q}} \label{fsec:qptas}
\def\pv{\mathbf{b}}
\renewcommand{\dem}{\mathsf{cap}}
Let the instance $\calI$ given to us have $m$ machines with demands $D_1,\ldots,D_m$ and cardinality constraints $f_1,\ldots,f_m$, and $n$ jobs with capacities
$c_1,\ldots,c_n$ (with $1$ copy of each job by taking possible duplicates). Let $D:= D_\mathrm{max}/D_\mathrm{min}$, which we assume to be $\leq \poly(n)$.
Fix $\epsilon> 0$. Our goal is to either prove there is no feasible solution, or find an assignment
giving each machine $i$ a capacity of $\geq D_i(1-O(\epsilon))$.
We start with a lemma which states that finding solutions satisfying cardinality constraints approximately suffices.
\begin{lemma}\label{flem:appx-feas}
	Given an assignment of jobs such that the load on any machine $i$  is at least $D_i(1 - \epsilon_1)$ such that machine $i$ gets $\leq (1+\epsilon_2)f_i$ jobs,
	we can find another assignment which satisfies the cardinality constraints and the load of any machine $i$ is $\geq D_i(1 - \epsilon_3)$ for $\epsilon_3 < 2\epsilon_1+\epsilon_2$.
\end{lemma}
\begin{proof}
For every machine $i$, let $S_i$ be the jobs currently allocated to it. We may assume $\eps_1 f_i \geq 1$, otherwise $|S_i| \leq f_i$.
Remove the $\floor{2\eps_1 f_i} \geq \eps_1 f_i$ least capacity jobs to obtain the set $S'_i$. Note that the total capacity of $S'_i$ is at least $(1-2\eps_1)$ times capacity of $S_i$, and therefore
at least $(1-2\epsilon_1 - \epsilon_2)D_i$.
\end{proof}
\noindent
\paragraph{Input Modification and Grouping.}
We now modify the data so that everything is rounded to the nearest power of $(1+\epsilon)$. More precisely we round  $f_i$  to the {\em smallest} power of $(1+\epsilon)$ larger than the original value and $D_i$ to the largest power of $(1+\epsilon)$ smaller than the original value.
If the original instance had a feasible solution, then so does the modified instance.
For technical reasons, we round $c_p$ to the smallest value of the form $\epsilon(1+\epsilon)^t$ larger than the original value.
Let $J_p$ be the set of jobs with modified capacity $c_p = \eps (1+\epsilon)^p$, and let $n_p = |J_p|$. Furthermore, armed with Lemma~\ref{flem:appx-feas}, any $(1-\epsilon)$-approximate solution to the new instance gives an $(1-O(\epsilon))$-approximate solution to the original instance.  \smallskip


We now divide the machines into groups. For $0\leq r\leq O(\log n)$ and $0\leq s\leq O(\log n)$, let $M^{(r,s)}$ be the number of machines with $D_i = (1+\epsilon)^r$ and $f_i = (1+\epsilon)^s$.
%\[
%M^{(r,s)} := \{i: (1+\epsilon)^{r-1} \le D_i  \le (1+\epsilon)^r, ~~\textrm{and}~~ (1+\epsilon)^{s-1} \le f_i  \le (1+\epsilon)^s  \}
%\]
%Throughout the paper we fix $\delta,\eps > 0$. As is standard, we assume we have a guess $T$ of the optimum; given such a guess we would either prove $\opt > T$ or obtain a feasible schedule of makespan at most $(1+\eps)T$.
%Subsequently, binary search would provide a schedule of makespan $\leq \opt(1+\eps)$. \smallskip
%
\def\sm{\mathsf{small}}
Call a job $p$  big  for machine $i$ if $c_p \geq \eps D_i$. If $i\in M^{(r,s)}$, then $p$ lies in the set $J_r \cup J_{r+1} \cup \cdots$.
Otherwise, $p$ is small for machine $i$.
We define a bipartite graph $H$ with jobs and machines on either side, with an edge $(i,p)$ iff $p$ is small for $i$.

For every $0\leq r,s \leq O(\log n)$, we define a set of {\em feasible configurations} $\Phi^{(r,s)}$. These consist of vectors $\phi \in \Z^K_{\geq 0}$ for $K=O(1/\epsilon)$ corresponding to big jobs assigned to machines $i$ in $M^{(r,s)}$.
To be precise, $\phi_k$ is supposed to count the number of jobs with $c_p = \eps (1+\epsilon)^{r+k}$ contained in the configuration $\phi$.
The last coordinate $\phi_K$ counts the number of jobs $p$ with $c_p > (1+\epsilon)^r$.
Let
$\dem(\phi) := \sum_{k=0}^K \phi_k \epsilon(1+\epsilon)^{r+k}$ be the total load of the configuration and $|\phi| = \sum_{k=0}^K \phi_k$ be its cardinality.
We let $\Phi^{(r,s)}$ be the collection of feasible minimal configurations, that is, $\phi$'s with (a) $|\phi|\leq (1+\epsilon)^s$ and (b) either $\dem(\phi)\leq (1+\epsilon)^r$ or $\dem(\phi) > (1+\epsilon)^r$ and $\dem(\phi') \leq (1+\epsilon)^r$ for any $\phi'$ obtained by decreasing any positive coordinate of $\phi$ by exactly $1$. Note that $|\Phi^{(r,s)}| \leq N_0 = (1/\epsilon)^{(1/\epsilon)}$. Also note that in any optimal solution, each machine $i\in M^{(r,s)}$ does get one configuration from $\Phi^{(r,s)}$.
Our algorithm constructs these classes and arbitrary numbers them. The $t$th member of $\Phi^{(r,s)}$ is denoted as $\phi^{(r,s,t)}$.% where $r,s$ would be clear from context.

%if $p_j > \eps T$, and small otherwise. Let $J_\sm$ 	be the set of small jobs.
%For any $1\leq k\leq K = O(1/\eps)$, let $J_k$ denote the big jobs with capacities $p_j \in \eps T\cdot \left((1+\eps)^{k-1},(1+\eps)^k\right]$.
%A configuration $\phi \in \Z^K_{\ge 0}$ is a vector where $\phi_k$ indicates the number of jobs of chosen from $J_k$. The total load $\dem(\phi)$ of a configuration is defined to be $\sum_{k=1}^K \phi_k (1+\eps)^{k-1}\eps T$ -- note that this is a {\em lower bound} on the actual load but is within a multiplicative $(1+\eps)$-factor. Finally, we let $|\phi| = \sum_{k=1}^K \phi_k$, the cardinality of the configuration. A configuration $\phi$ is {\em allowed} if $\phi_k \leq 1/\epsilon$. The number of allowed configurations is denoted as $N_0 \leq (1/\epsilon)^{(1/\epsilon)}$.
%We call this set $\Phi$ which we order in an arbitrary but fixed manner. The $i$th member is denoted as $\phi^{(i)}$.
%
%\smallskip
%Call a machine $i$ large if $f_i > 1/\eps^2$. Let $M^{(0)}$ denote the set of large machines, and let $m_0 = |M^{(0)}|$.
%Every other machine is small and these are partitioned into $N_1 = O(1/\eps \ln(1/\eps))$ groups as follows. For $1\leq \ell \leq N_1$, define
%$M^{(\ell)} := \{i: (1+\eps)^{\ell-1} < f_i \leq (1+\epsilon)^\ell\}$. Let $m_\ell := |M^{(\ell)}|$.
%
%\subsection{MILP Formulation}
\paragraph{Enumeration.}
For every $0\leq r,s \leq O(\log n)$ and $1\le t\le N_0$, we {\em guess}  the  integer $\pv^{(r,s)}_t \in \Z_{\geq 0}$ which indicates the number of machines in $M^{(r,s)}$ who are allocated the configuration $\phi^{(t)}$.
%That is, the number of big jobs from $J_k$, for $1\le k\le K$ assigned to these machines is $\phi^{(t)}_k$. Note that the number of variables is at most $N_0 N_1 = 2^{O(1/\eps \ln(1/\epsilon))}$.
%The number of such integer variables is $d := O(\log^2 n)2^{\tilde{O}(1/\epsilon)}$. For every machine $i$ and every job $p$ small for $i$, we have a variable $z_{ip} \in [0,1]$. This indicates whether small job $p$ is allocated to machine $i$.
These guesses must satisfy
%The first constraint states that the total number of machines need to be respected.
\begin{equation}\label{feq:milp1}
\forall 0\leq r,s \leq O(\log n), ~~~~ \sum_{t=1}^{N_0} \pv^{(r,s)}_t = |M^{(r,s)}|
\end{equation}
The number of such guesses is $\leq \prod_{r,s} |M^{(r,s)}|^{N_0} \leq C_\eps^{O(\log^3 n)}$ for some constant $C_\epsilon$ which is double-exponential in $(1/\epsilon)$.
Since machines in $M^{(r,s)}$ are all equivalent (in terms of demand and cardinality constraint), by symmetry we can assign the $\pv^{(r,s)}_t$ copies of $\phi^{(r,s,t)}$ as we like.
For a guess to be feasible, for every job of type $p$, at most $n_p$ copies must be used up in the guessed configurations.
For every guess we get a residual problem on the bipartite graph $H$.
Let $n'_p$ be the remaining number of jobs of type $p$.
Let $D'_i$ be the residual demand of machine $i$, that is, $D_i - \dem(\phi)$ where $\phi$ is allocated to it by the guess.
Let $f'_i$ be the residual cardinality constraint, that is, $f'_i = f_i - |\phi|$.
\paragraph{Rounding.}
The remaining copies of jobs must satisfy the residual demand. For this we simply write the assignment LP\eqref{feq:asslp1}-\eqref{feq:asslp4} which we rewrite below.
\begin{alignat}{4}
	& \quad \forall p,   &&\quad  \textstyle \sum_{i\sim p} z_{ip} \leq n'_p \label{feq:assgnlp1} \\
	& \quad \forall i\in [m] ,  &&\quad  \textstyle \sum_{p\sim i} c_pz_{ip}  \geq D'_i \label{feq:assgnlp2}\\
	& \quad \forall i\in [m], && \quad \textstyle \sum_{p\sim i} z_{ip}  \leq  f'_i \label{feq:assgnlp3}
\end{alignat}
where $i\sim p$ implies $c_p\leq \eps D_i$. If the residual LP has no solution, then our guess of big configurations is infeasible.
We are also guaranteed some guess is correct and we get a feasible solution to above LP.
Therefore, we apply Theorem~\ref{fthm:shmoystardos} to get an integral solution $\zz_{ip}$ satisfying \eqref{feq:assgnlp1},\eqref{feq:assgnlp3},
and $\forall i\in [m], ~~ \sum_{p\sim i} c_p\zz_{ip}  \geq D'_i - \epsilon D_i$.
Therefore in all every machine receives capacity $\geq D_i(1-\epsilon)$. The total running time is dominated by the enumeration step.
This proves~\Cref{fthm:q}.
%For the correct guess, the above LP is feasible. If so, using the fact that $c_p \leq D_i$ for all $i\sim p$, we can get a feasible solution with a slight hit in the demand.
%\begin{lemma}\label{flem:iter}
%If \eqref{feq:asslp1}-\eqref{feq:asslp3} is feasible, then there is an integral assignment $\zz_{ip}$ which satisfies \eqref{feq:assgnlp1} and \eqref{feq:assgnlp3}, and
%
%\end{lemma}
%\begin{proof}
%The proof is via iterative rounding. We look at a vertex solution and if some $z_{ip}$ is $0$ or $1$, we set $\zz_{ip}$ the same.
%If for any machine $i$, the number of $z_{ip}$s is $\leq 5$, we set $\zz_{ip} = 0$ for these and delete the constraint in \eqref{feq:assgnlp2} and \eqref{feq:assgnlp3} corresponding to $i$.
%For this machine $i$, the integral assignment is fixed with cardinality $\le f'_i$ and with total load $\geq D'_i - 5\max_p c_p \geq D'_i - 5\eps D_i$.
%
%Therefore, we may assume that the vertex solution $z$ has at least $5$ non-zeros for every row $i$ and at least $2$ non-zeros for every column $p$.
%Let the number of rows (machines) at any point be $m'$ and number of columns (jobs) be $n'$.  The above observation implies the fractional support is $\geq \max(5m',2n')$
%Since $z$ is a vertex solution, the fractional support is of cardinality $\leq n' + 2m'$.
%This is a contradiction.
%\end{proof}
%\begin{proof}[{\bf Proof of Theorem~\ref{fthm:q}}]
%The final algorithm is as follows.
%First modify the inputs and then perform the grouping obtaining the feasible configurations for every group.
%For every guess of $\pv^{(r,s)}_t$'s in the enumeration step, obtain the residual problem and check if \eqref{feq:assgnlp1}-\eqref{feq:assgnlp3} is feasible or not. If not, then the guess is infeasible.
%If yes, then Lemma~\ref{flem:iter} implies a residual solution which satisfies the cardinality constraints, and every machine $i$ obtains a total load of at least $D_i (1 - 5\epsilon)$.
%The running time is dominated by the number of guesses which is $C_\epsilon^{O(\log^3 n)}$.
%\end{proof}
%
%

%
%The second constraint states that the total number of big jobs need to be respected. We introduce another piece of notation.
%Let  $n(\phi,p)$ denotes the number of copies of $p$ in $\phi$.
%More precisely, if $\phi\in \Phi^{(r,s)}$, then $n(\phi,p)=0$ if $p$ is small for $i\in M^{(r,s)}$, otherwise $n(\phi,p) = \phi_k$ where $c_p = \epsilon(1+\epsilon)^{r+k}$.
%\begin{equation}\label{feq:milp2}
%\forall p\in [P], ~~~~ \sum_{r,s} \sum_{t=1}^{N_0} \pv^{(r,s)}_t n(\phi^{(r,s,t)},p) + \sum_{i=1}^m z_{ip} = n_p
%\end{equation}
%
%Next, we add the machine specific constraints. The first constraint among these states that the total load in any class $M^{(\ell)}$ is bounded.
%The first term in the LHS accounts for the total load from the big jobs to all the machines in $M^{(\ell)}$, and the second term accounts for the small jobs.
%\begin{equation}\label{feq:milp4}
%\forall 0\leq \ell \leq N_1, ~~~~ \sum_{t=1}^{N_0} \pv^{(\ell)}_t \dem(\phi^{(t)}) ~ + ~ \sum_{i\in M^{(\ell)}} \sum_{j\in J_\sm} z_{ij}p_j ~\leq~ m_\ell T
%\end{equation}
%%The next few machine specific constraints imposes the cardinality restriction.
%%First we impose that if $|\phi^{(t)}| > (1+\epsilon)^\ell$, then we must have $\pv^{(\ell)}_t = 0$.
%%\begin{equation}\label{feq:milp5}
%%\forall 1\leq \ell \leq N_1, 1\leq t\leq N_0, ~~~~ \pv^{(\ell)}_t\cdot\left((1+\epsilon)^\ell - |\phi^{(t)}| \right) \geq 0
%%\end{equation}
%Second, we impose that the total number of slots/cardinality used by machines in $M^{(\ell)}$ is at most $(1+\epsilon)^\ell m_\ell$ for $1\le \ell \le N_1$.
%We take care of  $\ell = 0$ separately (since $f_i$'s for machines in $M^{(0)}$ can be very different).
%\begin{equation}\label{feq:milp6}
%\forall 1\leq \ell \leq N_1, ~~~~ \sum_{t=1}^{N_0} \pv^{(\ell)}_t |\phi^{(t)}|~ + ~ \sum_{i\in M^{(\ell)}} \sum_{j\in J_\sm} z_{ij} ~\leq~ m_\ell (1+\epsilon)^\ell
%\end{equation}
%For machines in $M^{(0)}$, we just impose the cardinality constraints from small jobs. The reason is that any large configuration can only violate the cardinality constraint by an $\epsilon$-factor.
%\begin{equation}\label{feq:milp7}
%\forall i\in M^{(0)}, ~~~~ \sum_{j\in J_\sm} z_{ij} ~\leq~ f_i
%\end{equation}
%This completes the description of the MILP. It has $O_\epsilon(1)$ integer variables, $O(mn)$ non-integer variables, and $O(m+n) + O_\epsilon(1)$ constraints.
%We end this section with the following theorem.
%\begin{theorem}\label{fthm:p1}
%	If $T \geq \opt$, then the MILP given by \eqref{feq:milp1}-\eqref{feq:milp7} is feasible.
%\end{theorem}
%\begin{proof}
%\end{proof}
%\subsection{Solving the MILP}
%
%\subsection{Rounding}
%Given $T$ we check if the MILP has a feasible solution. If not, we know $T < \opt$.
%Let $\pv^{(\ell)}_t$ for $0\leq\ell\leq N_1$ and $1\leq t\leq N_0$, and $z_{ij}$ for $1\leq i\leq m$ and $j\in J_\sm$ be a feasible solution. Our objective is to find a schedule which respects the cardinality constraints
%and the total load on any machine is $\leq (1+O(\epsilon))T$.
%
%The first step is to note that getting to solutions which satisfy the cardinality constraints approximately suffices.
%
%%\end{document}
%%
%%We are given parameters $m$ the number of machines, $D$ the demand on each machine, and $F$ the cardinality of each machine. We are also given a collection $(c_1,\ldots,c_P)$ of capacities and we assume $c_P\leq D$.
%%We are interested in approximately capturing all {\em supply} vectors $(y_1,\ldots,y_P)$ such that given $\ceil{y_p}$ (maybe even $\floor{y_p}$) copies of jobs with capacity $c_p$ we can satisfy the $m$ demands satisfying the cardinality constraints.
%%
%%Call a capacity $p$ big if $c_p > \eps D$ and small otherwise. For any $1\leq k\leq K = O(1/\eps)$, let $C_k$ denote the jobs with capacities $c_p \in \eps D\cdot \left((1+\eps)^{k-1},(1+\eps)^k\right]$.
%%A configuration $\phi \in \Z^K_{\ge 0}$ is a vector such that $\phi_k$ indicates the number of jobs of capacity  chosen from $C_k$. The total capacity $\dem(\phi)$ of a configuration is defined to be $\sum_{k=1}^K \phi_k (1+\eps)^k\eps D$ -- note that this is an {\em upper bound} on the actual capacity, but off by a multiplicative $(1+\eps)$-factor. We also let $|\phi| = \sum_{k=1}^K \phi_k$, the cardinality of the configuration.
%%
%%A configuration is {\em allowed} iff $\phi_k \leq 1/\eps$ for all $k$. Therefore, the number of allowed configurations is $N_\eps$, a constant depending only on $\eps$. We order them in a arbitrarty but fixed manner -- the $i$th  allowed configuration is $\phi^{(i)}$.
%%
%%A {\em profile vector} $\pv = (\pv_1,\ldots,\pv_{N_\eps})$ is a vector where $\pv_i\in \Z_{\ge 0}$ indicates the number of machines which choose the allowed configuration $\phi^{(i)}$.
%%We let $M_i$ denote the $\pv_i$  machines getting $\phi^{(i)}$; that is these are the machines $j$ with $\sum_{t < i} \pv_t < j \le \sum_{t \leq i+1} \pv_t$.
%%We have $\sum_i \pv_i = m$, the number of machines. The total number of profile vectors is at most $m^{N_\eps}$.
%%Let $\mathbf{B}$ be the set of all feasible profile vectors.
%%
%%Fix a profile vector $\pv$ and define the following polytope. This has variables $(y_1,\ldots, y_P)$ and variables $z_{jp}$ for all $j = 1,\ldots,m$ and $p$ s.t. $c_p \leq \eps D$.
%%\begin{alignat}{4}
%%	\calP(\pv) ~= &  \{(y_1,\ldots,y_P):   && \notag \\
%%	& \quad \forall ~\textrm{small}~ p,   &&\quad  \textstyle \sum_{j=1}^m z_{jp} \leq y_p \label{feq:1} \\
%%	& \quad  \forall 1\leq k\leq K,           && \quad  \textstyle \sum_{i=1}^{N_\eps} \pv_i \phi^{(i)}_k  \leq \sum_{p\in C_k} y_p \label{feq:2} \\
%%	& \forall i\in [N_\eps], \textstyle j\in M_i , && \quad \textstyle \sum_{p:~\textrm{small}} z_{jp}c_p + \dem(\phi^{(i)}) \geq D \label{feq:3}\\
%%	& \forall i\in [N_\eps], \textstyle j\in M_i , && \quad \textstyle \sum_{p:~\textrm{small}} z_{jp} + |\phi^{(i)}| \leq F\} \label{feq:4}
%%\end{alignat}
%%The final polytope is the convex hull of all these polytopes.
%%\begin{equation}
%%\calP := \{y = (y_1,\ldots,y_P):\exists \pv\in\mathbf{B} ~\textrm{s.t.}~y\in \calP(\pv) \}
%%\end{equation}
%%
%%\begin{theorem}
%%	If $(y_1,\ldots,y_P)$ is a feasible vector for the scheduling problem, then $\calP$ is feasible. Conversely, given any solution $(y_1,\ldots,y_P)\in \calP$,
%%	we can find a feasible schedule giving $(1-O(\eps))D$ to every machine.
%%\end{theorem}
%%\begin{proof}
%%	One direction is easy, and we focus on the other direction. Let $\pv$ be the profile for which $y\in \calP(\pv)$.
%%	For every $1\le k\le K$, we scale down the capacity of all jobs in $C_k$ to $\eps D(1+\eps)^{k-1}$ and call them jobs of type $k$; note we have $\sum_{p\in C_k} y_p$ of them available fractionally.
%%	For $i\in [N_\eps]$, we assign all machines in $M_i$ the configuration $\phi^{(i)}$, that is, we assign $\phi^{(i)}_k$ jobs of type $k$ to machine $j\in M_i$. \eqref{feq:2} ensures that we have these many jobs available fractionally.
%%	%After assigning all these big jobs, we have a residual problem where machine $j\in M_i$ has obtained capacity $\dem(\phi^{(i)})(1+\eps)^{-1}$. Therefore, $\sum_{p:~\textrm{small}} z_{jp}c_p \geq
%%	After allocating the big jobs, the cardinality constraint of machine $j\in M_i$ becomes $F - |\phi^{(i)}| \geq \sum_{p: small} z_{jp}$. The residual problem can be rounded now as Ravishankar pointed out.
%%\end{proof}
%%
%%
%%
