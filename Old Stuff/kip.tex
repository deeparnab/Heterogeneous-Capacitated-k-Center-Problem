\documentclass{article}[11pt]
%In case of a space crunch...
	%\usepackage[subtle]{savetrees}
	%\usepackage{times}
	\usepackage{a4,geometry}
%In case of a space surplus
	%\usepackage[parfill]{parskip}

\usepackage{graphicx}
\usepackage{amsmath,amssymb,amsthm,mathtools}
\usepackage{paralist}
\usepackage{bm}
\usepackage{xspace}
\usepackage{url}
\usepackage{fullpage, prettyref}
\usepackage{boxedminipage}
\usepackage{wrapfig}
\usepackage{ifthen}
\usepackage{color}
\usepackage{xcolor}
\usepackage{framed}
\usepackage{algorithm}
\usepackage[pagebackref,letterpaper=true,colorlinks=true,pdfpagemode=none,urlcolor=blue,linkcolor=blue,citecolor=violet,pdfstartview=FitH]{hyperref}
\usepackage{fullpage}
\usepackage[noend]{algpseudocode}
\usepackage{enumitem}
%\usepackage{times}
% For restating theorems in appendix
\usepackage{thmtools}
\usepackage{thm-restate,cleveref}
%Usage:
	%\begin{restatable}[Goldbach's conjecture]{thm}{goldbach}
	%\label{thm:goldbach}
	%Every even integer greater than 2 can be expressed as the sum of two primes.
	%\end{restatable}
	%Then type \goldbach* to recall.

\newtheorem{theorem}{Theorem}[section]
\newtheorem{conjecture}[theorem]{Conjecture}
\newtheorem{lemma}[theorem]{Lemma}
\newtheorem{claim}[theorem]{Claim}
\newtheorem{question}{Question}
\newtheorem{corollary}[theorem]{Corollary}
\newtheorem{definition}{Definition}
\newtheorem{proposition}[theorem]{Proposition}
\newtheorem{fact}[theorem]{Fact}
\newtheorem{example}[theorem]{Example}
\newtheorem{assumption}[theorem]{Assumption}
\newtheorem{observation}[theorem]{Observation}
\newtheorem{remark}[theorem]{Remark}

\newcommand{\comment}[1]{\textit {\em \color{blue} \footnotesize[#1]}\marginpar{\tiny\textsc{\color{blue} To Do!}}}
\newcommand{\mcomment}[1]{\marginpar{\tiny\textsf{\color{red} #1}}}

\newcommand{\ignore}[1]{}

%% Calligraphic letters

\newcommand{\cA}{{\cal A}}
\newcommand{\cB}{\mathcal{B}}
\newcommand{\cC}{{\cal C}}
\newcommand{\cD}{\mathcal{D}}
\newcommand{\cE}{{\cal E}}
\newcommand{\cF}{\mathcal{F}}
\newcommand{\cG}{\mathcal{G}}
\newcommand{\cH}{{\cal H}}
\newcommand{\cI}{{\cal I}}
\newcommand{\cJ}{{\cal J}}
\newcommand{\cL}{{\cal L}}
\newcommand{\cM}{{\cal M}}
\newcommand{\cP}{\mathcal{P}}
\newcommand{\cQ}{\mathcal{Q}}
\newcommand{\cR}{{\cal R}}
\newcommand{\cS}{\mathcal{S}}
\newcommand{\cT}{{\cal T}}
\newcommand{\cU}{{\cal U}}
\newcommand{\cV}{{\cal V}}
\newcommand{\cX}{{\cal X}}


\newcommand{\R}{\mathbb R}
\newcommand{\N}{\mathbb N}
\newcommand{\F}{\mathbb F}
\newcommand{\Z}{{\mathbb Z}}
\newcommand{\eps}{\varepsilon}
\newcommand{\lam}{\lambda}
\newcommand{\sgn}{\mathrm{sgn}}
\newcommand{\poly}{\mathrm{poly}}
\newcommand{\polylog}{\mathrm{polylog}}
\newcommand{\littlesum}{\mathop{{\textstyle \sum}}}
\newcommand{\half}{{\textstyle \frac12}}
\newcommand{\la}{\langle}
\newcommand{\ra}{\rangle}
\newcommand{\wh}{\widehat}
\newcommand{\wt}{\widetilde}
\newcommand{\calE}{{\cal E}}
\newcommand{\calL}{{\cal L}}
\newcommand{\calF}{{\cal F}}
\newcommand{\calW}{{\cal W}}
\newcommand{\calH}{{\cal H}}
\newcommand{\calN}{{\cal N}}
\newcommand{\calO}{{\cal O}}
\newcommand{\calP}{{\cal P}}
\newcommand{\calV}{{\cal V}}
\newcommand{\calS}{{\cal S}}
\newcommand{\calT}{{\cal T}}
\newcommand{\calD}{{\cal D}}
\newcommand{\calC}{{\cal C}}
\newcommand{\calX}{{\cal X}}
\newcommand{\calY}{{\cal Y}}
\newcommand{\calZ}{{\cal Z}}
\newcommand{\calA}{{\cal A}}
\newcommand{\calB}{{\cal B}}
\newcommand{\calG}{{\cal G}}
\newcommand{\calI}{{\cal I}}
\newcommand{\calJ}{{\cal J}}
\newcommand{\calR}{{\cal R}}
\newcommand{\calK}{{\cal K}}
\newcommand{\calU}{{\cal U}}
\newcommand{\barx}{\overline{x}}
\newcommand{\bary}{\overline{y}}

\newcommand{\ba}{\boldsymbol{a}}
\newcommand{\bb}{\boldsymbol{b}}
\newcommand{\bp}{\boldsymbol{p}}
\newcommand{\bt}{\boldsymbol{t}}
\newcommand{\bv}{\boldsymbol{v}}
\newcommand{\bx}{\boldsymbol{x}}
\newcommand{\by}{\boldsymbol{y}}
\newcommand{\bz}{\boldsymbol{z}}
\newcommand{\br}{\boldsymbol{r}}
\newcommand{\bh}{\boldsymbol{h}}

\newcommand{\bA}{\boldsymbol{A}}
\newcommand{\bD}{\boldsymbol{D}}
\newcommand{\bG}{\boldsymbol{G}}

\newcommand{\bR}{\boldsymbol{R}}
\newcommand{\bS}{\boldsymbol{S}}
\newcommand{\bX}{\boldsymbol{X}}
\newcommand{\bY}{\boldsymbol{Y}}
\newcommand{\bZ}{\boldsymbol{Z}}

\newcommand{\NN}{\mathbb{N}}
\newcommand{\RR}{\mathbb{R}}

\newcommand{\abs}[1]{\left\lvert #1 \right\rvert}
\newcommand{\norm}[1]{\left\lVert #1 \right\rVert}
\newcommand{\ceil}[1]{\lceil#1\rceil}
\newcommand{\Exp}{\EX}
\newcommand{\floor}[1]{\lfloor#1\rfloor}

\newcommand{\EX}{\hbox{\bf E}}
\newcommand{\prob}{{\rm Prob}}

\newcommand{\gset}{Y}
\newcommand{\gcol}{{\cal Y}}

%% Hyper-linked References
\newcommand{\Sec}[1]{\hyperref[sec:#1]{\S\ref*{sec:#1}}} %section
\newcommand{\Eqn}[1]{\hyperref[eq:#1]{(\ref*{eq:#1})}} %equation
\newcommand{\Fig}[1]{\hyperref[fig:#1]{Fig.\,\ref*{fig:#1}}} %figure
\newcommand{\Tab}[1]{\hyperref[tab:#1]{Tab.\,\ref*{tab:#1}}} %table
\newcommand{\Thm}[1]{\hyperref[thm:#1]{Theorem\,\ref*{thm:#1}}} %theorem
\newcommand{\Fact}[1]{\hyperref[fact:#1]{Fact\,\ref*{fact:#1}}} %fact
\newcommand{\Lem}[1]{\hyperref[lem:#1]{Lemma\,\ref*{lem:#1}}} %lemma
\newcommand{\Prop}[1]{\hyperref[prop:#1]{Prop.~\ref*{prop:#1}}} %property
\newcommand{\Cor}[1]{\hyperref[cor:#1]{Corollary~\ref*{cor:#1}}} %corollary
\newcommand{\Conj}[1]{\hyperref[conj:#1]{Conjecture~\ref*{conj:#1}}} %conjecture
\newcommand{\Def}[1]{\hyperref[def:#1]{Definition~\ref*{def:#1}}} %definition
\newcommand{\Alg}[1]{\hyperref[alg:#1]{Alg.~\ref*{alg:#1}}} %algorithm
\newcommand{\Ex}[1]{\hyperref[ex:#1]{Ex.~\ref*{ex:#1}}} %example
\newcommand{\Clm}[1]{\hyperref[clm:#1]{Claim~\ref*{clm:#1}}} %example


\def\diam{\mathrm{diam}}
\def\dist{\mathrm{dist}}
\def\mckc{{\sffamily Movable Cap-$k$-Center}\xspace}
\def\cckp{{\sffamily Cardinality-Constrained Knapsack Cover}\xspace}
\def\opt{\mathsf{OPT}}
\def\capkc{{\sffamily Cap-$k$-Center }}
\def\x{{\mathsf x}}
\def\y{y^{{\mathsf{int}}}}
\def\Supp{\mathsf{Supp}\xspace}
\def\n{n^{(1)}}
\def\nn{n^{(2)}}
\def\effc{c_{\mathrm{eff}}}

\def\Sone{\calS_{\mathsf{round}}}
\def\Stwo{\calS_{\mathsf{nexp}}}

\def\Cb{C_{\mathsf{blue}}}
\def\Cbb{C_{\mathsf{black}}}
\def\Cr{C_{\mathsf{red}}}
\def\Cp{C_{\mathsf{purple}}}
\def\Cd{C_{\mathsf{del}}}

\def\pv{\mathbf{b}}
%\newcommand{\gset}{Y}
%\newcommand{\gcol}{{\cal Y}}
\newcommand{\brp}{{(p)}}
\renewcommand{\br}[1]{{(#1)}}
\newcommand{\bc}{{\bar c}}
\newcommand{\dem}{\mathsf{load}}

\DeclareMathOperator*{\argmax}{arg\,max}
\DeclareMathOperator*{\argmin}{arg\,min}


\begin{document}
\title{An EPTAS for the $k_i$-Partitioning Problem}
\author{Deeparnab Chakrabarty \\ MSR India \and Ravishankar Krishnaswamy \\ MSR India \and Amit Kumar \\ IIT Delhi}
\date{}
\maketitle
\section{Introduction}
In the $k_i$-Partitioning problem we are given $n$ numbers $p_1,\ldots,p_n$ which need to be partitioned into $m$ sets $S_1,\ldots,S_m$ such that for all $1\le i\le m$, $|S_i| \leq k_i$ for some given non-negative integers $k_1,\ldots,k_m$.
The objective is to minimize the maximum value of any set, that is minimize $\max_{i=1}^m \sum_{j\in S_m} p_j$. This is a generalization of the classic makespan minimization problem in scheduling on identical machines.
In the Graham notation, this problem is denoted as $P|k_i|C_{max}$. 

\paragraph{Relevance} Write.

\paragraph{History} Write.

\paragraph{Technical Difficulty} Write.  \\

\noindent
We given an EPTAS for the problem. Give Theorem statement. 

\paragraph{Technique}

	
\section{Algorithm}

Throughout the paper we fix $\delta,\eps > 0$. As is standard, we assume we have a guess $T$ of the optimum; given such a guess we would either prove $\opt > T$ or obtain a feasible schedule of makespan at most $(1+\eps)T$.
Subsequently, binary search would provide a schedule of makespan $\leq \opt(1+\eps)$. \smallskip

\def\sm{\mathsf{small}}
Call a job $j$ big if $p_j > \eps T$, and small otherwise. Let $J_\sm$ 	be the set of small jobs.
For any $1\leq k\leq K = O(1/\eps)$, let $J_k$ denote the big jobs with capacities $p_j \in \eps T\cdot \left((1+\eps)^{k-1},(1+\eps)^k\right]$.
A configuration $\phi \in \Z^K_{\ge 0}$ is a vector where $\phi_k$ indicates the number of jobs of chosen from $J_k$. The total load $\dem(\phi)$ of a configuration is defined to be $\sum_{k=1}^K \phi_k (1+\eps)^{k-1}\eps T$ -- note that this is a {\em lower bound} on the actual load but is within a multiplicative $(1+\eps)$-factor. Finally, we let $|\phi| = \sum_{k=1}^K \phi_k$, the cardinality of the configuration. A configuration $\phi$ is {\em allowed} if $\phi_k \leq 1/\epsilon$. The number of allowed configurations is denoted as $N_0 \leq (1/\epsilon)^{(1/\epsilon)}$.
We call this set $\Phi$ which we order in an arbitrary but fixed manner. The $i$th member is denoted as $\phi^{(i)}$.

\smallskip
Call a machine $i$ large if $k_i > 1/\eps^2$. Let $M^{(0)}$ denote the set of large machines, and let $m_0 = |M^{(0)}|$. 
Every other machine is small and these are partitioned into $N_1 = O(1/\eps \ln(1/\eps))$ groups as follows. For $1\leq \ell \leq N_1$, define 
$M^{(\ell)} := \{i: (1+\eps)^{\ell-1} < k_i \leq (1+\epsilon)^\ell\}$. Let $m_\ell := |M^{(\ell)}|$.

\subsection{MILP Formulation}
For every $0\leq \ell\leq N_1$ and $1\le t\le N_0$, we have an {\em integer variable} $\pv^{(\ell)}_t \in \Z_{\geq 0}$ which indicated the number of machines in $M^{(\ell)}$ whose allocation of big jobs is given by $\phi^{(t)}$.
That is, the number of big jobs from $J_k$, for $1\le k\le K$ assigned to these machines is $\phi^{(t)}_k$. Note that the number of variables is at most $N_0 N_1 = 2^{O(1/\eps \ln(1/\epsilon))}$.

\noindent
The first constraint states that the total number of machines need to be respected.
\begin{equation}\label{eq:milp1}
\forall 0\leq \ell \leq N_1, ~~~~ \sum_{t=1}^{N_0} \pv^{(\ell)}_t = m^{(\ell)}
\end{equation}
The second constraint states that the total number of big jobs need to be respected.
\begin{equation}\label{eq:milp2}
\forall 1\leq k \leq K, ~~~~ \sum_{\ell = 0}^{N_1} \sum_{t=1}^{N_0} \pv^{(\ell)}_t \phi^{(t)}_k = |J_k|
\end{equation}
For every machine $i$ and every small job $j$, we have a variable $z_{ij} \in [0,1]$. This indicates whether job $j$ is allocated to machine $i$.
\begin{equation}\label{eq:milp3}
\forall j\in J_\sm, ~~~~ \sum_{i=1}^m z_{ij} = 1
\end{equation}
Finally, we add the machine specific constraints. The first constraint among these states that the total load in any class $M^{(\ell)}$ is bounded.
The first term in the LHS accounts for the total load from the big jobs to all the machines in $M^{(\ell)}$, and the second term accounts for the small jobs.
\begin{equation}\label{eq:milp4}
\forall 0\leq \ell \leq N_1, ~~~~ \sum_{t=1}^{N_0} \pv^{(\ell)}_t \dem(\phi^{(t)}) ~ + ~ \sum_{i\in M^{(\ell)}} \sum_{j\in J_\sm} z_{ij}p_j ~\leq~ m_\ell T
\end{equation}
The next few machine specific constraints imposes the cardinality restriction. 
First we impose that if $|\phi^{(t)}| > (1+\epsilon)^\ell$, then we must have $\pv^{(\ell)}_t = 0$.
\begin{equation}\label{eq:milp5}
\forall 1\leq \ell \leq N_1, 1\leq t\leq N_0, ~~~~ \pv^{(\ell)}_t\cdot\left((1+\epsilon)^\ell - |\phi^{(t)}| \right) \geq 0
\end{equation}
Second, we impose that the total number of slots/cardinality used by machines in $M^{(\ell)}$ is at most $(1+\epsilon)^\ell m_\ell$ for $1\le \ell \le N_1$.
We take care of  $\ell = 0$ separately (since $k_i$'s for machines in $M^{(0)}$ can be very different).
\begin{equation}\label{eq:milp6}
\forall 1\leq \ell \leq N_1, ~~~~ \sum_{t=1}^{N_0} \pv^{(\ell)}_t |\phi^{(t)}|~ + ~ \sum_{i\in M^{(\ell)}} \sum_{j\in J_\sm} z_{ij} ~\leq~ m_\ell (1+\epsilon)^\ell
\end{equation}
For machines in $M^{(0)}$, we just impose the cardinality constraints from small jobs. The reason is that any large configuration can only violate the cardinality constraint by an $\epsilon$-factor.
\begin{equation}\label{eq:milp7}
\forall i\in M^{(0)}, ~~~~ \sum_{j\in J_\sm} z_{ij} ~\leq~ k_i
\end{equation}
This completes the description of the MILP. It has $O_\epsilon(1)$ integer variables, $O(mn)$ non-integer variables, and $O(m+n) + O_\epsilon(1)$ constraints.
We end this section with the following theorem.
\begin{theorem}\label{thm:p1}
If $T \geq \opt$, then the MILP given by \eqref{eq:milp1}-\eqref{eq:milp7} is feasible.
\end{theorem}
\begin{proof}
\end{proof}
\subsection{Solving the MILP}

\subsection{Rounding}
Given $T$ we check if the MILP has a feasible solution. If not, we know $T < \opt$. 
Let $\pv^{(\ell)}_t$ for $0\leq\ell\leq N_1$ and $1\leq t\leq N_0$, and $z_{ij}$ for $1\leq i\leq m$ and $j\in J_\sm$ be a feasible solution. Our objective is to find a schedule which respects the cardinality constraints
	and the total load on any machine is $\leq (1+O(\epsilon))T$.
	
The first step is to note that getting to solutions which satisfy the cardinality constraints approximately suffices.
\begin{lemma}
	Given an assignment of jobs such that the load on any machine is at most $T(1+\epsilon_1)$ such that machine $i$ gets $\leq (1+\epsilon_2)k_i$ jobs, 
	we can find another assignment which satisfies the cardinality constraints and the load of any machine is $\leq T(1+\epsilon_3)$. 
\end{lemma}
\begin{proof}
	{\bf This doesn't seem to be correct.}
	The issue is that all but one of the machines can be violating $k_i$'s and then mass can't be moved on to the one machine since the load would go to $n\eps T$.
\end{proof}
\end{document}
 