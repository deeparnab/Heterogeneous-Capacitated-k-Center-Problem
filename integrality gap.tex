\documentclass{article}[11pt]
%In case of a space crunch...
	%\usepackage[subtle]{savetrees}
	%\usepackage{times}
	\usepackage{a4,geometry}
%In case of a space surplus
	%\usepackage[parfill]{parskip}

\usepackage{graphicx}
\usepackage{amsmath,amssymb,amsthm,mathtools}
\usepackage{paralist}
\usepackage{bm}
\usepackage{xspace}
\usepackage{url}
\usepackage{fullpage, prettyref}
\usepackage{boxedminipage}
\usepackage{wrapfig}
\usepackage{ifthen}
\usepackage{color}
\usepackage{xcolor}
\usepackage{framed}
\usepackage{algorithm}
\usepackage[pagebackref,letterpaper=true,colorlinks=true,pdfpagemode=none,urlcolor=blue,linkcolor=blue,citecolor=violet,pdfstartview=FitH]{hyperref}
\usepackage{fullpage}
\usepackage[noend]{algpseudocode}
\usepackage{enumitem}
%\usepackage{times}
% For restating theorems in appendix
\usepackage{thmtools}
\usepackage{thm-restate,cleveref}
%Usage:
	%\begin{restatable}[Goldbach's conjecture]{thm}{goldbach}
	%\label{thm:goldbach}
	%Every even integer greater than 2 can be expressed as the sum of two primes.
	%\end{restatable}
	%Then type \goldbach* to recall.

\newtheorem{theorem}{Theorem}[section]
\newtheorem{conjecture}[theorem]{Conjecture}
\newtheorem{lemma}[theorem]{Lemma}
\newtheorem{claim}[theorem]{Claim}
\newtheorem{question}{Question}
\newtheorem{corollary}[theorem]{Corollary}
\newtheorem{definition}{Definition}
\newtheorem{proposition}[theorem]{Proposition}
\newtheorem{fact}[theorem]{Fact}
\newtheorem{example}[theorem]{Example}
\newtheorem{assumption}[theorem]{Assumption}
\newtheorem{observation}[theorem]{Observation}
\newtheorem{remark}{Remark}

\newcommand{\comment}[1]{\textit {\em \color{blue} \footnotesize[#1]}\marginpar{\tiny\textsc{\color{blue} To Do!}}}
\newcommand{\mcomment}[1]{\marginpar{\tiny\textsf{\color{red} #1}}}

\newcommand{\ignore}[1]{}

%% Calligraphic letters

\newcommand{\cA}{{\cal A}}
\newcommand{\cB}{\mathcal{B}}
\newcommand{\cC}{{\cal C}}
\newcommand{\cD}{\mathcal{D}}
\newcommand{\cE}{{\cal E}}
\newcommand{\cF}{\mathcal{F}}
\newcommand{\cG}{\mathcal{G}}
\newcommand{\cH}{{\cal H}}
\newcommand{\cI}{{\cal I}}
\newcommand{\cJ}{{\cal J}}
\newcommand{\cL}{{\cal L}}
\newcommand{\cM}{{\cal M}}
\newcommand{\cP}{\mathcal{P}}
\newcommand{\cQ}{\mathcal{Q}}
\newcommand{\cR}{{\cal R}}
\newcommand{\cS}{\mathcal{S}}
\newcommand{\cT}{{\cal T}}
\newcommand{\cU}{{\cal U}}
\newcommand{\cV}{{\cal V}}
\newcommand{\cX}{{\cal X}}


\newcommand{\R}{\mathbb R}
\newcommand{\N}{\mathbb N}
\newcommand{\F}{\mathbb F}
\newcommand{\Z}{{\mathbb Z}}
\newcommand{\eps}{\varepsilon}
\newcommand{\lam}{\lambda}
\newcommand{\sgn}{\mathrm{sgn}}
\newcommand{\poly}{\mathrm{poly}}
\newcommand{\polylog}{\mathrm{polylog}}
\newcommand{\littlesum}{\mathop{{\textstyle \sum}}}
\newcommand{\half}{{\textstyle \frac12}}
\newcommand{\la}{\langle}
\newcommand{\ra}{\rangle}
\newcommand{\wh}{\widehat}
\newcommand{\wt}{\widetilde}
\newcommand{\calE}{{\cal E}}
\newcommand{\calL}{{\cal L}}
\newcommand{\calF}{{\cal F}}
\newcommand{\calW}{{\cal W}}
\newcommand{\calH}{{\cal H}}
\newcommand{\calN}{{\cal N}}
\newcommand{\calO}{{\cal O}}
\newcommand{\calP}{{\cal P}}
\newcommand{\calV}{{\cal V}}
\newcommand{\calS}{{\cal S}}
\newcommand{\calT}{{\cal T}}
\newcommand{\calD}{{\cal D}}
\newcommand{\calC}{{\cal C}}
\newcommand{\calX}{{\cal X}}
\newcommand{\calY}{{\cal Y}}
\newcommand{\calZ}{{\cal Z}}
\newcommand{\calA}{{\cal A}}
\newcommand{\calB}{{\cal B}}
\newcommand{\calG}{{\cal G}}
\newcommand{\calI}{{\cal I}}
\newcommand{\calJ}{{\cal J}}
\newcommand{\calR}{{\cal R}}
\newcommand{\calK}{{\cal K}}
\newcommand{\calU}{{\cal U}}
\newcommand{\barx}{\overline{x}}
\newcommand{\bary}{\overline{y}}

\newcommand{\ba}{\boldsymbol{a}}
\newcommand{\bb}{\boldsymbol{b}}
\newcommand{\bp}{\boldsymbol{p}}
\newcommand{\bt}{\boldsymbol{t}}
\newcommand{\bv}{\boldsymbol{v}}
\newcommand{\bx}{\boldsymbol{x}}
\newcommand{\by}{\boldsymbol{y}}
\newcommand{\bz}{\boldsymbol{z}}
\newcommand{\br}{\boldsymbol{r}}
\newcommand{\bh}{\boldsymbol{h}}

\newcommand{\bA}{\boldsymbol{A}}
\newcommand{\bD}{\boldsymbol{D}}
\newcommand{\bG}{\boldsymbol{G}}

\newcommand{\bR}{\boldsymbol{R}}
\newcommand{\bS}{\boldsymbol{S}}
\newcommand{\bX}{\boldsymbol{X}}
\newcommand{\bY}{\boldsymbol{Y}}
\newcommand{\bZ}{\boldsymbol{Z}}

\newcommand{\NN}{\mathbb{N}}
\newcommand{\RR}{\mathbb{R}}

\newcommand{\abs}[1]{\left\lvert #1 \right\rvert}
\newcommand{\norm}[1]{\left\lVert #1 \right\rVert}
\newcommand{\ceil}[1]{\lceil#1\rceil}
\newcommand{\Exp}{\EX}
\newcommand{\floor}[1]{\lfloor#1\rfloor}

\newcommand{\EX}{\hbox{\bf E}}
\newcommand{\prob}{{\rm Prob}}

\newcommand{\gset}{Y}
\newcommand{\gcol}{{\cal Y}}

%% Hyper-linked References
\newcommand{\Sec}[1]{\hyperref[sec:#1]{\S\ref*{sec:#1}}} %section
\newcommand{\Eqn}[1]{\hyperref[eq:#1]{(\ref*{eq:#1})}} %equation
\newcommand{\Fig}[1]{\hyperref[fig:#1]{Fig.\,\ref*{fig:#1}}} %figure
\newcommand{\Tab}[1]{\hyperref[tab:#1]{Tab.\,\ref*{tab:#1}}} %table
\newcommand{\Thm}[1]{\hyperref[thm:#1]{Theorem\,\ref*{thm:#1}}} %theorem
\newcommand{\Fact}[1]{\hyperref[fact:#1]{Fact\,\ref*{fact:#1}}} %fact
\newcommand{\Lem}[1]{\hyperref[lem:#1]{Lemma\,\ref*{lem:#1}}} %lemma
\newcommand{\Prop}[1]{\hyperref[prop:#1]{Prop.~\ref*{prop:#1}}} %property
\newcommand{\Cor}[1]{\hyperref[cor:#1]{Corollary~\ref*{cor:#1}}} %corollary
\newcommand{\Conj}[1]{\hyperref[conj:#1]{Conjecture~\ref*{conj:#1}}} %conjecture
\newcommand{\Def}[1]{\hyperref[def:#1]{Definition~\ref*{def:#1}}} %definition
\newcommand{\Alg}[1]{\hyperref[alg:#1]{Alg.~\ref*{alg:#1}}} %algorithm
\newcommand{\Ex}[1]{\hyperref[ex:#1]{Ex.~\ref*{ex:#1}}} %example
\newcommand{\Clm}[1]{\hyperref[clm:#1]{Claim~\ref*{clm:#1}}} %example


\def\diam{\mathrm{diam}}
\def\dist{\mathrm{dist}}
\def\mckc{{\sffamily Movable Cap-$k$-Center}\xspace}
\def\cckp{{\sffamily Cardinality-Constrained Knapsack Cover}\xspace}
\def\opt{\mathsf{OPT}}
\def\capkc{{\sffamily Cap-$k$-Center }}
\def\x{{\mathsf x}}
\def\y{y^{{\mathsf{int}}}}
\def\Supp{\mathsf{Supp}\xspace}
\def\n{n^{(1)}}
\def\nn{n^{(2)}}
\def\effc{c_{\mathrm{eff}}}

\def\Sone{\calS_{\mathsf{round}}}
\def\Stwo{\calS_{\mathsf{nexp}}}

\def\Cb{C_{\mathsf{blue}}}
\def\Cbb{C_{\mathsf{black}}}
\def\Cr{C_{\mathsf{red}}}
\def\Cp{C_{\mathsf{purple}}}
\def\Cd{C_{\mathsf{del}}}

\def\pv{\mathbf{b}}
%\newcommand{\gset}{Y}
%\newcommand{\gcol}{{\cal Y}}
\newcommand{\brp}{{(p)}}
\renewcommand{\br}[1]{{(#1)}}
\newcommand{\bc}{{\bar c}}
\newcommand{\dem}{\mathsf{cap}}

\DeclareMathOperator*{\argmax}{arg\,max}
\DeclareMathOperator*{\argmin}{arg\,min}


\begin{document}
\section{Configuration Integrality Gap for $Q|k_i|C_{min}$}
We show that the configuration LP for the problem of allocating jobs to machines of different speeds to maximize the minimum processing time has super-constant integrality gap.
\begin{theorem}
	The integrality gap of CLP is $\Omega(\log n/\log\log n)$.
\end{theorem}
\begin{proof}
\def\M{\mathcal{M}}
Fix $K$. We present an instance $\calI_K$ for which configuration LP is feasible but any integral allocation must violate the demand of some machine by factor $K$.

First we describe the machines in $\calI_K$.
\begin{enumerate}
	\item There is $1$ machine $M_0$ with $D(M_0) = 1$ and $f(M_0) = 1$.
	\item There are $K$ machines $M_1,\ldots,M_K$ with $D(M_i) = K^{-i}$ and $f_i := f(M_i) = K^{2K + 1}\cdot K^{-2i}$.
	\item There are $K$ {\bf classes} of machines $\M_1,\M _2,\ldots, \M _K$. Machines in the same class are equivalent. 
	There are $f_i$ machines in $\M_i$ and they are numbered $N^{(i)}_1,\ldots,N^{(i)}_{f_i}$.
Each machine $N$  in class $i$ has $D(N) = \frac{1}{f_iK^i} = K^{-(2K+ 1)}\cdot K^i$ and $f(N) = 1$.
\end{enumerate}
Now we describe the jobs.
\begin{enumerate}
	\item There are $K$ ``big jobs'' $J_1,\ldots,J_K$ with $c(J_i) = 1$.
	\item There are $K$ other types of jobs of the same capacity. Job $J$ of type $i$ has capacity $c(J) = c_i :=  \frac{1}{f_iK^i} = K^{-(2K+1)}K^i$  and there are $n_i := f_i (1+1/K) = (K+1)K^{2K}K^{-2i}$ of them.
	We divide these $n_i$ jobs into two sets $S_i \cup T_i$ where $|S_i| = f_i$ and $|T_i| = f_i/K$. We order the jobs in $S_i$ arbitrarily and call them $P^{(i)}_1,\cdots,P^{(i)}_{f_i}$.
\end{enumerate}
So, the total number of machines in $\calI_K$ are $1 + K + \sum_{i=1}^K f_i  \leq K^{2K}$ and the number of jobs is $K + (1+1/K)\sum_{i=1}^K f_i \approx K^{2K}$. 

\begin{lemma}
	The Configuration LP is feasible.
\end{lemma}
\begin{proof}
We describe a fractional solution.
\begin{enumerate}
	\item For machine $M_0$ we satisfy as follows: set  $y(M_0,J_i) = 1/K$ for $i=1,\ldots,K$. Note $c(J_i) \geq D(M_0)$ and $|J_i| = 1 = f(M_0)$. 
	\item For machine $M_i$ we satisfy as follows: set $y(M_i,J_i) = 1-1/K$ and $y(M_i,S_i) = 1/K$. Recall $S_i$ are the $f_i$ jobs of type $i$. 
	\begin{itemize}
		\item Note $c(J_i) = 1 \geq D(M_i) = K^{-i}$ and 	$|J_i| = 1 \leq f(M_i) = K^{2K+1}K^{-2i}$ since $i\leq K$.
		\item Note $c(S_i) = |S_i|\cdot \frac{1}{f_iK^i} = \frac{1}{K^i} = D(M_i)$. Note $|S_i| = f_i = f(M_i)$.
	\end{itemize}
\item For $1\leq i\leq K$, for a machine $N^{(i)}_j$ in class $i$, where $1\leq j\leq f_i$, we satisfy it as follows: $y(N^{(i)}_j, P^{(i)}_j) = 1-1/K$ and $y(N^{(i)}_j, t) = 1/f_i$ for all $t\in T_i$.
Since $|T_i| = f_i/K$, the total fractional $y$-amount that $N^{(i)}_j$ gets is $1$. Also note that $N^{(i)}_j$ gets singleton jobs of type $i$ whose capacity is $\frac{1}{f_iK_i} = D(N^{(i)}_j)$.
\end{enumerate}
We need to show that no job is over allocated.
\begin{enumerate}
	\item The big jobs $J_i$ is given $1/K$ to $M_0$ and $(1-1/K)$ to $M_i$.
	\item For $1\leq i\leq K$, $1\leq j\leq f_i$, job $P^{(i)}_j \in S_i$ is given $1/K$ to $M_i$ and $(1-1/K)$ to $N^{(i)}_j\in \M_i$.
	\item For $1\leq i\leq K$, job $t\in T_i$ is given $1/f_i$ to the $f_i$ machines of $\M_i$.
\end{enumerate}
This completes the description of the feasible solution.
\end{proof}
\begin{lemma}
	The integral optimum must violate some machine by factor $\Omega(K)$.
	\end{lemma}
	\begin{proof}
Lets take machines in $\M_i$. Recall all machines here have demand of $\frac{1}{f_iK^i}$ and cardinality constraint of $1$.
Thus in the integral optimum, they {\bf must} get one job which is either big, or of type $i$ or larger. 
Now, the total number of jobs of type $j > i$ are 
\[
\sum_{j>i} f_j(1+1/K) = (K+1)K^{2K} \sum_{j > i} K^{-2j} \leq  (K+1)K^{2K}K^{-2i} \sum_{\ell=1}^\infty K^{-2\ell} = O\left(f_i/K\right)
\]
So, at least $(1 - \Theta(1/K))f_i$ of the machines in $\M_i$ get a job of type $i$ (or a big job but lets assume for now this don't happen -- can be ma).
Therefore, the number of type $i$ jobs left after satisfying machines $(M_0,\ldots,M_K)$ are only $\Theta(f_i/K)$.


Now take a machine $M_i$. We have $f(M_i) = f_i$ and $D(M_i) = 1/K^i$. 
First note that jobs of type $j < i$ are ``useless'' for $M_i$. Any $f_i$ of them (best to take them of type $(i-1)$)  gives capacity $f_i\cdot c_{i-1}  = \frac{f_i}{f_{i-1}K^{i-1}} = \frac{1}{K^{i+1}} =  \frac{1}{K}\cdot D(M_i)$. So any subset of these jobs that can fit in $M_i$ gives capacity $\leq D(M_i)/K$.
On the other hand, the total capacity of jobs remaining from type $j \geq i$ is  $\sum_{j\geq i} \Theta(f_j/K)\cdot \frac{1}{f_jK^j} = \Theta(1/K)\sum_{j\geq i} \frac{1}{K^j} = \Theta(D(M_i)/K)$.

Therefore, any machine $M_i$ can't get more than $D(M_i)/K$ from the ``small'' jobs. But then they all can't get big jobs. 
\end{proof}
The above two lemmas prove the theorem after noting that $K = \Theta(\log n/\log\log n)$ where $n$ is either the number of machines of jobs.
\end{proof}
\end{document}
 