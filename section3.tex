%\newpage
\section{Max-Min Allocation Problems and Supply Polyhedra}\label{sec:supplypolyhedra}
An instance of the \cckp problem has $m$ machines $M$ with demands $D_1,\ldots,D_m$ and cardinality constraints $f_1,\ldots, f_m$, and $n$  types of jobs $J$ with capacities $c_1,\ldots,c_n$ respectively.
In $Q||c_{min}$, there are no $f_i$'s, or equivalently $f_i = \infty$.
%In the version with cardinality constraints, that is $Q|f_i|C_{min}$ we are also given positive integers $f_1,\ldots, f_m$.

A {\em supply vector} $(s_1,\ldots,s_n)$ where each $s_j$ is a non-negative integer
is called {\em feasible} for instances of these problems if the ensemble formed by $s_j$ copies of jobs of capacity $c_j$ can be allocated feasibly to satisfy all the demands.
The {\em supply polyhedra} of these instances desires to capture these feasible supply vectors.

\begin{definition}[Supply Polyhedron]\label{def:supp-poly}
	Given an instance $\calI$ for a max-min allocation problem, a polyhedron $\calP(\calI)$ is called an $\alpha$-approximate supply polyhedron if
	(a) all feasible supply vectors lie in $\calP(\calI)$, and (b) given any non-negative integer vector $(s_1,\ldots,s_n)\in \calP(\calI)$ there exists an assignment
	of the $s_j$ jobs of capacity $c_j$ to the machines such that machine $i$ receives a total capacity of $\geq D_i/\alpha$.
\end{definition}

Ideally, we would like {\em exactly} supply polyhedra. One guess would be the convex hull of all the supply vectors; indeed this is the tightest polytope satisfying condition (a).
Unfortunately, there are instances of $Q||C_{min}$ (and even for the uniform case $P||C_{min}$) where the convex hull of supply vectors contains infeasible integer points.
This rules out exact or even $(1+\epsilon)$-approximate supply polyhedra.
%This instance is motivated by integrality gap examples for machine scheduling~\cite{bibid}.
\begin{theorem}\label{thm:no-supp}
	There cannot exist $\alpha$-approximate supply polyhedra (or convex sets) for $\alpha < 1.001$ for all  $P||C_{min}$ instances.
\end{theorem}
\begin{proof}
%\comment{\bf \Large	Needs to be written }.
The example is almost similar to the example in \cite{KurpiszMMMVW16} which was used to show integrality gap examples for strong LP relaxations for identical machines makespan minimization problem.
We just sketch a proof here. Recall the Petersen Graph with $10$ nodes and $15$ edges which has the following key property: it has six perfect matchings $M_1,\ldots,M_6$ such that each edge $(i,j)$ appears in exactly $2$ of these matchings; however, its edge set cannot be partitioned into $3$ perfect matchings.
The vertices are numbered $0,1,\ldots,9$.

Now we can describe the instance. Fix $k$ to be any positive integer.
We have $15$ types of jobs $p_{ij} = 2^i + 2^j$ for every edge $(i,j)$ of the Petersen graph.
We have $3k$ machines each with the same demand $D = \sum_{i=0}^9 2^i = 1023$.
Consider the six supply vectors $s^{(t)}$ for $1\leq t\leq 6$,  which contains $3k$ copies of the job corresponding to edge $(i,j)$ iff $(i,j)$ is in the matching $M_t$.
These are feasible supply vectors; indeed assign each of the $3k$ machines one jobs $p_{ij}$ for $(i,j) \in M_t$. Now any convex set (in particular polyhedra) containing these six supply vectors
must contain any convex combination. However the vector $\frac{1}{6}\sum_{t=1}^6 s^{(t)}$ is an integer vector with $k$ copies of each $(i,j)$ for all edges of the Petersen Graph.
This uses the fact that every edge is in exactly two perfect matchings. Since the edges of the  Petersen graph can't be partitioned into $3$ perfect matchings, any allocation of this supply vector
must give one machine demand $\leq 1022$. Therefore, there can't be any $\alpha$-approximate supply polyhedra for $P||C_{max}$.
\end{proof}
\begin{remark}\emph{
At this point, we should underscore the difference between supply polyhedra and say LP relaxations for solving  these allocation problems.
Given an instance of say $Q||C_{max}$ {\em along with} the supply vector (which is one standard way the problems are stated), there does exist a polytope capturing all the feasible allocations. It is the integer hull.
However, in general, the description of this integer hull uses the supply vector in describing these constraints and therefore are non-linear when the supplies are variables. Nevertheless, as we discuss below, many LP relaxations
studied in the literature imply supply polyhedra, and their integrality gaps imply the approximation factor for the polyhedra as well.
}
\end{remark}

For our purposes, we need more technical conditions from the supply polyhedra. The first is a natural condition which states that if one moves the supply to higher capacity jobs, then feasibility remains.
The second is related to polynomial time algorithms.

\begin{definition}
	A supply polyhedron $\calP(\calI)$ is {\em upward-feasible} if the following condition holds.
Reorder the jobs so that $c_1\le c_2 \le \cdots \le c_n$.
		If $(s_1,\ldots,s_n)\in \calP$ and $(t_1,\ldots,t_n)$ is a non-negative vector satisfying $t\succeq_\suff s$, that is, $\sum_{k\geq i} t_k \geq \sum_{k\geq i} s_k$, then $(t_1,\ldots,t_n)\in \calP$ as well.
\end{definition}
\begin{definition}[$\gamma$-Approximate Separation.]
A $\gamma$-approximate separation oracle for the supply polyhedron $\calP(\calI)$ is a polynomial time procedure which 	given any $y\in \R^n_{\geq 0}$,  either returns a hyperplane separating $y$ from $\calP$, or asserts that
		$y\in \calP(\calI')$ for the supply polyhedra of the instance $\calI'$ where all demands have been reduced by a factor $\gamma$.
%	\end{asparaitem}
\end{definition}
%Finally, note that if given
\subsection{Approximate Supply Polyhedra for $Q||C_{min}$}
\def\pv{\mathbf{b}}
%
%Let the instance $\calI$ of $Q||C_{min}$  have $m$ machines $M$ with demands $D_1 \geq \cdots \geq D_m$ and $n$ types of  jobs $J$ with capacities $c_1 \geq \cdots \geq c_n$.
%A supply vector $(s_1,\ldots,s_n)$ indicates the number of jobs of each type available; a supply vector is feasible if together they can satisfy all the demands.
%We wish to find a convex set/polyhedra which captures all the feasible supply vectors. In particular, any feasible supply vector should be in the set, and given any (integer) supply vector in the set
%there should be an allocation which satisfies the demands to an $\alpha$-factor.
%\smallskip
%
\noindent
For $Q||C_{min}$, the following assignment LP acts as a good supply polyhedra.
%A feasible supply vector $(s_1,\ldots,s_n)$ must lie in the following polytope.
%We assume we have a guess $D$ for the optimum value which is certified by a feasible solution to the following assignment LP. Below, $D_i := Ds_i$.
\begin{alignat}{4}
	\calP_\mathsf{ass}(\calI)  && = \{(s_1,\ldots,s_n):  && \notag \\
	&& \quad \forall j \in J,   &\quad  \textstyle \sum_{i\in M} z_{ij}  \leq  s_j \label{eq:asslp1} \tag{A1} \\
	&& \quad \forall i\in M ,  &\quad  \textstyle \sum_{j\in J}  z_{ij}  \min(c_j,D_i) \geq D_i \label{eq:asslp2} \tag{A2}\\
	&& \quad \forall i\in M, j\in J, & \quad z_{ij}   \geq 0 \label{eq:asslp3}  \} \tag{A3}
\end{alignat}
\begin{theorem}\label{thm:asslp}
	For any instance $\calI$ of $Q||C_{min}$, $\calP_\mathsf{ass}(\calI)$ is an upward feasible, $2$-approximate supply polyhedron with exact separation oracle.
%	Given $(s_1,\ldots,s_n) \in \calP_\mathsf{ass}$, there is an of assignment $\phi$ of the $s_j$ jobs of capacity $c_j$  to the machines such that for all $i\in M$,
%	$\sum_{j:\phi(j) = i} c_j \geq D_i/2$.
\end{theorem}
\noindent
We defer the proof of the above theorem to Section~\ref{sec:asslp}.

\subsection{Approximate Supply Polyhedra for $Q|f_i|C_{min}$}
For \cckp, a candidate supply polyhedra would be \eqref{eq:asslp1}-\eqref{eq:asslp3} along with
\begin{equation}
\textstyle \forall i\in M, \qquad \sum_{j\in J} z_{ij} ~\leq~ f_i \tag{A4} \label{eq:asslp4}
\end{equation}
which would enforce the cardinality constraint. Unfortunately, an example akin to that in Remark~\ref{rem:ig} shows that $\calP_\mathsf{ass}$ is not
an $\alpha$-approximate supply polyhedron for \cckp instances with $\alpha = o(n)$. We define a stronger supply polyhedron.
However, at this juncture we state a theorem regarding \eqref{eq:asslp1}-\eqref{eq:asslp4} which 
is based on the ideas from Shmoys-Tardos rounding~\cite{ShmoysT93}. 
%we will use at least twice.
\begin{theorem}[Shmoys-Tardos Rounding]
	\label{thm:shmoystardos}
	Let \eqref{eq:asslp1}-\eqref{eq:asslp4} have a feasible solution $z$ and let $C_i := \max_{j:z_{ij}>0} c_j$. 
There is an integral assignment $\zz_{ij}\in \{0,1\}$ which satisfies \eqref{eq:asslp1}, \eqref{eq:asslp4}, and \eqref{eq:asslp2} with the RHS replaced by $D_i - C_i$ for all $i$.
\end{theorem}
	\begin{proof} (Sketch) We proceed as in  the Shmoys-Tardos rounding~\cite{ShmoysT93} of the assignment LP. 
	We convert the instance into a bipartite matching instance where on one side we have the jobs $J$ with
		multiplicities $s_1,\ldots,s_n$, and on the other side we have the machines where we take $f'_i := \ceil{\sum_{j\in J} z_{ij}} \le f_i$ copies of each machine.
		The solution $z$ is converted to a (fractional) solution $\z$ on this bipartite graph where each job $j$ 
		is assigned  by $\z$ to an extent of at most $1$. Furthermore, for every machine $i$, each of its $f'_i$ copies, except perhaps for the last one, 
		 gets $\z$-mass exactly $1$.  This assignment also has the property that for any machine $i$ and job $j$, 
		if  $\z$ assigns (fractionally) $j$ to $\ell^{th}$ copy of $i$, then $c_j$ is at least the total fractional demand assigned by $z$ to 
		the $(\ell+1)^{th}$ copy of this machine. 
		 %and for any $\ell$, 
		 %the $\ell$th copy any job $j$ which gives $\z$-mass to this has $c_j$ more than the total fractional contribution to the $\ell+1$th copy.
		Since each copy of a machine (except for the last copy) gets $\z$-mass exactly $1$, there is an assignment of jobs to these copies such that each such copy of machine $i$ gets exactly one job. We give machine $i$ whatever its copies obtain; note that it obtains $f'_i \leq f_i$ jobs.
		The total capacity of jobs allocated is therefore $\ge D_i - \Delta$ where $\Delta$ is the fractional capacity assigned to $i$'s first copy. Since $\Delta\leq C_1$,  this proves the theorem.
%		
%		{\large NEEDS BETTER WRITING} We repeat the argument of Shmoys and Tardos~\cite{bibid}.
%	Form $\floor{\sum_{j\in J} z_{ij}} \le f_i$ copies of every machine; let $N_i$ be the copies of machine $i$. Order the jobs with multiplicities s.t. $c_1 \geq c_2 \geq \cdots \geq c_N$ where $N = \sum_j s_j$.
%	Modify $z_{ij}$ to get an assigment $z_{ij}$ for $i\in \cup N_i$ and $j\in [N]$ as follows. We do this for one machine $i$.
%	
%	Given $z_{ij}$'s we form $|N_i| + 1$ groups $S_1,\ldots, S_{|N_i|},S_{|N_i|+1}$ with $\sum_{j\in S_t} z_{ij} =1$ for all $1\leq t\leq |N_i|$ and $\sum_{j\in S_t} z_{ij} < 1$ for $t = |N_i|+1$.
%	Note that $\sum_{j\in S_t} z_{ij}c_j < c_{j'}$ for $j'\in S_{t-1}$. When we do this modification for all machines, we get a fractional matching solution where all the $N_i$ copies get fractional value $1$ but the jobs are at most $1$.
%	So, there is an integral matching. The total integral load on machine $i$ is at least $\sum_{t>1} \sum_{j\in S_t} z_{ij}c_j \geq D_i - C_i$ since $z_{ij} = 0$ for $c_j > C_i$.
%		
%		Cardinality constraint vacuously satisfied.
	\end{proof}
	

\noindent
In other words, for instances where $c_j$'s are $\ll D_i$'s, $\calP_\mathsf{ass}$ is a good supply polyhedron. But in general we need a supply polyhedra with stronger constraints. \smallskip

Let $\Supp$ be a multiset indicating infinitely many copies of jobs in $J$.
For every machine $i$, let $\calF_i :=\{S\in \Supp: |S| \leq f_i ~\textrm{ and } \sum_{j\in S} c_j \geq D_i\}$ denote all the feasible sets that can satisfy machine $i$.
Let $n(S,j)$ denote the number of copies of job of type $j$.
\begin{alignat}{4}
	\calP_\mathsf{conf}(\calI) && = \{(s_1,\ldots,s_n):  && \notag \\
	&& \quad \forall i \in M,   &\quad  \textstyle \sum_{S} z(i,S)  =  1 \label{eq:conflp1} \tag{C1} \\
	&& \quad \forall j\in J ,  &\quad  \textstyle \sum_{i\in M,S}  z(i,S)n(S,j) \le  s_j \label{eq:conflp2}\tag{C2} \\
	&& \quad \forall i\in M, S\notin \calF_i, & \quad z(i,S)  = 0 \label{eq:conflp3} \} \tag{C3}
\end{alignat}
\begin{theorem}\label{thm:conflp}
	For any instance $\calI$ of \cckp, $\calP_\mathsf{conf}(\calI)$ is an upward feasible, $O(\log D)$-approximate supply polyhedron with $(1+\epsilon)$-approximate separation oracle for any $\eps > 0$,
	where $D := D_{\mathrm{max}}/D_\mathrm{min}$.
%	Given $(s_1,\ldots,s_n) \in \calP_\mathsf{conf}$ for an instance $\calI$ of $Q|f_i|C_{min}$, there is an of assignment of the $s_j$ jobs of capacity $c_j$  to the machines such that for all $i\in M$
%	receives a total capapcity $\geq D_i/\alpha$ for $\alpha = O(\log D)$ where $D = D_{max}/D_{min}$.
\end{theorem}
\noindent
As a corollary, we get  \Cref{thm:cckp}.
We complement this with an almost matching integrality gap.
\begin{theorem}
	\label{thm:conf-ig}
	The integrality gap of $\calP_\mathsf{conf}$ is $\Omega\left(\frac{\log n}{\log\log n}\right)$. More precisely, there exists an instance $\calI$ of \cckp and a supply vector $(s_1,\ldots,s_n) \in \calP_\mathsf{conf}(\calI)$, but in any feasible allocation of $s_j$ jobs of capacity $c_j$ to the machines, there exists some machine  $i$ receiving $\leq O\left( D_i\frac{\log\log n}{\log n}\right)$.
\end{theorem}
We defer the proofs of \Cref{thm:conflp} and \Cref{thm:conf-ig} to Section~\ref{sec:conflp}.
