\newpage
\section{Max-Min Allocation Problems and Supply Polyhedra}\label{sec:regiongrowing}
An instance of the non-uniform max-min allocation problem $Q||C_{min}$, is described  $m$ machines $M$ with demands $D_1,\ldots,D_m$ and $n$  types of jobs $J$ with capacities $c_1,\ldots,c_n$. 
In the version with cardinality constraints, that is $Q|f_i|C_{min}$ we are also given positive integers $f_1,\ldots, f_m$.

A {\em supply vector} $(s_1,\ldots,s_n)$ where each $s_j$ is a non-negative integer
is called {\em feasible} for instances of these problems if the ensemble formed by $s_j$ copies of jobs of capacity $c_j$ can be allocated feasibly to satisfy all the demands.
The {\em supply polyhedra} of these instances desire to capture these feasible supply vectors.

\begin{definition}[Supply Polyhedron]
	Given an instance $\calI$ for a max-min allocation problem, a polyhedron $\calP(\calI)$ is called an $\alpha$-approximate supply polyhedron if 
	(a) all feasible supply vectors lie in $\calP(\calI)$, and (b) given any non-negative integer vector $(s_1,\ldots,s_n)\in \calP(\calI)$ there exists an assignment
	of the $s_j$ jobs of capacity $c_j$ to the machines such that machine $i$ receives a total capacity of $\geq D_i/\alpha$.
\end{definition}

Ideally, we would like {\em exactly} supply polyhedra. One guess would be the convex hull of all the supply vectors; indeed this is the tightest polytope satisfying condition (a).
Unfortunately, there are instances of $Q||C_{min}$ (and even for the uniform case $P||C_{min}$), the convex hull of supply vectors contain infeasible integer points. This instance is
motivated by integrality gap examples for machine scheduling~\cite{bibid}.
\begin{theorem}
	There cannot exist $\alpha$-approximate supply polyhedra (or convex sets) for $\alpha < 1.001$ for all  $Q||C_{min}$ instances.
\end{theorem}
\begin{proof}
\comment{	Needs to be written }.
\end{proof}
\begin{remark}\emph{
At this point, we should underscore the difference between supply polyhedra and say LP relaxations for solving  these allocation problems.
Given an instance of say $Q||C_{max}$ {\em along with} the supply vector (which is one standard way the problems are stated), there does exist a polytope capturing all the feasible allocations. It is the integer hull.
However, in general, the description of this integer hull uses the supply vector in describing these constraints and therefore are non-linear when the supplies are variables. Nevertheless, as we discuss below, many LP relaxations
studied in the literature imply supply polyhedra, and their integrality gaps imply the approximation factor for the polyhedra as well.
}
\end{remark}

For our purposes, we need more technical conditions from the supply polyhedra. The first is a natural condition which states if one moves the supply to higher capacity jobs, then feasibility remains.
The second is related to polynomial time algorithms.

\begin{definition}
	An supply polyhedra $\calP(\calI)$ for is said to be {\em useful} if it satisfies the following two conditions.
	\begin{asparaitem}
		\item[\emph{(Upward Feasibility.)}] Reorder the jobs so that $c_1\le c_2 \le \cdots \le c_n$.
		If $(s_1,\ldots,s_n)\in \calP$ and $(\bar{s}_1,\ldots,\bar{s}_n)$ is a vector satisfying $\sum_{k\geq i} \bar{s}_k \geq \sum_{k\geq i} s_k$, then $(\bar{s}_1,\ldots,\bar{s}_n)\in \calP$ as well.
		\item[\emph{($\beta$-Approximate Separation.)}] Given any $y\in \R^n_{\geq 0}$,  there is a polynomial time procedure which either returns a hyperplane separating $y$ from $\calP$, or asserts that 
		$y\in \calP(\calI')$ for the supply polyhedra of the instance $\calI'$ where all demands have been reduced by a factor $\beta$.
	\end{asparaitem}
\end{definition}

\begin{itemize}[noitemsep]
	\item Re-state natural LP for \mckc.
	\item Supply Polyhedra for $Q||C_{min}$.
	\item Supply Polyhedra for $Q|f_i|C_{min}$.
	\item Connection to Integrality Gaps ... how $\alpha$-appx supply polyhedra imply $\alpha$-approximation.
	\item Rounding Assignment  for Small Jobs.
\end{itemize}