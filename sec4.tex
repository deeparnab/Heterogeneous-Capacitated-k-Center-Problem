
\section{$(O(1), O(1))$-approximation algorithm}
%Of course, we will not be able to get such a clean decomposition into a \cckp instance and a collection of locally roundable pieces. Indeed, as we iteratively identify pieces which are either locally roundable or get added to the \cckp instance, the neighborhood structure of the remaining clients and facilities changes, and what may have been originally a locally roundable neighborhood piece may cease to be one subsequently. Our final algorithm then combine the above two ideas in a dynamic manner: we iteratively decompose the clients and facilities into three parts: the first being clients which are supported in the LP solution by locally-roundable neighborhoods, along with the facilities in these neighborhoods, the second is the clients and facilities which are isolated from each other and hence form the \cckp instance, and a third kind of clients (which we refer to as the deleted clients), which are in the proximity of the clients of the first two kinds, and their total demand can be charged to the demands in the first two kinds.
%In fact, it is only the second kind of clients and facilities which causes the natural LP to have an unbounded integrality gap, and forces us to use a stronger configuration LP.
In this section, we use the decomposition given by Theorem~\ref{thm:decomp} to obtain a rounding algorithm for the~\mckc problem. Given 
a fractional solution $(x,y)$ to the LP relaxation specified by constraints~(\ref{eq:lp1})--(\ref{eq:lp6}), we can use Theorem~\ref{thm:decomp} to partition 
the clients and the locations into parts which are either  locally roundable or complete neighborhoods. By definition, we can integrally open facilities
in locally roundable subsets. However, rounding complete neighborhoods turns out to be much more trickier. It is the presence of these kind of clients
and facilities which cause the natural LP to have an unbounded integrality gap, and forces us to use a stronger configuration LP.  For the clients
and the facilities in complete neighborhoods, we reduce the rounding problem to an instance of the \cckp problem. It turns out that one needs
to work with a configuration LP even for the \cckp problem. We show in Section~\ref{} how to round a fractional solution to the configuration LP 
relaxation for this problem. However, the LP relaxation for the~\mckc problem gets more complex because we do not know beforehand the 
clients and the faciltiies which will be used in the reduction to the~\cckp problem. Therefore, our configuration LP needs to have one constraint
for {\em every} such subset of clients and facilities. For each such subset, we will be able to extract a solution to the corresponding LP relaxation 
for the~\cckp problem. 

We now describe 
the configuration LP, show how it can be solved and then we give the rounding procedure. 


\subsection{Configuration LP}
\label{sec:lp}
We assume without loss of generality that we know the $\opt$ value (we can perform binary search over $\opt$ value). By scaling, we can also assume that
 $\opt = 1$. Recall that we can define a bipartite graph $G$ over the set of clients locations and facility locations, and we have an edge between
 two locations if the distance between them is at most 1. We know that there is feasible solution which assigns each client to a location in its neighborhood.
 We now describe the LP relaxation. It will seem that it has exponential number of constraints and variables. But we will worry about solving the LP in polynomial time in a latter section. For the time being, we will assume that a feasible solution consisting of variables
 %The following LP has a feasible solution
 $(y_{ip}, x_{ijp})$  can be found.
 %
 %For the time being, we assume this can be found.
  The variable $y_{ip}$ is 1 if we open a
center of capacity $c_p$ at $i$, 0 otherwise. Similarly, the variable $x_{ijp}$ is 1 if the client at location $j$ is assigned to a center of capacity $c_p$ located at $i$, 0 otherwise. The constraints~\eqref{eq:mkc1-full}-\eqref{eq:mkc3-full} are self explanatory. Constraint~\eqref{eq:mkc1-full} states that each client must be assigned to an open location, constraint~\eqref{eq:mkc2-full} requires that a center of capacity $c_p$ can satisfy a total of at most
$c_p$ demands. Constraint~\eqref{eq:mkc3-full} is simply stating that we can assign a demand to a center of capacity $c_p$ at location $i$ only if we open such a center at $i$. Constraint~\eqref{eq:mkc4-full} captures the fact that we have limited number of centers of each capacity, i.e., for
every $p$, we have only $\sum_{q \geq p} n_q$ centers of capacity $c_p$ or smaller. Note that we could have written a constraint for each capacity type $p$ separately -- for each type $p$, we only have $n_p$ centers of type $p$. However, it is easier to work with aggregate constraints~\eqref{eq:mkc4-full}. It allows us to open an extra center of smaller capacity at the expense of opening one fewer center of higher capacity.

Now we describe constraint~\eqref{eq:proj}.
Let $\Supp$ denote all ordered tuples of the form
$(n_1', \ldots, n_k')$, where $n_p' \leq n_p$ for $p=1, \ldots, k$. Further, for a such a tuple $S$, let $n(S,p)$ denote $n_p'$.
For a tuple  $S \in \Supp$, we abuse notation and let $|S|$ denote $\sum_p n(S,p)$.
 Consider a subset of clients $J$, and the locations $\Gamma(J)$.
 In any feasible solution, the clients in $J$ must be assigned to facilities opened at locations in $\Gamma(J)$ (recall that $\Gamma(J)$ denotes
 the neighbors of $J$ in the bipartite graph $G$).
 Suppose we open $n_p'$ centers of capacity $c_p$ in $\Gamma(J)$ --- let $S$ denote the corresponding tuple
 in $\Supp$. Clearly, the fact that we can open at most one facility at each location implies that (i) $|S| \leq |\Gamma(J)|$ , and (ii) the total capacity in $S$, i.e.,
 $\sum_p n(S,p) \cdot c_p$, is at least the total demand in $J$. Let $\calF_J$ denote the set of all tuples
 in $\Supp$ which satisfy these two properties. For every subset of clients $J$ and $S \in \calF_J$, we have a variable $z_{S,J}$ which is 1 if and only if
 for all $p$, $1 \leq p \leq k$,
 we open $n(S,p)$ facilities of capacity $c_p$ in $\Gamma(J)$. Now the constraints in~\eqref{eq:conflpnew} should be clear -- for each $p$, the total number of
 facilities of size $c_p$ opened in $\Gamma(J)$ must be equal to $\sum_{S \in \calF_j} n(S,p) z_{S,J}$. Note that we have an exponential number of constraints of type~\eqref{eq:proj}, one for each subset $J$. Further, for each such $J$, we have defined an exponential number of variables. We show in Section~\ref{sec:ellipsoid} how to solve such an LP. The idea behind these constraints is the following: suppose $J_1, \ldots, J_\ell$ are the set of 
 clients in the collection of complete neighborhoods formed by the decomposition obtained by Theorem~\ref{thm:decomp}. Then we would like to 
 use these constraints for $J_1, \ldots, J_\ell$. For each of these subsets $J_i$, we will use the variables $z_{S, J_i}$ to construct a feasible solution 
 to the configuration LP for the~\cckp problem. 




\begin{eqnarray}
\sum_{i\in F} \sum_{p\in [P]}  x_{ijp} \geq 1 & \forall j\in C  \label{eq:mkc1-full} \\
\sum_{j\in C} d_j x_{ijp} \leq c_p y_{ip} & \forall i \in F, \forall p\in [P] \label{eq:mkc2-full} \\
x_{ijp} \leq y_{ip} & \forall i\in F, j\in C,\forall p\in [P] \label{eq:mkc3-full}\\
\sum_{q \geq p} y_{iq}   \leq \sum_{q\geq p} n_q & \forall p\in [t] \label{eq:mkc4-full}\\
\nonumber \\
y \in \calP(J) & \forall J\subseteq C \label{eq:proj}
\end{eqnarray}
\noindent
where
\begin{eqnarray}
\calP(J) := & \{ y_{ip}: \exists z_{S,J}\in [0,1] \textrm{ for all } S\in \calF_J \textrm{ such that }  \notag \\
& \forall p\in [P], ~~ \sum_{i\in \Gamma(J)} y_{ip} \geq \sum_{S\in \calF_{J}} z_{S,J}\cdot n(S,p), ~~~~ \textrm{and}~~~~ \sum_{S\in \calF_{J}} z_{S,J} \geq 1 \} \label{eq:conflpnew}
\end{eqnarray}
where $\calF_{J} := \{S\in \Supp: |S| \leq |\Gamma(J)|, \sum_{p\in S} c_p \geq D(J)\}$ and $n(S,p)$ is the number of copies of  $p$ in $S$. \bigskip


\subsection{Solving the LP: Ellipsoid Method}
\label{sec:ellipsoid}
In this section we show how to solve the LP relaxation given in Section~\ref{sec:lp}. The set of constraints specify a feasible region for the set of variables $(x_{ijp}, y_{ip})$, which is convex. Therefore, we can invoke the ellipsoid algorithm to solve the LP as long as we can find a separating hyperplane for any infeasible solution. So let $(x_{ijp}, y_{ip})$ be a tentative solution. Since the  constraints in~\eqref{eq:mkc1-full}-\eqref{eq:mkc3-full} are polynomial in number, we can check feasibility for these constraints easily. Consider a {\em fixed} set  of clients $J$.
We show how to check the constraints~\eqref{eq:proj} for $J$. We re-write the constraints~\eqref{eq:conflpnew} for $J$ below:
\begin{eqnarray*}
\sum_{S \in \calF_J} z_{S,J} & \geq & 1 \\
\sum_{S\in \calF_{J}} z_{S,J}\cdot n(S,p) &\leq  & \sum_{i\in \Gamma(J)} y_{ip}  \ \ \ \forall p \in [P] \\
z_{S,J} & \geq &  0 \ \ \ \forall S \in \calF_J
\end{eqnarray*}

Note that we do not explicitly need to say $z_{S,J} \leq 1$ as existence of a solution to the above constraints also guarantees another solution
which satisfies $z_{S,J} \leq 1$ for all $S \in \calF_J$. Treating $y_{ip}$ values as constants, we know by Farkas' lemma that either there is a feasible solution to
the above constraints, or there is a feasible solution to the following set of constraints (where the variables are $\alpha_p, p \in [P]$),
which we call dual LP:
 \begin{eqnarray}
\sum_{p \in P} \left( \sum_{i \in \Gamma(J)} y_{ip} \right) \cdot  \alpha_p & > & 1 \\
\label{eq:ellipsoid1}
 \sum_{p \in [P]} n(S,p) \cdot \alpha_p & \leq & 1 \ \ \ \forall S \in \calF_J \\
 \label{eq:ellipsoid2}
\alpha_p & \geq & 0
\end{eqnarray}

Now, if there is no feasible solution to the above set of constraints, then we know that the solution $y \in \calP(J)$. Otherwise, if there is a
feasible solution, then the first constraint  gives a separating hyperplane. Thus, it is enough to find a solution to the above set of constraints, or
declare that it is infeasible. We check the above constraints again by the ellipsoid algorithm. Thus, given $\alpha_p$ values we need to check that
for all $S \in \calF_J$, $\sum_{p \in [P]} n(S,p) \cdot \alpha_p \leq 1$. We can restate the  problem as a general version of the knapsack cover
problem. We are given a knapsack of capacity 1. We are also given $n_p$ items of size $\alpha_p$ and profit $c_p$ each. We would like to 
select at most $|\Gamma(J)|$ items which cover  the knapsack, i.e., whose total $\alpha_p$ values add up to at least $1$, and maximize the 
total profit. If we can get profit at least $D(J)$, we know that there is a violating constraint (given by the set of items $S$)  to the above set 
of constraints; otherwise the values $\alpha_p$ are feasible. This general version of knapsack cover can be solved exactly using dynamic programming. 
Using standard rounding techniques, we can give an FPTAS for this problem (\textcolor{red}{Do we need a reference here?}). Thus, we can give a polynomial
time algorithm which given the values $\alpha_p$ either (i) declares that the dual LP is infeasible (and so, $y \in \calP(J)$), or (ii) declares that 
$\alpha_p$ values are feasible even if  we relax the  set $\calF_J$ to a set $\calF_J^\varepsilon$ defined as : $\{S\in \Supp: |S| \leq |\Gamma(J)|, \sum_{p\in S} c_p \geq (1- \varepsilon) D(J)\}$, where $\varepsilon$ is an arbitrarily small constant. Thus, in terms of the primal LP, 
we get an algorithm which either declares (i) $y \in \calP(J)$, or (ii) finds a separating hyperplane with respect to $\calP(J)$ where we modify $\calF_J$ to 
$\calF_J^\varepsilon$. Running the ellipsoid algorithm (and assuming that the configuration LP is feasible), we can ensure that for a fixed $J$, 
we get a solution $y$ which lies in $\calP(J)$ provided we change $\calF_J$ to $\calF_J^\varepsilon$. 


Now, we need to ensure that $y \in \calP(J)$ for all subsets $J$. This looks difficult, and we observe that we only need to ensure this constraint
for those sets $J$ which belong to the collection of complete neighborhoods. Thus we get the following result. 

\begin{theorem}
\label{thm:ellipsoid}
We can find in polynomial time a solution $(x,y)$ to the configuration LP such that if $J_1, \ldots, J_\ell$ are the clients in the 4-complete neighborhood
given by Theorem~\ref{thm:decomp} for the solution $(x,y)$, then $y \in \calP(J_i)$, for $i=1, \ldots, \ell$ provided we modify the definitions 
$\calF(J_i)$ to $\calF^\varepsilon(J_i)$. 
\end{theorem}

\noindent
When we use the theorem, we will ignore the distinction between $\calF(J_i)$ and $\calF^\varepsilon(J_i)$ as it will only affect the capacity violation
by a factor $(1+\varepsilon)$. 
\subsection{Rounding the Configuration LP}
Given a solution $(x,y)$ to the configuration LP, we now show how to round it to an integral solution. We first use the decomposition algorithm given
by Theorem~\ref{thm:decomp} to get the partitions $\calS$ and $\calT$ respectively. Observe that it is enough to assign the demands in $\Cb \cup \Cbb$.
Any client $j \in \Cd$ will get assigned to the same facility as $\phi(j)$. This will violate the capacity by another factor of 32. 

\noindent
Lemma~\ref{lem:sk} implies that we can assign the clients in $\Cb$ to integrally opened facilities in $\calS$. 
\begin{corollary} \label{cor:combine-local}
We can efficiently find a rounding $Y$ for the family $\calS$ of locally-roundable sets such that
\begin{enumerate}
\item[(i)] for each $j \in \Cb$, the demand $d_j$ is assigned to integrally open facilities at distance at most $20$ from $j$,
\item[(ii)] the capacity of each integrally open facility is violated by a factor of $64$, and
\item[(iii)] for all $p$, $\sum_{i \in L(\calS)} \sum_{q \geq p} Y_{iq} \leq \floor{\frac12 \cdot \sum_{i \in L(\calS)} \sum_{q \geq p} y_{iq}}$.
\end{enumerate}
\end{corollary}
\begin{proof}
We use the rounding $Y:=\y$ given by the proof of Lemma~\ref{lem:sk}.  Statement~(iii) follows directly from this Lemma. Now, we assign clients
in $\Cb$ to facilities in  a set $S_k$ to the extent $\sum_{i \in S_k, p} x_{ijp}$. This can be done provided we violate the capacities of
integrally open facilities by a factor 32. Now the third statement in Theorem~\ref{thm:decomp} shows that we have assigned at least half of 
each client in $\Cb$. Doubling the capacity violation, we can assign all demand in $\Cb$ to open facilities in $\calS$ -- this gives statement~(ii). 
Since the diameter of each 
of the sets $S_k$ is at most 20, the statement~(i) follows. 
\end{proof}

\noindent
Now we show how to assign the clients in $\calT$. 
\begin{lemma}\label{lem:rounding-local-neighborhoods}
Consider the disjoint subsets $T_1,\ldots,T_L \subseteq F$ where each $T_\ell$ is a $4$-complete neighborhood with $J_\ell\subseteq C$ 
being responsible for it. Recall that $L(\calT)$ denotes the set of locations in $T_1 \cup \cdots \cup T_L$. 
%Given any solution $y$ which satisfies \eqref{eq:conflpnew} for the sets $J_\ell, \ell \in [L]$, 
We can find an integral solution $Y$ such that
 (a) $\sum_{q\geq p} \sum_{i\in L(\calT)} Y_{iq} \leq \ceil{\sum_{q\geq p} \sum_{i\in L(\calT)} y_{iq}}$ for all $p\in [P]$, and
 (b) $\sum_{i\in T_\ell,p\in [P]} c_p Y_{ip} \geq D(J_\ell)/8\alpha$, where $\alpha$ is the integrality gap of the configuration LP relaxation for
 the \cckp problem. 
\end{lemma}
\begin{proof}
The proof proceeds in the following manner: we first create an instance $\calI$ of the \cckp problem based on the sets in $\calT$, clients in $\Cbb$, and the LP solution $(x,y)$. Next, we derive a feasible fractional solution to a suitable configuration LP for the instance $\calI$ of the \cckp problem. Finally, we show that any feasible integer solution to the configuration LP for the \cckp instance can be used to find a placement of centers within the
locations in $\calT$ to satisfy all the demand in $\Cbb$.

The instance $\calI$ needs a set of demands $D_1, \ldots, D_m$ with corresponding bounds $f_1, \ldots, f_m$. We also need to specify how
many items of each type are available. Let $n_{\geq p}'$ denote $\ceil{\sum_{q \geq p} \sum_{i \in L(\calT)} y_{ip}}$. These are total
number of available facilities of type $p$ or higher. Instead of specifying the numbers of available items of each type in $\calI$, we will specify
how many items of size $c_p$ or higher are available in the instance $\calI$. Clearly, this does not change the problem specification. 

One natural way of defining $\calI$ is as follows: for each $T_\ell$, we have a demand of requirement $D_\ell$ which is the total requirements
of clients in $J_\ell$. We set $f_\ell$ as $|\Gamma(J_\ell)|$. Finally, for each type $p$, we have $n_{\geq p}'$ items of size $c_p$ or higher. 
Now the configuration LP for the~\cckp instance needs a feasible solution $y(S,\ell)$, where $\ell$ is a demand and $S$ is a feasible assignment 
for this demand. We could naively set $y(S,\ell) := z_{S,J_\ell}$, where $z$ is given by  the constraints~(\ref{eq:conflpnew}). However, there
is a technical problem -- the configuration corresponding to $S$ may not be feasible. For example, it is possible that there is a configuration $S$ 
for which $z_{S, J_\ell} > 0$ and 
$S$ uses more than $n_{\geq p}'$ items of type $p$ or higher. Constraint~(\ref{eq:conflpnew}) only guarantees that the {\em average} number
of items of type $p$ or higher in the solution $z$ would be at most $n_{\geq p}'$. Thus, we need to first massage the solution $z$ so that it is
supported on feasible configurations only. We now show how this can be done. 
\begin{claim}
\label{cl:massage}
Let $S \in \calF(J_\ell)$. Then we can truncate $S$ to a vector $S'$ such that  (i) for all $p \in P$, $\sum_{q \geq p} n(S',p) \leq n_{\geq p}'$, 
(ii) $\sum_p n(S',p) \cdot c_p \geq \ceil{D(J_\ell)/2}$. 
\end{claim}
\begin{proof}
Let $n_{\geq p}$ denote $\sum_{q \geq p} n_q$. Note that Corollary~\ref{cor:combine-local} implies that for each $p \in [P]$, $n_{\geq p}' 
\geq \ceil{n_{\geq p}/2}$. Now we give a procedure for truncating $S$ to $S'$. Arrange the items in $S$ in descending order of $c_p$ values. 
Let this order be $i_1, i_2, \ldots, i_m$, where $m = |S|$ (recall that $|S|$ denotes $\sum_p n(S,p)$). Also observe that the items in $S$
which are of size at least $c_p$ or higher form a prefix of this sequence. Now define $S'$ as the set of items $\{i_1, i_3, i_5, \ldots\}$, i.e., 
the items located at odd indices in this order. Clearly, the total capacity of items in $S'$ is at least $\ceil{D(J_\ell)/2}$. Further, for any $p$, 
 $S'$ has at most $\ceil{n_{\geq p}/2}$ items of size $c_p$ or higher (because $S$ had at most $n_{\geq p}$ such items). 
 The claim follows because the latter quantity is at most $n_{\geq p}'$. 
\end{proof}

We are now ready to define the instance $\calI$. As before, for each $T_\ell \in \calT$, we have a demand $\ell$ of requirement $D_\ell := D(J_\ell)/2$. 
As before, we set $f_\ell := |\Gamma(J_\ell)|$. For each type $p$, we have $n_{\geq p}'$ items of size $c_p$ or higher. Now 
we define a fractional solution to the configuration LP for $\calI$. For each set $T_\ell \in \calT$, consider the set of variables $z_{S, J_\ell}$
corresponding to the constraints~\ref{eq:conflpnew}. Corresponding to each such variable, we define a variable $y(S', \ell) := z_{S, J_\ell}$,
where $S'$ is the truncation of $S$ given by Claim~\ref{cl:massage}. 
Since the total size of items in $S'$ is at least $D_\ell$, we see that 
it is a valid configuration. We now show that it is a feasible solution as well. Constraint~\ref{eq:config1} follows directly from the fact that 
$\sum_{S \in \calF_{J_\ell}} z_{S, J_\ell} \geq 1$. Now we check constraint~\ref{eq:config2}. 

%
%Needs to be written.
%
%%Indeed, in this instance $\calI$, there is one bucket $B_\ell$ corresponding to each $T_\ell \in \calT$. The demand $D_\ell$ of this bucket is precisely $\sum_{j \in J_\ell} d_j$. To determine the number of centers available of each type, we let the LP solution guide us. Indeed, for each type $p$, we set $n_p = \ceil{\sum_{p' \leq p} \sum_{i \in L(\calF)} y_{ip}} - \ceil{\sum_{p' < p} \sum_{i \in L(\calF)} y_{ip}}$. Finally, we set the cardinality constraint $f_\ell$ to be $|T_{\ell}|$.
%%
%%Then, it is easy to see that the fractional configurations opened in $\calP(J_\ell)$ form a feasible fractional solution to the configuration LP of the \cckp instance defined above. Moreover, we can easily map back an integer solution to a placement of facilities to satisfy (a) and (b) above.
\end{proof}

\begin{theorem}
\label{thm:final}
There is a rounding algorithms which opens at most $n_p$ facilities of type $p$ and assigns clients to facilities such  that 
(i) each client $j$ is assigned to facilities which are at distance at most 21 from it , and (ii) the capacity of each facility is violated by a constant factor. 
\end{theorem}
\begin{proof}
The proof follows by combining Corollary~\ref{cor:combine-local} and Lemma~\ref{lem:rounding-local-neighborhoods}, and the fact that 
clients in $\Cd$ can use the mapping $\phi$ given by Theorem~\ref{thm:decomp}. 
\textcolor{red}{Need to figure out the exact capacity violation.}
\end{proof}




