\newcommand{\barcalS}{\bar{\cal S}\xspace}

\section{Special Case: \cckp}
We first consider the following very special case of the \mckc problem where the demands are all isolated from one another, and around each demand $j$ of value $D_j$, there are $f_j$ facility locations in its vicinity. More formally, in this problem, we are given a collection of demand requirements $D_1, D_2, \ldots, D_m$, along with cardinality constraints $f_1, f_2, \ldots, f_m$ for each demand. There are also a collection of items with different capacities: $(n_1,c_1), (n_2,c_2),\ldots, (n_T,c_T)$ with $c_1 \leq c_2 \le \cdots \le c_T$,
to indicate we have $n_t$ items of capacity $c_t$. The goal is to assign the items to these demand requirements, so that (i) any demand $j$ is assigned no more than $f_j$ items, and (ii) the requirement of all demands are satisfied by the items assigned. We assume that there is a feasible assignment which achieves these two goals. In an $\alpha$-approximation, our target is to find an assignment such that all demands are met up to a factor of $\alpha$, i.e., demand $j$ is satisfied to extent $D_j/\alpha$.
Our main result in this section is an LP-rounding constant-factor approximation for \cckp. Before we present the algorithm, we first show why some natural approaches fail, which will then motivate the configuration LP we use, which is by now standard for these kinds of max-min (the so-called santa-claus type) allocation problems.

\subsection{Integrality Gap Examples}
It turns out that several natural LP relaxations have unbounded integrality gaps. For the following examples, we suitably make copies of every item, and hence assume that $n_t = 1$ for all $i$. After making copies thusly, let $I$ denote the set of all items. Further, we will abuse notation and let $c_t$ denote the
size of an item $t \in I$.

We begin with the most natural relaxation, where we have variables $z_{tj}$ for item $t$ and demand $j$, which denotes the fractional assignment of item $t$ to demand $j$. The constraints are as follows:
\begin{align*}
\sum_{i} z_{tj} \min(c_t, D_j) & \geq D_j, \ \  \forall j=1, \ldots, m \\
\sum_i z_{tj} & \leq f_j, \ \ \ \forall j = 1, \ldots, m \\
\sum_j z_{tj} & \leq 1,  \ \ \ \forall t \in I \\
z_{tj} & \geq 0, \ \ \ \ \ \forall t \in I, j = 1, \ldots, m
\end{align*}

Consider the following integrality gap example:~each of the $m$ demands have $D_j = 1$ and $f_j = 2$, for $j=1, \ldots, m$. There are $m-1$ ``large'' items of size $1$ each,
 and $m$ ``small'' items of size $1/m$ each. Notice that in any feasible integral solution, there will be at least one demand $j$ which is assigned only $2$ small items of total size $2/m$. But notice that the following solution is feasible for the above LP --- set $z_{tj} = 1/m (1-1/m)$ for every large item $t$ and demand $j$, and $z_{tj} = 1/m^2$ for every small item $t$ and demand $j$.

To overcome the above bad example, we use the structure of feasible solutions to construct a stronger LP relaxation. Indeed, we know that in any feasible integral solution, every demand receives $j$ at least $D_j/2$ of its requirement from items which are larger than size $\mu_j = \frac{D_j}{2f_j}$. Therefore, in the stronger LP, we say that an item $t$ is \emph{eligible} for demand $j$ (denoted by $t \sim j$) iff $c_t \geq \mu_j$, and then ask for an allocation which satisfies demand $j$'s requirement to an extent of at least $D_j/2$.
\begin{align*}
\sum_{t: t \sim j} z_{tj} \min(c_t, D_j/2) & \geq D_j/2, \ \  \forall j=1, \ldots, m \\
\sum_{t: t \sim j} z_{tj} & \leq f_j, \ \ \ \forall j = 1, \ldots, m \\
\sum_{j: t \sim j} z_{tj} & \leq 1,  \ \ \ \forall t \in I\\
z_{tj} & \geq 0, \ \ \ \ \ \forall i \in I,j=1, \ldots, m
\end{align*}

It turns out that these constraints are also not enough. Here is an integrality gap example (don't remember this one...).

\newcommand{\barD}{{\bar D}}

\subsection{Configuration LP relaxation}
Our final LP relaxation is in fact a by-now standard configuration LP relaxation which is useful for such allocation problems. For the following, let $\Delta$ be a large constant which will determine our final approximation factor $\alpha$. For clarity in presentation, we have not tried to optimize for the exact value of $\alpha$, but we will remark on the limitations of this approach toward the end of this section. {\bf do this!}. We say that an item of type $p$ is {\em large} for demand $j$ if $c_p \geq \frac{D_j}{4\Delta}$, otherwise it is said to be \emph{small}. In the configuration LP, we shall have variables $y(S,j)$ for every demand $j$, and certain subsets $S$ of items (which depend on $j$). 
The subset $S$ will be of two types -- (i) $S$ could be a singleton set containing exactly one large item for demand $j$ --- we call such configurations {\em large}, or (ii) $S$ could be a multi-set of small items, whose total size is at least $D_j$ and whose cardinality is at most $f_j$ --- we call such configurations {\em small}. The collection of all large configurations for a demand $j$ is denoted by $\calS^L_j = \{ \{t\} \text{ s.t } c_t \geq \frac{D_j}{4 \Delta}  \}$. Similarly, the collection of all small configurations for demand $j$ is denoted by $\calS^S_j = \{ ((1,n'_1), (2,n'_2), \ldots, (T,n'_T)) \text{ s.t } \sum_{t \in [T]} c_t n'_t \geq D_j \text{ and } \sum_{t \in [T]} n'_t \leq f_j \}$. For a given small configuration $S \in \calS^S_j$ for demand $j$, we let $n(S,t)$ to denote the number of items of size $c_t$ in this configuration. For technical reasons, we will allow also allow configurations where $n(S,t)$ is larger than $n_t$, the given budget of number of items of type $t$. This will be useful for our overall algorithm for \mckc, where we would not know these budgets up front. With this notation, the LP is now straightforward to specify:
\begin{align}
\label{eq:config1}
\sum_{S \in \calS^L_j \cup \calS^S_j} y(S,j) & \geq 1, \ \ \ \forall j=1, \ldots, m \\
\label{eq:config2}
\sum_{t \in [T]} y(S,j) \, n(S,t) & \leq n_t \ \ \ \forall t  \in I \\
\notag
y({S,j}) & \geq 0
\end{align}
It is easy to see that this is a valid relaxation. In any integral solution, either a demand $j$ is assigned a large item or the multi-set of small items assigned to it is a valid small configuration, i.e., it belongs to $\calS^S_j$.

\subsubsection{Solving the LP: Ellipsoid Method}
The LP relaxation can be solved by the ellipsoid method. The LP has exponential number of variables and polynomial (in fact, linear) number of
constraints. Solving the primal LP  can be reduced to
a separation oracle for the dual LP:
\begin{align}
\label{eq:config3}
 \sum_{j=1}^m \alpha_j - \sum_{i \in I} \beta_i & \leq 0 \\
\label{eq:config4}
\alpha_j & \geq  \beta_i \ \ \ \forall \ {\mbox{large configuration $(\{i\}, j)$}} \\
\label{eq:config5}
\alpha_j & \geq  \sum_{i \in S} \beta_i \ \ \ \forall \ {\mbox{small configuration $(S, j)$}} \\
\notag
\alpha_j, \beta_i & \geq 0
\end{align}


Given a solution $\alpha, \beta$, we need an approximate separation oracle -- if some constraint is violated to an extend of $(1+\varepsilon)$-factor
or more, we need to output such a constraint in polynomial time. It is easy to verify the constraints~(\ref{eq:config3}) and~(\ref{eq:config4}). There
are exponential number of constraints of type~(\ref{eq:config5}). For a fixed $j$, consider the corresponding knapsack cover problem -- we are given
a knapsack of size $D_j$, and items of size $c_i$ (note that we are only interested in items which are small with respect to $j$). Item $i$ has
profit $\beta_i$. We would now like to cover the knapsack with these items (i.e., total size of items must be at least $D_i$) and check if the total
profit can exceed $\alpha_j$. Since there is a PTAS for the knapsack cover problem, we can check if a constraint of type~(\ref{eq:config5})
is violated by more than $(1+\varepsilon)$-factor.

\subsubsection{The Rounding Algorithm}

In this section, we prove the following theorem.
\begin{theorem} \label{thm:round-conf}
Given a feasible fractional solution $\{y(S,j)\}$ to the configuration LP above, we can efficiently round it to obtain an integer solution which is a $100$-approximation for the given \cckp instance.
\end{theorem}


The rounding algorithm requires several steps. We begin with some definitions.
We first round the quantities $c_t$ and $D_j$ down to nearest power of $\Delta$, and let the rounded quantities be $\bc_t$ and $\barD_j$ respectively. 
We also slightly expand the set of small and configurations to satisfy some weaker conditions involving $\bc_t$ and $\barD_j$ as follows: we define a set $\barcalS^L_j$ of large configurations to be $\barcalS^L_j = \{ \{t\} \text{ s.t } \bc_t \geq \frac{\barD_j}{4 \Delta^2}  \}$. Similarly, the expanded collection of all small configurations for demand $j$ is denoted by $\barcalS^S_j = \{ ((1,n'_1), (2,n'_2), \ldots, (T,n'_T)) \text{ s.t } \sum_{t \in [T]} \bc_t n'_t \geq \frac{\barD_j}{\Delta} \text{ and } \sum_{t \in [T]} n'_t \leq f_j \}$. Since we scale down the demands and capacities by a factor of $\Delta$, it is easy to see that $\calS^L_j \subseteq \barcalS^L_j$ and $\calS^S_j \subseteq \barcalS^S_j$.


Our algorithm starts with a feasible fractional solution to the following relaxed configuration LP, and over time modifies the solution to make \emph{most} of the \emph{large configuration assignments} integral $0/1$ while remaining feasible to the relaxed LP. Finally, it rounds the small item types to $0/1$ using the classical algorithm of Lenstra Shmoys and Tardos~\cite{LST}. 

\medskip \noindent {\bf Demand Buckets.}
The first step of our algorithm is to partition the demands into buckets depending on their requirement values $\barD_j$. To this end, we say that demand $j$ is of \emph{class $p$} if
$\barD_j$ is $\Delta^p$ -- we let $B^\brp$ to denote the \emph{bucket} of demands of class $p$. We say that an item of type $t$ is \emph{large} with respect to class $p$ if its size is at least $\frac{\Delta^{p-2}}{4}$, i.e., it can belong to a large configuration for a class $p$ demand.  For a demand $j$, let $\barcalS^L_j$ and $\barcalS^S_j$ denote the large and the small configurations corresponding to $j$ respectively.

We now further define some useful variables which can be derived from $\{y(S,j)\}$. Indeed, fix a solution $y$ to the configuration LP. For a demand $j$, define $y^L(j)$ to be $\sum_{(S,j) \in \barcalS^L_j} y(S,j)$, i.e., the total fractional extent to which $j$ is satisfied by large configurations. Define $y^S(j)$ similarly corresponding to small configurations. Likewise, we extend this definition to buckets of demands as well. For a bucket $p$, define $y^L(B^\brp)$ to be $\sum_{j \in B^\brp} y^L(j)$, and define $y^S(B^\brp)$ analogously. Now, for an \emph{item} of type $t$, define $y(t)$ to be $\sum_{(S,j) : t \in S} n(S,t) y(S,j)$. This is the fractional extent to which it is assigned in the solution $y$. Again, we can extend this definition to large and small configurations, and over different buckets of demands. Given a bucket $p$, let $I^\brp$ denote the item types which are large for class $p$. For an item type $t \in I^\brp$, define $y^L(t,p)$ as $\sum_{(\{t\}, j): j \in B^\brp} y(\{t\}, j)$. This is the extent to which items of type $t$ are assigned as large jobs to demands of bucket $p$.

\medskip \noindent We are now equipped to describe our rounding algorithm. It proceeds in the following steps:



\medskip \noindent
{\bf Step 1: Rounding Large Job Assignments.} We first rearrange the large job assignments such that for each class $p$, there is at most one variable $y_{S,j}$ such that $S \in \barcalS^L_j, j \in B^\brp$, which is strictly between 0 and 1. In other words, there is at most one \emph{strictly fractional} large configuration assigned in any class.

We perform the following steps starting from the lowest class $p$ onwards.
We may remove some demands and items during this process. This will happen if we integrally assign an item to a demand, and we satisfy the requirement of the demand. In this case, we may remove both the demand and the items assigned to this demand. Initially, we have  not removed any demand or
item. Before we start the steps for class $p$, the following invariant holds for every class $p' < p$:

\begin{itemize}
\item[({\bf I1})] After removing the integrally assigned demands (and the corresponding items), for each class $p' < p$, there is at most one  demand $j_{p'} \in B^{(p')}$ such that $y^l(j)$ is positive. Further, if such a demand $j_{p'}$ exists, then (i) $y^l(j_{p'})$ is strictly less than 1, and
(ii) $f_{j_{p'}} \leq f_{j'}$ for any $j' \in B^{(p')}$.
\end{itemize}

Suppose this invariant holds for all classes upto $p-1$. Now we describe the iteration for class $p$.
Define a total order $\prec$ on $B^\brp$, where $j' \prec j$ if $f_{j'} \leq f_{j}$.  We now iteratively transfer mass from large configurations $(S,j) \in \barcalS^L_j$ to $(S',j') \in \cC^l(j')$, where $j' \prec j$. Suppose $j,j' \in B^\brp$ such that the following conditions hold: (i) $j' \prec j$, (ii) $y^l(j') < 1, y^l(j) > 0$. We modify $y$ such that
$y^l(j')$ will increase and $y^l(j)$ will decrease. Let $(S,j)$ be a large configuration with $y(S,j) > 0$. Since $y^l(j') < 1$, we know that there is a small configuration $(T,j')$ with $y(T,j') > 0$. We perform the following changes continuously at the same rate: increase $y(S,j')$ and $y(T,j)$ and decrease $y(S,j)$ and $y(T,j')$. We stop when one of the variables becomes 0. Note that $(S,j')$ is also a large configuration because $j,j'$ are of the same class. Further, $(T,j)$ is a small configuration because $|T| \leq f_{j'} \leq f_j$. It is easy to see that this process maintains feasibility of the LP solution. We keep on performing this process as long as possible --- since we are always transferring large configuration assignments towards jobs which appear earlier in the total order, this process will stop. Further, if $\alpha$ denotes $y^l(B^\brp)$, then the following condition holds when the process stops: arrange the jobs in $B^\brp$ according to the order $\prec$. Let this ordering be $j_1, j_2, \ldots, j_r$. Let $k$ denote $\lfloor \alpha \rfloor$. Then $y^l(j_u) = 1$ for $u=1, \ldots, k$ and $y^l(j_u) = 0$ for $u > k+1$. $y^l(j_{k+1})$ is equal to the fractional part of $\alpha$.

After the above step, we have ensured that there is at most one demand (i.e., $j_{k+1}$) of class $p$
with strictly fractional $y^l(j)$ value in the modified LP solution. However, even the demands with $y^l(j) = 1$, i.e.,  $j_1, \ldots, j_k$,  may currently be satisfied by many large configurations. Now we iteratively perform a second transformation to ensure that each such demand is in fact integrally satisfied by a single large item, and hence we can remove such demands and the corresponding large item, and proceed.

Indeed, consider a  job $j \in B^\brp$ for which there are two large configurations, say $(\{i_1\}, j), (\{i_2\}, j)$ with positive $y$ assignments. Suppose $c_{i_1} < c_{i_2}$. Let $(T,j')$ be another configuration containing $i_1$ with $y(T,j') > 0$ (if such a configuration exists). Note that $j'$ may not be of class $p$. Let $T'$ denote the set obtained from $T$ be adding $i_2$ and removing $i_1$. Since $i_1$ has smaller size than $i_2$, the total size of items in $T'$ will exceed that of $T$. So $(T',j')$ will be a valid configuration. It is possible that $(T,j')$ was a small configuration, but $i_2$ is a large job for $j'$. For this to happen $j'$ must belong to a class higher than $p$. In this case we should replace $(T',j')$ by the large configuration $(\{i_2\},j')$ in the discussion below. %Since we have not processed demands of class larger than $p$ yet, invariant~[(I1)] continues to hold.

%If $(T,j')$ was a large configuration, then it is possible that $j'$ is of class less than $p$. In this case it will be the unique demand
% of this class guaranteed by invariant~[(a)]. Further $(T',j')$ will also be a large configuration, and so, the invariant continues to hold if we raise $y(T',j')$.

We change the following variables at the same rate: increase $y(\{i_1\},j)$ and $y(T',j')$ and decrease $y(\{i_2\},j)$ and $y(T,j')$. Again, it is easy to check the feasibility of LP solution. We stop when one of the variables becomes 0, and repeat the process. We claim that the invariant~[(I1)] continues
to hold (for $p-1$) during this process. Indeed, we only need to be worried for the case when $j'$ is of class less than $p$.  In this case it will be the unique demand  of this class guaranteed by invariant~[(a)]. Further $(T',j')$ will also be a large configuration. Therefore, $y^l(j')$ does not change, and so,
the invariant still holds.

When the process ends, all demands $j \in B^\brp$  except perhaps for one demand satisfy the property that $y^l(j)$ is either 0 or 1. Further, for any demand $j$, there is at most one large configuration $(S,j)$ with positive $y(S,j)$ value. If a large configuration $(\{i\},j)$ satisfies $y(\{i\},j)=1$, then we assign $i$ to $j$. There will be no other non-zero variable  $y(S,j)$ involving $i$. So, we can remove $i$ and $j$ from the instance. Thus, for each class $p$, there is at most one large configuration $(S,j), j \in B^\brp,$  such that $y(S,j)$ strictly between 0 and 1.  Thus, we  satisfy the invariant for class $p$ as well. Observe that right now all the large assignment of such a demand $j$ is satisfied by just one large configuration. However, when we carry out the
above steps for a class larger than $p$, we may redistribute the large assignment of $j$ over several large configurations, but $y^l(j)$ will remain unchanged.


We will now show that it is possible to use only small configurations to satisfy all the demands.
For each class $p$, let $j_p$ be the {\em special} demand given by invariant~[(a)]. Let $F$ denote the set of special demands.

\medskip \noindent
{\bf Step 2: Modifying Small Configurations.} For each special demand $j_p \in F$, we know that $y^s(j_p) > 0$. Let $S_p$ be a small configuration such that $y^s(j_p, S_p) > 0$ (actually any small configuration for $j_p$ will do -- we just need to use the fact that there exists one).
 Since all items in $S_p$ are of size at most $\frac{C^p}{4C}$, we know that the size of $S_p$ is at most $C^p + C^{p-1}/4 \leq 2C^p$. We now make the sets $S_p$ mutually disjoint. For each $p$, define $S_p'$ as $S_p \setminus (S_1 \cup \ldots \cup S_{p-1})$. Clearly, size of $S_p'$ is still at least $C^p - 2\sum_{p' < p} C^{p'} \geq C^p/2. $

We define a new LP solution ${\bar y}$.
For each special demand $j_p \in F$,
we set ${\bar y}(S_p',j_p) = 0.5$, and ${\bar y}(S,j_p) = 0$ for all other configurations. Further, for any demand $j$ not in $F$, we set ${\bar y}(S,j) =  y(S,j)/2$. The solution ${\bar y}$ satisfies the demand $\barD_j$ of every demand $j$ to an extent of at least $\barD_j/4$ using small configurations only.  Although we can now invoke the rounding algorithm of Lenstra et al.~\cite{}.

\begin{lemma}
\label{lem:lst}
Suppose there is a fractional solution to the configuration LP which assigns $\barD_j/4$ amount to each demand $j$ using small configurations only.
Then there is an integral solution which satisfies each demand to an extent of at least $\barD_j/8$.
\end{lemma}
\begin{proof}
Needs to be written.
\end{proof}

Thus we get the following result.
\begin{theorem}
\label{thm:config}
Given a solution $y$ to the configuration LP, there is an integral assignment of items to demands which satisfies each demand $j$ to the extent of at least
$D_j/128$.
\end{theorem}
\begin{proof}
The demands which get removed during the above rounding algorithm are assigned items to the extend of $\barD_j$. The remaining demands
get items adding up to at least $\barD_j/8$ (Lemma~\ref{lem:lst}). Since $\barD_j \geq D_j/C$ and we had rounded item sizes up to powers of $C$ as
well, we see that each demand $j$ is assigned items of total size at least $D_j/4C^2$. Take $C=4$ to get the desired result.
(Do we need to round item sizes?? -- may be we can save a $C$ factor).
\end{proof}
