\newcommand{\barcalS}{\bar{\cal S}\xspace}
\newcommand{\brt}{{(t)}}
%\newcommand{\calR}{\cal{R}\xspace}

\section{Special Case: \cckp}
We first consider the following very special case of the \mckc problem where the demands are all isolated from one another, and around each demand $j$ of value $D_j$, there are $f_j$ facility locations in its vicinity. More formally, in this problem, we are given a collection of demand requirements $D_1, D_2, \ldots, D_m$, along with cardinality constraints $f_1, f_2, \ldots, f_m$ for each demand. There are also a collection of items with different capacities: $(n_1,c_1), (n_2,c_2),\ldots, (n_P,c_P)$ with $c_1 \leq c_2 \le \cdots \le c_P$,
to indicate we have $n_p$ items of capacity $c_p$. The goal is to assign the items to these demand requirements, so that (i) any demand $j$ is assigned no more than $f_j$ items, and (ii) the requirement of all demands are satisfied by the items assigned. We assume that there is a feasible assignment which achieves these two goals. In an $\alpha$-approximation, our target is to find an assignment such that all demands are met up to a factor of $\alpha$, i.e., demand $j$ is satisfied to extent $D_j/\alpha$.
Our main result in this section is an LP-rounding constant-factor approximation for \cckp. Before we present the algorithm, we first show why some natural approaches fail, which will then motivate the configuration LP we use, which is by now standard for these kinds of max-min (the so-called santa-claus type) allocation problems.

\subsection{Integrality Gap Examples}
It turns out that several natural LP relaxations have unbounded integrality gaps. For the following examples, we suitably make copies of every item, and hence assume that $n_p = 1$ for all $p$. After making copies thusly, let $I$ denote the set of all items. Further, we will abuse notation and let $c_p$ denote the
size of an item $p \in I$.

We begin with the most natural relaxation, where we have variables $z_{pj}$ for item $p$ and demand $j$, which denotes the fractional assignment of item $p$ to demand $j$. The constraints are as follows:
\begin{align*}
\sum_{p} z_{pj} \min(c_p, D_j) & \geq D_j, \ \  \forall j=1, \ldots, m \\
\sum_p z_{pj} & \leq f_j, \ \ \ \forall j = 1, \ldots, m \\
\sum_j z_{pj} & \leq 1,  \ \ \ \forall p \in I \\
z_{pj} & \geq 0, \ \ \ \ \ \forall p \in I, j = 1, \ldots, m
\end{align*}

Consider the following integrality gap example:~each of the $m$ demands have $D_j = 1$ and $f_j = 2$, for $j=1, \ldots, m$. There are $m-1$ ``large'' items of size $1$ each,
 and $m$ ``small'' items of size $1/m$ each. Notice that in any feasible integral solution, there will be at least one demand $j$ which is assigned only $2$ small items of total size $2/m$. But notice that the following solution is feasible for the above LP --- set $z_{pj} = 1/m (1-1/m)$ for every large item $t$ and demand $j$, and $z_{pj} = 1/m^2$ for every small item $t$ and demand $j$.

To overcome the above bad example, we use the structure of feasible solutions to construct a stronger LP relaxation. Indeed, we know that in any feasible integral solution, every demand receives $j$ at least $D_j/2$ of its requirement from items which are larger than size $\mu_j = \frac{D_j}{2f_j}$. Therefore, in the stronger LP, we say that an item $p$ is \emph{eligible} for demand $j$ (denoted by $p \sim j$) iff $c_p \geq \mu_j$, and then ask for an allocation which satisfies demand $j$'s requirement to an extent of at least $D_j/2$.
\begin{align*}
\sum_{p: p \sim j} z_{pj} \min(c_p, D_j/2) & \geq D_j/2, \ \  \forall j=1, \ldots, m \\
\sum_{p: p \sim j} z_{pj} & \leq f_j, \ \ \ \forall j = 1, \ldots, m \\
\sum_{j: p \sim j} z_{pj} & \leq 1,  \ \ \ \forall p \in I\\
z_{pj} & \geq 0, \ \ \ \ \ \forall i \in I,j=1, \ldots, m
\end{align*}

It turns out that these constraints are also not enough. Here is an integrality gap example (don't remember this one...).

\newcommand{\barD}{{\bar D}}

\subsection{Configuration LP relaxation}
Our final LP relaxation is in fact a by-now standard configuration LP relaxation which is useful for such allocation problems. In the section, we consider the problem in full generality and consider instances where there are $n_p$ items of capacity $c_p$ for $1 \leq p \leq P$. For the following, let $\Delta$ be a large constant which will determine our final approximation factor $\alpha$. For clarity in presentation, we have not tried to optimize for the exact value of $\alpha$, but we will remark on the limitations of this approach toward the end of this section. {\bf do this!}. We say that an item of type $p$ is {\em large} for demand $j$ if $c_p \geq \frac{D_j}{4\Delta}$, otherwise it is said to be \emph{small}. In the configuration LP, we shall have variables $y(S,j)$ for every demand $j$, and certain subsets $S$ of items (which depend on $j$).
The subset $S$ will be of two types -- (i) $S$ could be a singleton set containing exactly one large item for demand $j$ --- we call such configurations {\em large}, or (ii) $S$ could be a multi-set of small items, whose total size is at least $D_j$ and whose cardinality is at most $f_j$ --- we call such configurations {\em small}. The collection of all large configurations for a demand $j$ is denoted by $\calS^L_j = \{ \{p\} \text{ s.t } c_p \geq \frac{D_j}{4 \Delta}  \}$. Similarly, the collection of all small configurations for demand $j$ is denoted by $\calS^S_j = \{ ((1,n'_1), (2,n'_2), \ldots, (P,n'_P)) \text{ s.t } \sum_{p \in [T]} c_p n'_p \geq D_j \text{ and } \sum_{p \in [P]} n'_p \leq f_j \text{ and } n'_p \leq n_p \text{ for all } 1 \leq p \leq P \}$. For a given small configuration $S \in \calS^S_j$ for demand $j$, we let $n(S,p)$ to denote the number of items of size $c_p$ in this configuration. With this notation, the LP is now straightforward to specify:
\begin{align}
\label{eq:config1}
\sum_{S \in \calS^L_j \cup \calS^S_j} y(S,j) & \geq 1, \ \ \ \forall j=1, \ldots, m \\
\label{eq:config2}
\sum_{j \in [m], S \in \calS^L_j \cup \calS^S_j} y(S,j) \, n(S,p) & \leq n_p, \ \ \ \forall t  \in I \\
\notag
y({S,j}) & \geq 0
\end{align}
It is easy to see that this is a valid relaxation. In any integral solution, either a demand $j$ is assigned a large item or the multi-set of small items assigned to it is a valid small configuration, i.e., it belongs to $\calS^S_j$.

\subsubsection{Solving the LP: Ellipsoid Method}
{\bf RKNOTE: Do this.}

The LP relaxation can be solved by the ellipsoid method. The LP has exponential number of variables and polynomial (in fact, linear) number of
constraints. Solving the primal LP  can be reduced to
a separation oracle for the dual LP:
\begin{align}
\label{eq:config3}
 \sum_{j=1}^m \alpha_j - \sum_{i \in I} \beta_i & \leq 0 \\
\label{eq:config4}
\alpha_j & \geq  \beta_i \ \ \ \forall \ {\mbox{large configuration $(\{i\}, j)$}} \\
\label{eq:config5}
\alpha_j & \geq  \sum_{i \in S} \beta_i \ \ \ \forall \ {\mbox{small configuration $(S, j)$}} \\
\notag
\alpha_j, \beta_i & \geq 0
\end{align}


Given a solution $\alpha, \beta$, we need an approximate separation oracle -- if some constraint is violated to an extend of $(1+\varepsilon)$-factor
or more, we need to output such a constraint in polynomial time. It is easy to verify the constraints~(\ref{eq:config3}) and~(\ref{eq:config4}). There
are exponential number of constraints of type~(\ref{eq:config5}). For a fixed $j$, consider the corresponding knapsack cover problem -- we are given
a knapsack of size $D_j$, and items of size $c_i$ (note that we are only interested in items which are small with respect to $j$). Item $i$ has
profit $\beta_i$. We would now like to cover the knapsack with these items (i.e., total size of items must be at least $D_i$) and check if the total
profit can exceed $\alpha_j$. Since there is a PTAS for the knapsack cover problem, we can check if a constraint of type~(\ref{eq:config5})
is violated by more than $(1+\varepsilon)$-factor.

\subsubsection{The Rounding Algorithm}

In this section, we prove the following theorem.
\begin{theorem} \label{thm:round-conf}
Given a feasible fractional solution $\{y(S,j)\}$ to the configuration LP above, we can efficiently round it to obtain an integer solution which is a $100$-approximation for the given \cckp instance.
\end{theorem}


Our algorithm starts with a feasible fractional solution to the configuration LP, and over time modifies the solution to make \emph{most} of the \emph{large configuration assignments} integral $0/1$ while remaining feasible to a slightly \emph{relaxed} configuration LP. Finally, it rounds the small item types to $0/1$ using the classical algorithm of Lenstra Shmoys and Tardos~\cite{LST}. Overall, the rounding algorithm requires several steps which we next present one by one.

\medskip \noindent {\bf Step 1: Rounding Sizes and Demand Requirements.}
We first round the quantities $c_p$ and $D_j$ down to nearest power of $\Delta$, and let the rounded quantities be $\bc_p$ and $\barD_j$ respectively.
We also slightly expand the set of small and configurations to satisfy some weaker conditions involving $\bc_p$ and $\barD_j$ as follows: we define a set $\barcalS^L_j$ of \emph{relaxed} large configurations to be $\barcalS^L_j = \{ \{p\} \text{ s.t } \bc_p \geq \frac{\barD_j}{4 \Delta^2}  \}$. Similarly, the expanded collection of \emph{relaxed} small configurations for demand $j$ is denoted by $\barcalS^S_j = \{ ((1,n'_1), (2,n'_2), \ldots, (P,n'_P)) \text{ s.t } \sum_{p \in [P]} \bc_p n'_p \geq \frac{\barD_j}{\Delta} \text{ and } \sum_{p \in [P]} n'_p \leq f_j \text{ and } n'_p \leq n_p \text{ for all } 1 \leq p \leq P \}$.

\begin{claim}
The relaxed large and small configurations satisfy $\calS^L_j \subseteq \barcalS^L_j$ and $\calS^S_j \subseteq \barcalS^S_j$.
\end{claim}

\begin{proof}
Since we scale down the demands and capacities by a factor of $\Delta$, it is easy to see.
\end{proof}

\begin{corollary}
The solution $\{y(S,j)\}$ is feasible to the relaxed configuration LP where we allow relaxed large and small configurations.
\end{corollary}


\medskip \noindent {\bf Step 2: ``Bucketing'' Demand Requirements.}
The next step of our algorithm is to partition the demands into buckets depending on their requirement values $\barD_j$. To this end, we say that demand $j$ belongs to \emph{bucket $t$} if
$\barD_j$ is $\Delta^t$, and let $B^\brt$ to denote the bucket of all such demands $\{j \text{ s.t } \barD_j = \Delta^t\}$. We say that an item type $p$ is \emph{large} with respect to bucket $t$ if its size $\bc_p$ is at least $\frac{\Delta^{t-2}}{4}$, i.e., it can belong to a relaxed large configuration for a bucket-$t$ demand.  %For a demand $j$, let $\barcalS^L_j$ and $\barcalS^S_j$ denote the large and the small configurations corresponding to $j$ respectively.

We now further define some useful variables which can be derived from $\{y(S,j)\}$. Indeed, fix a solution $y$ to the relaxed configuration LP. For a demand $j$, define $y^L(j)$ to be $\sum_{(S,j) \in \barcalS^L_j} y(S,j)$, i.e., the total fractional extent to which $j$ is satisfied by large configurations. Define $y^S(j)$ similarly corresponding to small configurations. Likewise, we extend this definition to buckets of demands as well. For a bucket $t$, define $y^L(B^\brt)$ to be $\sum_{j \in B^\brt} y^L(j)$, and define $y^S(B^\brt)$ analogously. Now, for a \emph{type $p$} of items, define $y(p)$ to be $\sum_{(S,j)} n(S,p) y(S,j)$, the fractional extent to which this type is assigned across all items in the solution $y$. From the LP feasibility constraint, we know that $y(p) \leq n_p$. Finally, given a bucket $t$, let $I^\brt$ denote the set of item types which are large for bucket $t$. For a bucket $t$ and item type $p \in I^\brt$, define $y^L(p,t)$ as $\sum_{(\{p\}, j): j \in B^\brt} y(\{p\}, j)$, which is the extent to which items of type $p$ are assigned as large demands to demands in bucket $t$.

\medskip \noindent
{\bf Step 3: Rounding Large Demand Assignments.} In this step, we modify the large demand assignments in the LP solution $\{y(S,j)\}$ such that for each bucket $t$, there is \emph{at most one strictly fractional variable} $0 < y_{S,j} < 1$ over all $j \in B^\brt$ and $S \in \barcalS^L_j$. To this end, we repeatedly perform the following steps, starting from the lowest bucket $t$ onwards. As we make some assignments integral, we say that all demands $j$ for which there exists $S \in \barcalS^L_j \cup \barcalS^S_j$ such that $y(S,j)=1$ belong to a set $\calR$, the set of \emph{rounded demands}. Before we start the steps for bucket $t$, we always maintain the following invariant holds for every bucket $t' < t$ (which is vacuously true for the lowest bucket):
\begin{framed}
\begin{itemize}
\item[({\bf I})] For each bucket $t' < t$, there is at most one  demand $j_{t'} \in B^{(t')} \setminus \calR$ such that $y^l(j_{t'})$ is in $(0,1)$. Further, if such a demand $j_{p'}$ exists, then $f_{j_{t'}} \leq f_{j'}$ for any $j' \in B^{(t')} \setminus \calR$.
\end{itemize}
\end{framed}

Suppose this invariant holds for all buckets upto $t-1$. Now we describe the iteration for bucket $t$.

\medskip \noindent {\bf Step 3a: Intra-Bucket Rounding.}  Define a total order $\prec$ on $B^\brt \setminus \calR$, where $j \prec j'$ if $f_{j} \leq f_{j'}$.  We now ensure that the demands with the smallest $f_j$ value get preference when it comes to being satisfied by large configurations. Indeed, suppose there exists $j,j' \in B^\brt \setminus \calR$ such that the following conditions hold: (i) $j \prec j'$, (ii) $y^l(j) < 1, y^l(j') > 0$. We then \emph{modify} $y$ such that
$y^l(j)$ will increase and $y^l(j')$ will decrease until either the former hits $1$ or the latter hits $0$. To this end, let $(S,j') \in \barcalS^L_{j'}$ be a large configuration  with $y(S,j') > 0$. Now, since $y^l(j) < 1$, we know that there is a small configuration $(T,j) \in \barcalS^S_j$ with $y(T,j) > 0$. We then perform the following changes continuously at a uniform rate: increase $y(S,j)$ and $y(T,j')$ and decrease $y(S,j')$ and $y(T,j)$. We stop when one of the variables becomes 0 or 1.

\begin{claim} \label{cl:swap1}
The modified solution above is feasible for the relaxed configuration LP.
\end{claim}
\begin{proof}
To see why the modified solution is feasible for the relaxed configuration LP, observe that $(S,j)$ is also a valid relaxed large configuration because $j$ and $j'$ are of the same bucket $t$ and hence have the same rounded demand requirement of $\Delta^t$. Similarly, observe that $(T,j')$ is a valid small configuration for $j'$ because $|T| \leq f_{j} \leq f_{j'}$, and the remaining constraints for being a small configuration (i.e., the total capacity of items in $T$ should exceed the $\barD_{j'}/\Delta$, and the numbers $n(T,p) \leq n_p$ for all types) are satisfied trivially since they enforce the same constraints for $j$ and $j'$. As a result, since we increase $y(S,j)$ and decrease $y(T,j)$ at the same rate, constraint~\cref{eq:config1} is satisfied for $j$. Similarly, since we increase $y(T,j')$ and decrease $y(S,j')$ at the same rate, constraint~\cref{eq:config1} is satisfied for $j'$. Likewise, since we increase $y(S,j)$ and decrease $y(S,j')$ at the same rate, constraint~\cref{eq:config2} is satisfied for the item type corresponding to the large item in $S$. Finally, since we increase $y(T,j')$ and decrease $y(T,j)$ at the same rate, constraints~\cref{eq:config2} are satisfied for all item types corresponding to items in $T$.
\end{proof}

We keep on performing this process as long as possible --- since we always transfer large configuration assignments to demands which appear earlier in the total order, this process will stop at some point. At this point, we add whichever demands are integrally assigned by large configurations to the set $\calR$, i.e., all $j \in B^\brt$ for which there exists $S \in \barcalS^L_j \cup \barcalS^S_j$ such that $y(S,j) = 1$, are added to $\calR$.

\begin{claim} \label{cl:step3a}
 Let $\alpha := y^l(B^\brt \setminus \calR)$ denote the total fractional assignment of large configurations to the non-integrally rounded demands in $B^\brt$. then the following condition holds at the termination of the above iterative process: suppose we arrange the demands in $B^\brt \setminus \calR$ according to the order $\prec$. Let this ordering be $j_1, j_2, \ldots, j_r$, and let $k$ denote $\lfloor \alpha \rfloor$. Then we have that $y^l(j_u) = 1$ for $u=1, \ldots, k$, $y^l(j_u) = 0$ for $u > k+1$, and lastly, $y^l(j_{k+1})$ is equal to the fractional part of $\alpha = \alpha - \floor{\alpha}$.
 \end{claim}

 \begin{proof}
 Complete...
 \end{proof}

\begin{claim}
After step (3a), the invariant (I) continues to hold for all buckets $t' < t$.
\end{claim}

\begin{proof}
This is true because we do not alter the assignments in buckets $p' < p$ in the step.
\end{proof}


To summarize, after the above step, we have ensured that there is at most one demand (i.e., $j_{k+1}$ from the claim above) of bucket $t$
with strictly fractional $y^l(j)$ value in the modified LP solution. However, even the demands with $y^l(j) = 1$, i.e.,  $j_1, \ldots, j_k$,  may currently be satisfied by many large configurations. In the next step of the algorithm, we iteratively perform a \emph{second transformation} to ensure that each such demand is in fact \emph{integrally satisfied by a single large item}, and hence we can add such demands to $\calR$ and proceed.



\medskip \noindent {\bf Step 3b: Inter-Bucket Rounding.}
We repeat this step while there  exists a  demand $j \in B^\brt$ for which there are two large configurations, say $(\{p_1\}, j), (\{p_2\}, j)$ with positive $y$ assignments, and suppose $\bc_{p_1} < \bc_{p_2}$ without loss of generality. There are now two cases depending on whether item type $p_1$ is contained in any other non-integrally assigned configuration or not.

\medskip \noindent {\bf Case 3b(i): There exists demand $j' \notin \calR$ and configuration $(T,j')$ such that $n(T,p_1) \geq 1$ and $y(T,j') > 0$.}
In this case, note that $j'$ may not belong to bucket $t$. Now, if item type $p_2$ is a small item type for demand $j'$, then let $T'$ denote the multi-set obtained from $T$ be adding one item of type $p_2$ and removing one item of type $p_1$. Since $p_1$ has smaller size than $p_2$, clearly the total size of items in $T'$ will exceed that of $T$. On the other hand, if $p_2$ is a large item type for $j'$, then let $T' = \{p_2\}$ and we use the large configuration of $(T',j')$. In both cases, we perform the following updates at a uniform rate: increase $y(\{p_1\},j)$ and $y(T',j')$ and decrease $y(\{p_2\},j)$ and $y(T,j')$, and stop when one of them reaches $0$ or $1$. If in the process some demand is integrally assigned, we include it in $\calR$.

\medskip \noindent {\bf Case 3b(ii): There exists no demand $j \notin \calR$ and configuration $(T,j')$ such that $n(T,p_1) \geq 1$ and $y(T,j') > 0$.}
In this case, we have at least $1$ free capacity of item type $p_1$, and so we set $y(\{p_1\},j)= 1$ and all other $y(S,j) = 0$ for $S \neq \{i_1\}$. We add $j$ to the rounded set $\calR$ and repeat this step (3b).


\begin{claim} \label{cl:step3b}
If $\{y(S,j)\}$ is a feasible to the relaxed configuration LP, then after a single execution of either case 3b(i) or case 3b(ii), the modified solution remains feasible for the relaxed configuration LP, and moreover, invariant~(I) continues holds for all buckets $\leq t-1$.
\end{claim}
\begin{proof}
Again, it is easy to check the feasibility of LP solution much like~\Cref{cl:swap1}. We also claim that the invariant~[(I1)] continues to hold (for bucket $t-1$) during this process. Indeed, we only need to be worried for the case when $j'$ is of bucket less than $t$, as otherwise we are not modifying the configurations in lower buckets. In this case it will be the unique fractional demand of this bucket guaranteed by invariant~[I1]. Further $(T',j')$ will also be a large configuration for this demand $j'$. Therefore, $y^l(j')$ does not change, and so, the invariant still holds.
\end{proof}

  To summarize, when the process ends, all demands $j \in B^\brt$  except perhaps for one demand satisfy the property that $y^l(j)$ is either 0 or 1. Further, for any demand $j$, there is at most one large configuration $(S,j)$ with positive $y(S,j)$ value. If a large configuration $(\{i\},j)$ satisfies $y(\{i\},j)=1$, then we include $j$ in $\calR$. Thus, for bucket $t$, there is at most one large configuration $(S,j), j \in B^\brt \setminus \calR$  such that $y(S,j)$ strictly between $0$ and $1$. Moreover, since this fractional demand is the demand $j_{k+1}$ identified in Claim~\ref{cl:step3a}, it has the smallest $f$ among all $j \in B^\brt \setminus \calR$. Thus, we satisfy the invariant for bucket $t$ as well. Observe that right there is exactly one large configuration with positive $y$ value serving $j_k$. However, when we carry out the step (3b) for a bucket larger than $t$, we may redistribute the large assignment of $j$ over several large configurations, but $y^l(j)$ will remain unchanged.


