
\medskip \noindent {\bf Step 4: Rounding all remaining demands $j \notin \calR$ using small configurations.}
In the final part of the proof, we will snow that it is in fact possible \emph{to use only small configurations} to satisfy all the unrounded demands $j \notin \calR$.
To this end, for each bucket $t$, let $j_t$ be the {\em unique} demand which with $0 < y^l(j_t) < 1$ given by invariant~[(I)], and let $\calD_s$ denote the set of all these unique demands in all buckets. Now, for each unique demand $j_t \in \calD_s$, we know that $y^s(j_t) > 0$, since $y^l(j_t) < 1$. Then, let $S_t$ be a small configuration such that $y^s(S_t,j_t) > 0$ (actually any small configuration for $j_t$ will do -- we just need to use the fact that there exists one). Since all items in $S_t$ are of (rounded) size at most $\frac{\Delta^{t-2}}{4}$, we know that the total (rounded) size of $S_t$ is at most $\Delta^{t-1} + \Delta^{t-2}/4 \leq 2 \Delta^{t-1}$.

The main idea now is to actually make the sets $S_t$ \emph{mutually disjoint}. For each $t$, let $S_t'$ to be $S_t \setminus (S_1 \cup \ldots \cup S_{t-1})$. Clearly, total size of $S_t'$ is still at least $\Delta^{t-1} - 2\sum_{t' < t} \Delta^{t'-1} \geq \Delta^{t-1}/2$.


We define a new LP solution ${\bar y}$.
For each special demand $j_p \in F$,
we set ${\bar y}(S_p',j_p) = 0.5$, and ${\bar y}(S,j_p) = 0$ for all other configurations. Further, for any demand $j$ not in $F$, we set ${\bar y}(S,j) =  y(S,j)/2$. The solution ${\bar y}$ satisfies the demand $\barD_j$ of every demand $j$ to an extent of at least $\barD_j/4$ using small configurations only.  Although we can now invoke the rounding algorithm of Lenstra et al.~\cite{}.

\begin{lemma}
\label{lem:lst}
Suppose there is a fractional solution to the configuration LP which assigns $\barD_j/4$ amount to each demand $j$ using small configurations only.
Then there is an integral solution which satisfies each demand to an extent of at least $\barD_j/8$.
\end{lemma}
\begin{proof}
Needs to be written.
\end{proof}

Thus we get the following result.
\begin{theorem}
\label{thm:config}
Given a solution $y$ to the configuration LP, there is an integral assignment of items to demands which satisfies each demand $j$ to the extent of at least
$D_j/128$.
\end{theorem}
\begin{proof}
The demands which get removed during the above rounding algorithm are assigned items to the extend of $\barD_j$. The remaining demands
get items adding up to at least $\barD_j/8$ (Lemma~\ref{lem:lst}). Since $\barD_j \geq D_j/C$ and we had rounded item sizes up to powers of $C$ as
well, we see that each demand $j$ is assigned items of total size at least $D_j/4C^2$. Take $C=4$ to get the desired result.
(Do we need to round item sizes?? -- may be we can save a $C$ factor).
\end{proof}
